%------------------------------------------------------------------------------
% Template file for the submission of articles to IUCr journals in LaTeX2e
% using the iucrjournals document class (file iucrjournals.cls)
% This work has been dedicated to the public domain
% License: CC0 1.0 Universal
% https://creativecommons.org/publicdomain/zero/1.0/
%------------------------------------------------------------------------------
% This template file and associated class and style files produce documents in
% a preprint style suitable for submission and review purposes.
% The iucrjournals.cls requires a small selection of packages from standard TeXLive
% distributions and contains a minimal set of macros to define content and apply
% formatting. BibTeX and iucr.bst should be used for references (using harvard.sty).
% If you wish to use additional packages, please reference them in this document and
% please only use packages included in standard TeXLive distributions in order to
% avoid compilation problems during the submission process.
%------------------------------------------------------------------------------

\documentclass{iucrjournals}

% Add extra packages here, e.g.
% \usepackage{myfavouritepackage}
\usepackage{amssymb}
\usepackage[fleqn]{amsmath}

\usepackage{booktabs}
\usepackage[table]{xcolor}
%\usepackage[table]{xcolor}
\usepackage{algorithm}
\usepackage{algpseudocode}
\usepackage{subcaption}
\usepackage{caption}

%internal refs with macros, for consistency:
\newcommand{\labsec}[1]{\label{sec:#1}}
\newcommand{\refsecnumonly}[1]{\ref{sec:#1}}
\newcommand{\refsec}[1]{Section~\refsecnumonly{#1}}
\newcommand{\Refsec}[1]{Section~\refsecnumonly{#1}}%same as \refsec, but always capitalised!
\newcommand{\reftwosections}[2]{Sections~\refsecnumonly{#1} and \refsecnumonly{#2}}
\newcommand{\refthreesections}[3]{Sections~\refsecnumonly{#1}, \refsecnumonly{#2}, and \refsecnumonly{#3}}
\newcommand{\refsectionrange}[2]{Sections~\refsecnumonly{#1}--\refsecnumonly{#2}}
\newcommand{\refappendix}[1]{\refsecnumonly{#1}}

\newcommand{\labfig}[1]{\label{fig:#1}}
\newcommand{\reffignumonly}[1]{\ref{fig:#1}}
\newcommand{\reffig}[1]{Figure~\reffignumonly{#1}}
\newcommand{\Reffig}[1]{Figure~\reffignumonly{#1}}%same as \reffig, but always capitalised!
\newcommand{\refthreefigs}[3]{Figures~\reffignumonly{#1}, \reffignumonly{#2}, and \reffignumonly{#3}}
\newcommand{\reftwofigures}[2]{Figures~\reffignumonly{#1} and \reffignumonly{#2}}
\newcommand{\reffigrange}[2]{Figures~\reffignumonly{#1}--\reffignumonly{#2}}
\newcommand{\refsubfigfrommaincaption}[1]{Figure~\protect\subref{fig:#1}}
\newcommand{\Refsubfigfrommaincaption}[1]{Figure~\protect\subref{fig:#1}}%same, but always capitalized!
\newcommand{\reftwosubfigsfrommaincaption}[2]{Figures~\protect\subref{fig:#1} and ~\protect\subref{fig:#2}}
% \protect\subref{fig:lcbragg_singlenormal_thetan60_decomposed_noanti}

\newcommand{\labtab}[1]{\label{tab:#1}}
\newcommand{\reftabnumonly}[1]{\ref{tab:#1}}
\newcommand{\reftab}[1]{Table~\reftabnumonly{#1}}
\newcommand{\Reftab}[1]{Table~\reftabnumonly{#1}}%same as \reftab, but always capitalised!
%%%%\newcommand{\lablisting}[1]{\label{lst:#1}}
%%%%\newcommand{\reflistingnumonly}[1]{\ref{lst:#1}}
%%%%\newcommand{\reflisting}[1]{Listing~\reflistingnumonly{#1}}
%%%%\newcommand{\Reflisting}[1]{Listing~\reflistingnumonly{#1}}%same as \reflisting, but always capitalised!
\newcommand{\labeqn}[1]{\label{eqn:#1}}
\newcommand{\refeqnnumonly}[1]{\ref{eqn:#1}}
\newcommand{\refeqn}[1]{Eq.~\refeqnnumonly{#1}}
\newcommand{\Refeqn}[1]{Eq.~\refeqnnumonly{#1}}%same as \refeqn, but always capitalised!
\newcommand{\reftwoeqns}[2]{Eqs.~\refeqnnumonly{#1} and \refeqnnumonly{#2}}
\def\LRA{\Leftrightarrow}
\def\RA{\Rightarrow}

\title{Revisiting Becker-Coppens (1974): Updated Recipes for Estimating Extinction Factors in Spherical Crystallites}

% Authors and affiliations (uses the standard authblk package):
% Author affiliations are indicated by lowercase letters in square brackets in the \author macro.
% Affiliations (referenced by the lowercase letters in square brackets) are listed after all the authors have been defined.
% The email addresses of corresponding/contact authors can be included using:
% \IUCrCemaillink{corrauthor@org.org}
% Other co-author email addresses can be included using:
% \IUCrEmaillink{coauthor@org.org}
% ORCiDs can be included using:
% \IUCrOrcidlink{xxxx-xxxx-xxxx-xxxx}
% Author footnotes can be included using:
% \IUCrAufn{Text...}
% and to apply the same footnote to another author use:
% \IUCrAufn[1]{}
% where the number in square brackets refers to the numerical order of the
% previously defined footnote.
% For example:
% \author[a]{Anne Author\IUCrCemaillink{corrauthor@org.org}\IUCrOrcidlink{xxxx-xxxx-xxxx-xxxx}}


\author[a]{Thomas Kittelmann\IUCrCemaillink{thomas.kittelmann@ess.eu}\IUCrOrcidlink{0000-0002-7396-4922}}%
\author[b]{Secundus Segunda\IUCrEmaillink{coauthor@org.org}\IUCrOrcidlink{xxxx-xxxx-xxxx-xxxx}\IUCrAufn{Unique note.}}%
\author[a,b]{Trinity Terzi\IUCrEmaillink{anothercoauthor1@org.org}\IUCrOrcidlink{xxxx-xxxx-xxxx-xxxx}\IUCrAufn{Shared note.}}%
\author[a,b]{Clover Dufour\IUCrEmaillink{anothercoauthor2@org.org}\IUCrAufn[2]{}}

\affil[a]{ESS DMSC, European Spallation Source ERIC, Denmark}
\affil[b]{Different Department, Different Organization, ..., Country }

\begin{document}
\maketitle

\begin{synopsis}
A technical correction and enhancement of the Becker-Coppens (1974) recipe for
estimating extinction factors in spherical crystallites is provided, addressing
limitations in scenarios with higher Bragg angles as well as providing improved
precision in general. The work utilises modern computing capabilities and
improved recipes are provided for easy community adoption.
\end{synopsis}

\begin{abstract}
We present a technical correction and update of the widely used recipes by
Becker-Coppens (1974), for the estimation of primary and secondary extinction
factors in perfect spherical domains or particles. In the original publication,
these extinction factors were evaluated numerically from a complicated integral,
and simplified analytical approximations to these numerical evaluations
constituted the provided recipes. However, the original recipes are plagued by
issues of numerical precision in general, and even suffer from complete
numerical break-down in case of strong primary extinction effects in
back-scattering scenarios ($\theta_\texttt{Bragg}\gtrsim50^\circ$). Using modern
computing capabilities, the numerical evaluations of the integrals are
revisited, and improved recipes are provided with precision guarantees for all
Bragg angles and levels of extinction. The new recipes are provided both in a
``standard'' version, believed to be of suitable precision for any actual
analysis of diffraction or transmission data, and a ``luxury'' reference version
with even higher precision.

The performance of the new recipes is naturally compared to those of the
original work, and in order to facilitate easy adoption by the community,
reference implementations are provided for \texttt{C}, \texttt{C++}, and
\texttt{Python} language.
\end{abstract}

\keywords{extinction; Bragg diffraction; neutron scattering; X-ray diffraction; }

\section{The Becker-Coppens recipes for extinction}

Extinction in the context of diffraction refers to the phenomenon where the
intensity of a diffracted wave is reduced or diminished due to interference
effects. The exact nature of these interference effects might be complicated,
but at least for the case of {{\em} secondary} extinction effects, this can
usually be explained by the particle undergoing multiple Bragg scattering's on
the same reflection plane. Each scattering event will alternate the neutron's
orientation between the diffracted and undiffracted directions, with the net
effect being that the diffracted flux is diminished compared to what one would
expect  if multiple scattering effects were not possible. The extinction factor,
$y$, is defined as the difference between the diffracted flux predicted by the
kinematical model ($I_\text{kin}$) with no multiple scattering or other dynamic
effects, and the diffracted flux which is actually observed once these effects
are accounted for ($I_\text{obs}$):
\begin{align}
y \equiv \frac{I_\text{obs}}{I_\text{kin}}
\end{align}
The extinction factor $y$, is a real number in the unit interval which in
general will depend on both the state of the particle undergoing diffraction as
well as of the details of the material in which the process takes place.  In all
cases, $y$ is expressed as a function of both the Bragg angle $\theta$ as well
as a parameter $x$ which capture the general strength of extinction effects in
the material. The exact definition of $x$ depends to some degree on the specific
model, but in general it will be proportional to $(Fl\lambda/V)^2$ where $F$ is
the structure factor of the crystal reflection in question, $V$ the volume of
the crystal unit cell, $\lambda$ the wavelength of the particle being scattered,
and $l$ some relevant length scale representing the path length of the probe
through the coherent domains, crystallites, or grains in which the diffraction
is taking place. In the case where the extinction is governed by a mosaic
distribution, the definition of $x$ will usually also depend on parameters of
this mosaic distribution, such as in particular its width.

Several models for $y$ has been proposed over the years, with early work by
Zachariasen \cite{zachariasen1967} improved upon by other authors like
Cooper-Rouse \cite{cooper1970extinction}, Becker-Coppens \cite{BC1974}, and
Sabine \cite{sabine2006flow}. Of these, the Becker-Coppens model is arguably the
more highly referenced and utilised work, so it was naturally considered by the
authors when working on implementing extinction models in \texttt{NCrystal}
\cite{ncrystalmain,ncrystalelastic} in order to facilitate the modelling and
analysis of Bragg edge transmission and diffraction data. (TODO: Cite Be
extinction paper here?  But only if it also considered BC and not just
Sabine?). Unfortunately, in case of stronger extinction effects, the model
appears to suffer from an obvious breakdown at large Bragg angles, as can be
seen in \reffig{ypbreakdownorigrecipe}.

\begin{figure}[ht] %
\begin{center}
\includegraphics[width=0.6\textwidth]{../data/ncrystal_bc_yp_breakdown_be.pdf}
\end{center}
\caption{Bla. Beryllium.} % NB \protect\cite{...} is required in floating figures
\labfig{ypbreakdownorigrecipe}
\end{figure}

As will be shown in the present work, this breakdown does not constitute a
fundamental problem with the Becker-Coppens model, but is merely a result of
flawed numerical evaluations entering into the recipes (algorithms) provided in
that work. It is the intent of the present work to provide updated recipes which
does not change the underlying theory or assumptions going ito the Becker-Coppens
model, but merely ensures a consistent and precise numerical evaluation of
it. These new recipes for evaluation of $y$ will be referred to as BC2025 in the
present work, while the recipes from the original work will be referred to as
BC1974. It will be seen that the $y$ values provided by the BC2025 recipes do
not drastically disagree agree with those of the BC1974 recipes in the regions
of intermediate extinction levels, where the model has arguably seen most usage
and validation over the years in X-ray or neutron diffraction.

In the work by Becker-Coppens \cite{BC1974}, four specific models and recipes
are provided for the determination of $y$: one for primary extinction, and three
for secondary extinction, where the three latter models only differs in the
choice of distribution used to describe the mosaic spread of coherent domains
within the crystalline materials: Gaussian, Lorentzian, or Fresnellian.

\subsection{Mathematical formulation of the models}

Readers are referred to the original work \cite{BC1974} for the detailed
description and background of the models, while the present work will focus
exclusively on the mathematical expressions from the models which must be
evaluated. To determine the extinction parameter $y$ as a function of $x$ and
$\theta$, the following integral must be evaluated:
\begin{align}
  y_M(\theta,x) = \frac{6c_M}{4\pi}\int_{0}^{\infty} f_M(\eta)\varphi\left(\theta,c_Mxf_M(\eta)\right)d\eta
  \labeqn{yintegral}
\end{align}
Here, the subscript $M$ indicates the model, and can either be $P$
(primary), $G$ (secondary Gaussian), $L$ (secondary Lorentzian), or $F$
(secondary Fresnellian). The variable $c_M$ is a constant, with model dependent
values of $c_P=c_G=c_F=1$ and $c_L=4/3$.

TODO?: Note that the definition of $x$ itself also depends on the model, but we will
drop the index on $x$ for clarity.
TODO? (note we have used that $f(\eta)$ is an even function to change the integration
domain to $[0,\infty)$).


The function $\varphi(\theta,s)$ is an extinction correction function given by:
\begin{align}
  \varphi(\theta,s) = \varphi^0(s) + \sin^{3/2}\theta\left\{\varphi^\pi\left(s\sqrt{\sin\theta}\right)-\varphi^0\left(s\sqrt{\sin\theta}\right)\right\}
  \labeqn{phi}
\end{align}
Here $\varphi^0(s)$ and $\varphi^\pi(s)$ are the limiting functions for $\theta=0$
and $\theta=\pi/2$ respectively, and are given as:\footnote{Note that the expression for
$\varphi^\pi(s)$ in eq. 32 of \cite{BC1974} has a mistake, where a factor of $1/2$
should be moved from the second to the third term.}
\begin{align}
  \varphi^0(s) &= \frac{3}{64s^3}\left\{8s^2+4s\exp(-4s)-(1-\exp(-4s))\right\}\\
  \varphi^\pi(s) &= \frac{3}{4s^3}\left\{s^2-s+\frac{1}{2}\log(1+2s)\right\}
  \labeqn{phizeroandphipi}
\end{align}
All the $\varphi$--functions are even functions of $s$, and as can be seen in
\reffig{phiofs}, monotonically decreasing functions of $|s|$, attaining
values $1$ at $s=0$ and $0$ at $s=\pm\infty$.
\begin{figure}[ht] %
\begin{center}
\includegraphics[width=0.6\textwidth]{generated/phi_of_s.pdf}
\end{center}
\caption{Bla.} % NB \protect\cite{...} is required in floating figures
\labfig{phiofs}
\end{figure}

The $f_M(\eta)$ functions, are essentially unit-less functions capturing the
shape of the Bragg diffraction cross section as a function of deviation from the
nominal direction, normalised so $f_M(0)=1$. The difference in extinction levels
predicted by the different Becker-Coppens models are essentially due to the
difference of these $f_M(\eta)$ functions. For the secondary extinction models,
the shape of $f_M(\eta)$ are given by the mosaic distribution of coherent
crystal domains within material particles, while for primary extinction it is
due to the intrinsic response of a crystalline domain which is perfect but of
finite size. Additionally, the exact final form of the $f_M(\eta)$ functions are
given based on spherical geometries for the crystallites or particles. The exact
form of these functions, plotted in \reffig{fofeta}, are as follows:
\begin{figure}[ht] %
\begin{center}
\includegraphics[width=0.6\textwidth]{generated/f_of_eta.pdf}
\end{center}
\caption{Bla.} % NB \protect\cite{...} is required in floating figures
\labfig{fofeta}
\end{figure}
\begin{align}
  f_P(\eta) &= \frac{\eta^2-\eta\sin(2\eta)+sin^2(\eta)}{\eta^4} \\
  f_G(\eta) &= \exp\left\{-\frac{9}{16\pi}\eta^2\right\} \\
  f_L(\eta) &= \frac{1}{1+\eta^2} \\
  f_F(\eta) &= \frac{ \sin^2(3\eta/4) } { (3\eta/4)^2 }
  \labeqn{fofeta}
\end{align}

MENTION: goal of a recipe is to provide yM(x,theta). Define precision definition.

Precision of Becker-Coppens recipes:

Mention that we are not just doing $y/(y_\text{ref}-1)$, since at low $x$, $y$ is
very near 1, and one would not consider $0.999\pm0.01$ to be a very precise
result, since it could predict $y>1$. So instead we will define the precision
as:
\begin{align}
  \text{precision}(y) \equiv \frac{|y-y_\text{ref}|}{\min(y_\text{ref},1-y_\text{ref})}
\end{align}
Naturally, for this definition to be useful in practice, this requires
$y_\text{ref}$ to be available in a precision much higher than the precision on
$y$. Bla mpmath.

\section{Original recipes}

Original recipes and their problems:


   - Start by mentioning that we can produce new high-prec integrals and
     recipes, to be discussed in later sections. Here they will be used as
     reference. (In section XXX we will discuss how the integral in \refeqn{yintegral}....)
   - table 1/3/4
   - the actual recipes
   - the breakdown plots

\begin{align}
  y_{p}(\theta,x)=\left\{1+2x+\frac{A(\theta)x^2}{1+B(\theta)x}\right\}
  \labeqn{ypfitfctorig}
\end{align}

\begin{align}
  A(\theta) &= 0.20 + 0.45\cos(2\theta) \\
  B(\theta) &= 0.22 - 0.12(0.5-\cos(2\theta))^2
  \labeqn{ypfitfctparsorig}
\end{align}

tables: Mention OCR(tesseract)+human checking.

\begin{table}[ht]
\caption{Updated BC Table 1 with a few extra columns and rows.}
\smallskip
\tiny
\begin{center}
\begin{tabular}{lrrrrrrrrrrrrr}
$\sin(\theta)$ & 0 & 0.05 & 0.1 & 0.2 & 0.3 & 0.4 & 0.5 & 0.6 & 0.7 & 0.8 & 0.9 & 0.99 & 1 \\
$x$ &  &  &  &  &  &  &  &  &  &  &  &  &  \\
0.001 & 9991 & 9991 & 9991 & 9991 & 9991 & 9991 & 9991 & 9991 & 9991 & 9991 & 9991 & 9991 & 9991 \\
0.01 & 9907 & 9907 & 9907 & 9907 & 9907 & 9907 & 9907 & 9907 & 9907 & 9907 & 9907 & 9907 & 9907 \\
0.1 & 9134 & 9134 & 9134 & 9134 & 9135 & 9137 & 9140 & 9143 & 9147 & 9152 & 9158 & 9164 & 9165 \\
0.2 & 8400 & 8400 & 8401 & 8403 & 8406 & 8412 & 8420 & 8430 & 8443 & 8459 & 8477 & 8495 & 8497 \\
0.4 & 7238 & 7238 & 7240 & 7245 & 7256 & 7273 & 7295 & 7322 & 7355 & 7394 & 7438 & 7482 & 7487 \\
0.6 & 6370 & 6371 & 6373 & 6383 & 6402 & 6429 & 6464 & 6507 & 6558 & 6615 & 6680 & 6743 & 6751 \\
0.8 & 5704 & 5705 & 5709 & 5724 & 5750 & 5786 & 5832 & 5887 & 5950 & 6021 & 6099 & 6176 & 6184 \\
1 & 5182 & 5183 & 5188 & 5207 & 5239 & 5283 & 5337 & 5400 & 5472 & 5551 & 5639 & 5723 & 5733 \\
1.5 & 4271 & 4273 & 4280 & 4308 & 4351 & 4406 & 4471 & 4546 & 4628 & 4717 & 4812 & 4903 & 4913 \\
2 & 3687 & 3690 & 3700 & 3733 & 3782 & 3842 & 3911 & 3988 & 4072 & 4161 & 4256 & 4345 & 4355 \\
2.5 & 3281 & 3285 & 3296 & 3333 & 3384 & 3446 & 3515 & 3591 & 3673 & 3760 & 3850 & 3936 & 3945 \\
3 & 2982 & 2986 & 2999 & 3037 & 3089 & 3150 & 3218 & 3292 & 3370 & 3453 & 3539 & 3620 & 3629 \\
4 & 2565 & 2571 & 2584 & 2624 & 2675 & 2733 & 2796 & 2864 & 2935 & 3010 & 3088 & 3160 & 3168 \\
5 & 2285 & 2291 & 2305 & 2344 & 2393 & 2447 & 2505 & 2568 & 2633 & 2701 & 2771 & 2837 & 2844 \\
6 & 2081 & 2088 & 2102 & 2139 & 2185 & 2236 & 2290 & 2347 & 2408 & 2470 & 2535 & 2594 & 2601 \\
7 & 1924 & 1931 & 1945 & 1981 & 2024 & 2071 & 2122 & 2176 & 2231 & 2289 & 2349 & 2404 & 2410 \\
8 & 1798 & 1805 & 1819 & 1854 & 1894 & 1939 & 1986 & 2036 & 2089 & 2143 & 2198 & 2250 & 2256 \\
9 & 1694 & 1702 & 1715 & 1748 & 1787 & 1829 & 1874 & 1921 & 1970 & 2021 & 2073 & 2122 & 2127 \\
10 & 1607 & 1614 & 1627 & 1659 & 1696 & 1736 & 1778 & 1823 & 1869 & 1918 & 1967 & 2013 & 2018 \\
12 & 1467 & 1474 & 1486 & 1515 & 1549 & 1585 & 1624 & 1665 & 1707 & 1751 & 1796 & 1838 & 1842 \\
14 & 1358 & 1365 & 1376 & 1404 & 1435 & 1468 & 1504 & 1541 & 1580 & 1621 & 1663 & 1701 & 1705 \\
16 & 1270 & 1277 & 1288 & 1313 & 1342 & 1374 & 1407 & 1442 & 1478 & 1516 & 1555 & 1591 & 1595 \\
18 & 1198 & 1204 & 1214 & 1238 & 1266 & 1295 & 1326 & 1359 & 1394 & 1429 & 1466 & 1500 & 1503 \\
20 & 1136 & 1143 & 1152 & 1175 & 1201 & 1229 & 1258 & 1290 & 1322 & 1356 & 1390 & 1422 & 1426 \\
25 & 1016 & 1022 & 1031 & 1051 & 1074 & 1099 & 1125 & 1153 & 1182 & 1212 & 1243 & 1272 & 1275 \\
30 & 0927 & 0933 & 0941 & 0959 & 0980 & 1003 & 1027 & 1052 & 1079 & 1106 & 1135 & 1161 & 1164 \\
100 & 0508 & 0511 & 0515 & 0525 & 0536 & 0549 & 0562 & 0576 & 0590 & 0605 & 0621 & 0635 & 0637 \\
1000 & 0161 & 0162 & 0163 & 0166 & 0170 & 0174 & 0178 & 0182 & 0187 & 0191 & 0196 & 0201 & 0201 \\
\end{tabular}

\end{center}
\end{table}

\begin{table}[ht]
\caption{Precision of original BC Table 1 compared to updated values (NB: MENTION PRECISION DEF USED).}
\smallskip
\tiny
\begin{center}
\begin{tabular}{lrrrrrrrrrr}
$\sin(\theta)$ & 0.05 & 0.1 & 0.2 & 0.3 & 0.4 & 0.5 & 0.6 & 0.7 & 0.8 & 0.9 \\
$x$ &  &  &  &  &  &  &  &  &  &  \\
0.1 & {\cellcolor[HTML]{F8F8FF}} \color[HTML]{000000} -0.50 & {\cellcolor[HTML]{F8F8FF}} \color[HTML]{000000} -0.47 & {\cellcolor[HTML]{F8F8FF}} \color[HTML]{000000} -0.42 & {\cellcolor[HTML]{F8F8FF}} \color[HTML]{000000} -0.43 & {\cellcolor[HTML]{F8F8FF}} \color[HTML]{000000} -0.45 & {\cellcolor[HTML]{F8F8FF}} \color[HTML]{000000} -0.48 & {\cellcolor[HTML]{F8F8FF}} \color[HTML]{000000} -0.52 & {\cellcolor[HTML]{F6F6FF}} \color[HTML]{000000} -0.57 & {\cellcolor[HTML]{F6F6FF}} \color[HTML]{000000} -0.57 & {\cellcolor[HTML]{F6F6FF}} \color[HTML]{000000} -0.56 \\
0.2 & {\cellcolor[HTML]{FAFAFF}} \color[HTML]{000000} -0.40 & {\cellcolor[HTML]{FAFAFF}} \color[HTML]{000000} -0.33 & {\cellcolor[HTML]{FCFCFF}} \color[HTML]{000000} -0.24 & {\cellcolor[HTML]{FAFAFF}} \color[HTML]{000000} -0.29 & {\cellcolor[HTML]{FAFAFF}} \color[HTML]{000000} -0.33 & {\cellcolor[HTML]{FAFAFF}} \color[HTML]{000000} -0.38 & {\cellcolor[HTML]{F8F8FF}} \color[HTML]{000000} -0.47 & {\cellcolor[HTML]{F8F8FF}} \color[HTML]{000000} -0.53 & {\cellcolor[HTML]{F8F8FF}} \color[HTML]{000000} -0.49 & {\cellcolor[HTML]{F8F8FF}} \color[HTML]{000000} -0.43 \\
0.4 & {\cellcolor[HTML]{FCFCFF}} \color[HTML]{000000} -0.23 & {\cellcolor[HTML]{FEFEFF}} \color[HTML]{000000} -0.12 & {\cellcolor[HTML]{FFFEFE}} \color[HTML]{000000} 0.05 & {\cellcolor[HTML]{FEFEFF}} \color[HTML]{000000} -0.06 & {\cellcolor[HTML]{FEFEFF}} \color[HTML]{000000} -0.12 & {\cellcolor[HTML]{FCFCFF}} \color[HTML]{000000} -0.20 & {\cellcolor[HTML]{FAFAFF}} \color[HTML]{000000} -0.32 & {\cellcolor[HTML]{FAFAFF}} \color[HTML]{000000} -0.36 & {\cellcolor[HTML]{FCFCFF}} \color[HTML]{000000} -0.22 & {\cellcolor[HTML]{FFFEFE}} \color[HTML]{000000} 0.00 \\
0.6 & {\cellcolor[HTML]{FEFEFF}} \color[HTML]{000000} -0.09 & {\cellcolor[HTML]{FFFEFE}} \color[HTML]{000000} 0.03 & {\cellcolor[HTML]{FFFCFC}} \color[HTML]{000000} 0.25 & {\cellcolor[HTML]{FFFEFE}} \color[HTML]{000000} 0.12 & {\cellcolor[HTML]{FFFEFE}} \color[HTML]{000000} 0.04 & {\cellcolor[HTML]{FEFEFF}} \color[HTML]{000000} -0.05 & {\cellcolor[HTML]{FCFCFF}} \color[HTML]{000000} -0.16 & {\cellcolor[HTML]{FEFEFF}} \color[HTML]{000000} -0.13 & {\cellcolor[HTML]{FFFCFC}} \color[HTML]{000000} 0.20 & {\cellcolor[HTML]{FFF6F6}} \color[HTML]{000000} 0.65 \\
0.8 & {\cellcolor[HTML]{FFFEFE}} \color[HTML]{000000} 0.01 & {\cellcolor[HTML]{FFFEFE}} \color[HTML]{000000} 0.13 & {\cellcolor[HTML]{FFFAFA}} \color[HTML]{000000} 0.39 & {\cellcolor[HTML]{FFFCFC}} \color[HTML]{000000} 0.23 & {\cellcolor[HTML]{FFFCFC}} \color[HTML]{000000} 0.14 & {\cellcolor[HTML]{FFFEFE}} \color[HTML]{000000} 0.05 & {\cellcolor[HTML]{FFFEFE}} \color[HTML]{000000} 0.01 & {\cellcolor[HTML]{FFFEFE}} \color[HTML]{000000} 0.12 & {\cellcolor[HTML]{FFF6F6}} \color[HTML]{000000} 0.67 & {\cellcolor[HTML]{FFEAEA}} \color[HTML]{000000} 1.41 \\
1 & {\cellcolor[HTML]{FFFEFE}} \color[HTML]{000000} 0.04 & {\cellcolor[HTML]{FFFCFC}} \color[HTML]{000000} 0.20 & {\cellcolor[HTML]{FFF8F8}} \color[HTML]{000000} 0.48 & {\cellcolor[HTML]{FFFCFC}} \color[HTML]{000000} 0.26 & {\cellcolor[HTML]{FFFCFC}} \color[HTML]{000000} 0.18 & {\cellcolor[HTML]{FFFEFE}} \color[HTML]{000000} 0.12 & {\cellcolor[HTML]{FFFEFE}} \color[HTML]{000000} 0.13 & {\cellcolor[HTML]{FFFAFA}} \color[HTML]{000000} 0.37 & {\cellcolor[HTML]{FFEEEE}} \color[HTML]{000000} 1.14 & {\cellcolor[HTML]{FFDEDE}} \color[HTML]{000000} 2.21 \\
1.5 & {\cellcolor[HTML]{FFFEFE}} \color[HTML]{000000} 0.00 & {\cellcolor[HTML]{FFFCFC}} \color[HTML]{000000} 0.14 & {\cellcolor[HTML]{FFF8F8}} \color[HTML]{000000} 0.46 & {\cellcolor[HTML]{FFFCFC}} \color[HTML]{000000} 0.21 & {\cellcolor[HTML]{FFFEFE}} \color[HTML]{000000} 0.09 & {\cellcolor[HTML]{FFFEFE}} \color[HTML]{000000} 0.11 & {\cellcolor[HTML]{FFFAFA}} \color[HTML]{000000} 0.34 & {\cellcolor[HTML]{FFF2F2}} \color[HTML]{000000} 0.92 & {\cellcolor[HTML]{FFDEDE}} \color[HTML]{000000} 2.28 & {\cellcolor[HTML]{FFC2C2}} \color[HTML]{000000} 4.15 \\
2 & {\cellcolor[HTML]{FCFCFF}} \color[HTML]{000000} -0.20 & {\cellcolor[HTML]{FEFEFF}} \color[HTML]{000000} -0.10 & {\cellcolor[HTML]{FFFCFC}} \color[HTML]{000000} 0.26 & {\cellcolor[HTML]{FEFEFF}} \color[HTML]{000000} -0.05 & {\cellcolor[HTML]{FCFCFF}} \color[HTML]{000000} -0.15 & {\cellcolor[HTML]{FEFEFF}} \color[HTML]{000000} -0.02 & {\cellcolor[HTML]{FFFAFA}} \color[HTML]{000000} 0.40 & {\cellcolor[HTML]{FFECEC}} \color[HTML]{000000} 1.31 & {\cellcolor[HTML]{FFD0D0}} \color[HTML]{000000} 3.22 & {\cellcolor[HTML]{FFAAAA}} \color[HTML]{000000} 5.88 \\
2.5 & {\cellcolor[HTML]{F8F8FF}} \color[HTML]{000000} -0.50 & {\cellcolor[HTML]{F8F8FF}} \color[HTML]{000000} -0.44 & {\cellcolor[HTML]{FEFEFF}} \color[HTML]{000000} -0.04 & {\cellcolor[HTML]{FAFAFF}} \color[HTML]{000000} -0.37 & {\cellcolor[HTML]{F8F8FF}} \color[HTML]{000000} -0.46 & {\cellcolor[HTML]{FCFCFF}} \color[HTML]{000000} -0.21 & {\cellcolor[HTML]{FFFAFA}} \color[HTML]{000000} 0.38 & {\cellcolor[HTML]{FFE8E8}} \color[HTML]{000000} 1.58 & {\cellcolor[HTML]{FFC4C4}} \color[HTML]{000000} 4.00 & {\cellcolor[HTML]{FF9494}} \color[HTML]{000000} 7.37 \\
3 & {\cellcolor[HTML]{EEEEFF}} \color[HTML]{000000} -1.15 & {\cellcolor[HTML]{F4F4FF}} \color[HTML]{000000} -0.79 & {\cellcolor[HTML]{FAFAFF}} \color[HTML]{000000} -0.41 & {\cellcolor[HTML]{F4F4FF}} \color[HTML]{000000} -0.72 & {\cellcolor[HTML]{F4F4FF}} \color[HTML]{000000} -0.77 & {\cellcolor[HTML]{F8F8FF}} \color[HTML]{000000} -0.44 & {\cellcolor[HTML]{FFFAFA}} \color[HTML]{000000} 0.34 & {\cellcolor[HTML]{FFE4E4}} \color[HTML]{000000} 1.80 & {\cellcolor[HTML]{FFBABA}} \color[HTML]{000000} 4.66 & {\cellcolor[HTML]{FF8080}} \color[HTML]{F1F1F1} 8.66 \\
4 & {\cellcolor[HTML]{EAEAFF}} \color[HTML]{000000} -1.50 & {\cellcolor[HTML]{E8E8FF}} \color[HTML]{000000} -1.55 & {\cellcolor[HTML]{FFD8D8}} \color[HTML]{000000} 2.71 & {\cellcolor[HTML]{EAEAFF}} \color[HTML]{000000} -1.37 & {\cellcolor[HTML]{EAEAFF}} \color[HTML]{000000} -1.37 & {\cellcolor[HTML]{F2F2FF}} \color[HTML]{000000} -0.86 & {\cellcolor[HTML]{FFFCFC}} \color[HTML]{000000} 0.18 & {\cellcolor[HTML]{FFE0E0}} \color[HTML]{000000} 2.10 & {\cellcolor[HTML]{FFACAC}} \color[HTML]{000000} 5.64 & {\cellcolor[HTML]{FF6262}} \color[HTML]{F1F1F1} 10.69 \\
5 & {\cellcolor[HTML]{E0E0FF}} \color[HTML]{000000} -2.15 & {\cellcolor[HTML]{DEDEFF}} \color[HTML]{000000} -2.26 & {\cellcolor[HTML]{E6E6FF}} \color[HTML]{000000} -1.77 & {\cellcolor[HTML]{E2E2FF}} \color[HTML]{000000} -1.99 & {\cellcolor[HTML]{E2E2FF}} \color[HTML]{000000} -1.95 & {\cellcolor[HTML]{ECECFF}} \color[HTML]{000000} -1.25 & {\cellcolor[HTML]{FEFEFF}} \color[HTML]{000000} -0.03 & {\cellcolor[HTML]{FFDEDE}} \color[HTML]{000000} 2.28 & {\cellcolor[HTML]{FFA2A2}} \color[HTML]{000000} 6.36 & {\cellcolor[HTML]{FF4C4C}} \color[HTML]{F1F1F1} 12.25 \\
6 & {\cellcolor[HTML]{D6D6FF}} \color[HTML]{000000} -2.76 & {\cellcolor[HTML]{D4D4FF}} \color[HTML]{000000} -2.93 & {\cellcolor[HTML]{DCDCFF}} \color[HTML]{000000} -2.36 & {\cellcolor[HTML]{DADAFF}} \color[HTML]{000000} -2.52 & {\cellcolor[HTML]{DCDCFF}} \color[HTML]{000000} -2.40 & {\cellcolor[HTML]{E8E8FF}} \color[HTML]{000000} -1.61 & {\cellcolor[HTML]{FCFCFF}} \color[HTML]{000000} -0.23 & {\cellcolor[HTML]{FFDCDC}} \color[HTML]{000000} 2.34 & {\cellcolor[HTML]{FF9A9A}} \color[HTML]{000000} 6.88 & {\cellcolor[HTML]{FF3A3A}} \color[HTML]{F1F1F1} 13.43 \\
7 & {\cellcolor[HTML]{CECEFF}} \color[HTML]{000000} -3.30 & {\cellcolor[HTML]{CCCCFF}} \color[HTML]{000000} -3.53 & {\cellcolor[HTML]{D4D4FF}} \color[HTML]{000000} -2.88 & {\cellcolor[HTML]{D4D4FF}} \color[HTML]{000000} -2.96 & {\cellcolor[HTML]{D6D6FF}} \color[HTML]{000000} -2.81 & {\cellcolor[HTML]{E2E2FF}} \color[HTML]{000000} -1.93 & {\cellcolor[HTML]{F8F8FF}} \color[HTML]{000000} -0.44 & {\cellcolor[HTML]{FFDCDC}} \color[HTML]{000000} 2.36 & {\cellcolor[HTML]{FF9696}} \color[HTML]{000000} 7.20 & {\cellcolor[HTML]{FF2E2E}} \color[HTML]{F1F1F1} 14.31 \\
8 & {\cellcolor[HTML]{C6C6FF}} \color[HTML]{000000} -3.84 & {\cellcolor[HTML]{C2C2FF}} \color[HTML]{000000} -4.11 & {\cellcolor[HTML]{CECEFF}} \color[HTML]{000000} -3.37 & {\cellcolor[HTML]{CECEFF}} \color[HTML]{000000} -3.39 & {\cellcolor[HTML]{D0D0FF}} \color[HTML]{000000} -3.18 & {\cellcolor[HTML]{DEDEFF}} \color[HTML]{000000} -2.28 & {\cellcolor[HTML]{F6F6FF}} \color[HTML]{000000} -0.66 & {\cellcolor[HTML]{FFDEDE}} \color[HTML]{000000} 2.32 & {\cellcolor[HTML]{FF9292}} \color[HTML]{000000} 7.49 & {\cellcolor[HTML]{FF2424}} \color[HTML]{F1F1F1} 15.04 \\
9 & {\cellcolor[HTML]{BEBEFF}} \color[HTML]{000000} -4.39 & {\cellcolor[HTML]{BCBCFF}} \color[HTML]{000000} -4.60 & {\cellcolor[HTML]{C6C6FF}} \color[HTML]{000000} -3.84 & {\cellcolor[HTML]{C8C8FF}} \color[HTML]{000000} -3.79 & {\cellcolor[HTML]{CCCCFF}} \color[HTML]{000000} -3.54 & {\cellcolor[HTML]{DADAFF}} \color[HTML]{000000} -2.60 & {\cellcolor[HTML]{F2F2FF}} \color[HTML]{000000} -0.88 & {\cellcolor[HTML]{FFDEDE}} \color[HTML]{000000} 2.29 & {\cellcolor[HTML]{FF8E8E}} \color[HTML]{000000} 7.68 & {\cellcolor[HTML]{FF1A1A}} \color[HTML]{F1F1F1} 15.62 \\
10 & {\cellcolor[HTML]{B8B8FF}} \color[HTML]{000000} -4.86 & {\cellcolor[HTML]{B4B4FF}} \color[HTML]{000000} -5.11 & {\cellcolor[HTML]{C0C0FF}} \color[HTML]{000000} -4.28 & {\cellcolor[HTML]{C2C2FF}} \color[HTML]{000000} -4.17 & {\cellcolor[HTML]{C6C6FF}} \color[HTML]{000000} -3.90 & {\cellcolor[HTML]{D4D4FF}} \color[HTML]{000000} -2.88 & {\cellcolor[HTML]{F0F0FF}} \color[HTML]{000000} -1.09 & {\cellcolor[HTML]{FFE0E0}} \color[HTML]{000000} 2.17 & {\cellcolor[HTML]{FF8E8E}} \color[HTML]{000000} 7.80 & {\cellcolor[HTML]{FF1414}} \color[HTML]{F1F1F1} 16.06 \\
12 & {\cellcolor[HTML]{AAAAFF}} \color[HTML]{000000} -5.84 & {\cellcolor[HTML]{A6A6FF}} \color[HTML]{000000} -6.06 & {\cellcolor[HTML]{B4B4FF}} \color[HTML]{000000} -5.17 & {\cellcolor[HTML]{B8B8FF}} \color[HTML]{000000} -4.90 & {\cellcolor[HTML]{BCBCFF}} \color[HTML]{000000} -4.56 & {\cellcolor[HTML]{CCCCFF}} \color[HTML]{000000} -3.45 & {\cellcolor[HTML]{E6E6FF}} \color[HTML]{000000} -1.66 & {\cellcolor[HTML]{FFE2E2}} \color[HTML]{000000} 1.93 & {\cellcolor[HTML]{FF8C8C}} \color[HTML]{000000} 7.90 & {\cellcolor[HTML]{FF0C0C}} \color[HTML]{F1F1F1} 16.66 \\
14 & {\cellcolor[HTML]{9C9CFF}} \color[HTML]{F1F1F1} -6.74 & {\cellcolor[HTML]{9898FF}} \color[HTML]{F1F1F1} -7.00 & {\cellcolor[HTML]{A6A6FF}} \color[HTML]{000000} -6.02 & {\cellcolor[HTML]{ACACFF}} \color[HTML]{000000} -5.62 & {\cellcolor[HTML]{B2B2FF}} \color[HTML]{000000} -5.26 & {\cellcolor[HTML]{C4C4FF}} \color[HTML]{000000} -4.05 & {\cellcolor[HTML]{E2E2FF}} \color[HTML]{000000} -2.04 & {\cellcolor[HTML]{FFE8E8}} \color[HTML]{000000} 1.62 & {\cellcolor[HTML]{FF8C8C}} \color[HTML]{000000} 7.84 & {\cellcolor[HTML]{FF0606}} \color[HTML]{F1F1F1} 17.05 \\
16 & {\cellcolor[HTML]{9090FF}} \color[HTML]{F1F1F1} -7.61 & {\cellcolor[HTML]{8C8CFF}} \color[HTML]{F1F1F1} -7.90 & {\cellcolor[HTML]{9C9CFF}} \color[HTML]{F1F1F1} -6.80 & {\cellcolor[HTML]{A2A2FF}} \color[HTML]{F1F1F1} -6.35 & {\cellcolor[HTML]{AAAAFF}} \color[HTML]{000000} -5.87 & {\cellcolor[HTML]{BCBCFF}} \color[HTML]{000000} -4.61 & {\cellcolor[HTML]{DADAFF}} \color[HTML]{000000} -2.56 & {\cellcolor[HTML]{FFECEC}} \color[HTML]{000000} 1.26 & {\cellcolor[HTML]{FF8E8E}} \color[HTML]{000000} 7.78 & {\cellcolor[HTML]{FF0202}} \color[HTML]{F1F1F1} 17.30 \\
18 & {\cellcolor[HTML]{8484FF}} \color[HTML]{F1F1F1} -8.41 & {\cellcolor[HTML]{8080FF}} \color[HTML]{F1F1F1} -8.76 & {\cellcolor[HTML]{9090FF}} \color[HTML]{F1F1F1} -7.54 & {\cellcolor[HTML]{9898FF}} \color[HTML]{F1F1F1} -7.00 & {\cellcolor[HTML]{A0A0FF}} \color[HTML]{F1F1F1} -6.49 & {\cellcolor[HTML]{B2B2FF}} \color[HTML]{000000} -5.24 & {\cellcolor[HTML]{D2D2FF}} \color[HTML]{000000} -3.04 & {\cellcolor[HTML]{FFF2F2}} \color[HTML]{000000} 0.88 & {\cellcolor[HTML]{FF9090}} \color[HTML]{000000} 7.61 & {\cellcolor[HTML]{FF0000}} \color[HTML]{F1F1F1} 17.48 \\
20 & {\cellcolor[HTML]{7878FF}} \color[HTML]{F1F1F1} -9.24 & {\cellcolor[HTML]{7474FF}} \color[HTML]{F1F1F1} -9.56 & {\cellcolor[HTML]{8282FF}} \color[HTML]{F1F1F1} -8.51 & {\cellcolor[HTML]{9090FF}} \color[HTML]{F1F1F1} -7.64 & {\cellcolor[HTML]{9696FF}} \color[HTML]{F1F1F1} -7.13 & {\cellcolor[HTML]{AAAAFF}} \color[HTML]{000000} -5.83 & {\cellcolor[HTML]{CACAFF}} \color[HTML]{000000} -3.61 & {\cellcolor[HTML]{FFF8F8}} \color[HTML]{000000} 0.53 & {\cellcolor[HTML]{FF9292}} \color[HTML]{000000} 7.40 & {\cellcolor[HTML]{FF0000}} \color[HTML]{F1F1F1} 17.52 \\
25 & {\cellcolor[HTML]{5C5CFF}} \color[HTML]{F1F1F1} -11.16 & {\cellcolor[HTML]{5858FF}} \color[HTML]{F1F1F1} -11.42 & {\cellcolor[HTML]{6C6CFF}} \color[HTML]{F1F1F1} -10.08 & {\cellcolor[HTML]{7878FF}} \color[HTML]{F1F1F1} -9.20 & {\cellcolor[HTML]{8282FF}} \color[HTML]{F1F1F1} -8.53 & {\cellcolor[HTML]{9696FF}} \color[HTML]{F1F1F1} -7.13 & {\cellcolor[HTML]{B8B8FF}} \color[HTML]{000000} -4.86 & {\cellcolor[HTML]{F8F8FF}} \color[HTML]{000000} -0.52 & {\cellcolor[HTML]{FF9C9C}} \color[HTML]{000000} 6.75 & {\cellcolor[HTML]{FF0000}} \color[HTML]{F1F1F1} 17.44 \\
30 & {\cellcolor[HTML]{4444FF}} \color[HTML]{F1F1F1} -12.76 & {\cellcolor[HTML]{4444FF}} \color[HTML]{F1F1F1} -12.84 & {\cellcolor[HTML]{5656FF}} \color[HTML]{F1F1F1} -11.60 & {\cellcolor[HTML]{6464FF}} \color[HTML]{F1F1F1} -10.62 & {\cellcolor[HTML]{6E6EFF}} \color[HTML]{F1F1F1} -9.86 & {\cellcolor[HTML]{8484FF}} \color[HTML]{F1F1F1} -8.47 & {\cellcolor[HTML]{A6A6FF}} \color[HTML]{000000} -6.02 & {\cellcolor[HTML]{EAEAFF}} \color[HTML]{000000} -1.47 & {\cellcolor[HTML]{FFA6A6}} \color[HTML]{000000} 6.03 & {\cellcolor[HTML]{FF0404}} \color[HTML]{F1F1F1} 17.22 \\
\end{tabular}

\end{center}
\end{table}

\begin{table}[ht]
\caption{Updated BC Table 3 with a few extra columns and rows.}
\smallskip
\tiny
\begin{center}
\begin{tabular}{lrrrrrrrrrrrrr}
$\sin(\theta)$ & 0 & 0.05 & 0.1 & 0.2 & 0.3 & 0.4 & 0.5 & 0.6 & 0.7 & 0.8 & 0.9 & 0.99 & 1 \\
$x$ &  &  &  &  &  &  &  &  &  &  &  &  &  \\
0.001 & 9989 & 9989 & 9989 & 9989 & 9989 & 9989 & 9989 & 9989 & 9989 & 9989 & 9989 & 9989 & 9989 \\
0.01 & 9895 & 9895 & 9895 & 9895 & 9895 & 9895 & 9895 & 9895 & 9895 & 9895 & 9895 & 9895 & 9895 \\
0.1 & 9025 & 9025 & 9026 & 9026 & 9027 & 9029 & 9032 & 9036 & 9040 & 9046 & 9053 & 9060 & 9061 \\
0.2 & 8201 & 8201 & 8201 & 8203 & 8208 & 8214 & 8223 & 8235 & 8249 & 8267 & 8287 & 8308 & 8310 \\
0.4 & 6895 & 6895 & 6896 & 6903 & 6915 & 6934 & 6959 & 6990 & 7027 & 7070 & 7120 & 7170 & 7176 \\
0.6 & 5920 & 5921 & 5923 & 5935 & 5956 & 5987 & 6027 & 6075 & 6132 & 6197 & 6270 & 6342 & 6350 \\
0.8 & 5174 & 5175 & 5179 & 5196 & 5226 & 5266 & 5318 & 5380 & 5451 & 5531 & 5620 & 5707 & 5717 \\
1 & 4590 & 4591 & 4596 & 4618 & 4655 & 4704 & 4765 & 4836 & 4917 & 5008 & 5106 & 5201 & 5212 \\
1.5 & 3576 & 3579 & 3587 & 3618 & 3667 & 3729 & 3803 & 3887 & 3980 & 4080 & 4188 & 4291 & 4302 \\
2 & 2933 & 2936 & 2947 & 2985 & 3040 & 3108 & 3186 & 3273 & 3367 & 3468 & 3575 & 3676 & 3687 \\
2.5 & 2491 & 2495 & 2508 & 2550 & 2607 & 2677 & 2755 & 2841 & 2933 & 3031 & 3133 & 3229 & 3240 \\
3 & 2169 & 2174 & 2188 & 2232 & 2290 & 2359 & 2436 & 2519 & 2607 & 2700 & 2797 & 2888 & 2898 \\
4 & 1730 & 1736 & 1752 & 1797 & 1854 & 1919 & 1990 & 2067 & 2147 & 2230 & 2317 & 2397 & 2406 \\
5 & 1444 & 1451 & 1467 & 1512 & 1566 & 1627 & 1692 & 1762 & 1834 & 1910 & 1987 & 2059 & 2067 \\
6 & 1243 & 1250 & 1266 & 1309 & 1360 & 1417 & 1477 & 1541 & 1607 & 1675 & 1746 & 1810 & 1818 \\
7 & 1093 & 1101 & 1116 & 1157 & 1205 & 1258 & 1314 & 1372 & 1433 & 1496 & 1560 & 1619 & 1625 \\
8 & 0976 & 0985 & 1000 & 1039 & 1084 & 1133 & 1185 & 1239 & 1296 & 1353 & 1412 & 1467 & 1473 \\
9 & 0883 & 0892 & 0907 & 0944 & 0986 & 1032 & 1081 & 1131 & 1184 & 1237 & 1292 & 1342 & 1348 \\
10 & 0807 & 0816 & 0830 & 0865 & 0906 & 0949 & 0995 & 1042 & 1091 & 1141 & 1192 & 1239 & 1244 \\
12 & 0690 & 0698 & 0712 & 0744 & 0780 & 0819 & 0860 & 0902 & 0945 & 0989 & 1034 & 1075 & 1080 \\
14 & 0604 & 0612 & 0624 & 0654 & 0687 & 0722 & 0759 & 0797 & 0836 & 0875 & 0916 & 0952 & 0957 \\
16 & 0537 & 0545 & 0557 & 0584 & 0615 & 0647 & 0680 & 0715 & 0750 & 0786 & 0823 & 0856 & 0860 \\
18 & 0484 & 0492 & 0503 & 0529 & 0557 & 0586 & 0617 & 0649 & 0681 & 0714 & 0748 & 0778 & 0782 \\
20 & 0441 & 0449 & 0459 & 0483 & 0509 & 0537 & 0565 & 0595 & 0625 & 0655 & 0686 & 0714 & 0718 \\
25 & 0362 & 0369 & 0378 & 0399 & 0421 & 0445 & 0469 & 0494 & 0519 & 0545 & 0571 & 0594 & 0597 \\
30 & 0308 & 0314 & 0322 & 0341 & 0360 & 0381 & 0402 & 0423 & 0445 & 0467 & 0490 & 0510 & 0513 \\
100 & 0104 & 0106 & 0110 & 0117 & 0125 & 0132 & 0140 & 0148 & 0156 & 0164 & 0173 & 0180 & 0181 \\
1000 & 0012 & 0013 & 0013 & 0014 & 0015 & 0016 & 0017 & 0018 & 0019 & 0020 & 0021 & 0022 & 0022 \\
\end{tabular}

\end{center}
\end{table}

\begin{table}[ht]
\caption{Precision of original BC Table 3 compared to updated values (NB: MENTION PRECISION DEF USED).}
\smallskip
\tiny
\begin{center}
\begin{tabular}{lrrrrrrrrrr}
$\sin(\theta)$ & 0.05 & 0.1 & 0.2 & 0.3 & 0.4 & 0.5 & 0.6 & 0.7 & 0.8 & 0.9 \\
$x$ &  &  &  &  &  &  &  &  &  &  \\
0.1 & {\cellcolor[HTML]{FFF8F8}} \color[HTML]{000000} 1.70 & {\cellcolor[HTML]{FFF6F6}} \color[HTML]{000000} 1.99 & {\cellcolor[HTML]{FFF4F4}} \color[HTML]{000000} 2.34 & {\cellcolor[HTML]{FFF6F6}} \color[HTML]{000000} 2.22 & {\cellcolor[HTML]{FFF6F6}} \color[HTML]{000000} 2.02 & {\cellcolor[HTML]{FFF8F8}} \color[HTML]{000000} 1.74 & {\cellcolor[HTML]{FFF8F8}} \color[HTML]{000000} 1.37 & {\cellcolor[HTML]{FFFCFC}} \color[HTML]{000000} 0.89 & {\cellcolor[HTML]{FFFCFC}} \color[HTML]{000000} 0.72 & {\cellcolor[HTML]{FFFAFA}} \color[HTML]{000000} 1.08 \\
0.2 & {\cellcolor[HTML]{FFF8F8}} \color[HTML]{000000} 1.62 & {\cellcolor[HTML]{FFF6F6}} \color[HTML]{000000} 1.93 & {\cellcolor[HTML]{FFF4F4}} \color[HTML]{000000} 2.43 & {\cellcolor[HTML]{FFF6F6}} \color[HTML]{000000} 2.20 & {\cellcolor[HTML]{FFF6F6}} \color[HTML]{000000} 1.96 & {\cellcolor[HTML]{FFF8F8}} \color[HTML]{000000} 1.68 & {\cellcolor[HTML]{FFFAFA}} \color[HTML]{000000} 1.26 & {\cellcolor[HTML]{FFFAFA}} \color[HTML]{000000} 1.01 & {\cellcolor[HTML]{FFFAFA}} \color[HTML]{000000} 1.17 & {\cellcolor[HTML]{FFF8F8}} \color[HTML]{000000} 1.46 \\
0.4 & {\cellcolor[HTML]{FFF8F8}} \color[HTML]{000000} 1.55 & {\cellcolor[HTML]{FFF6F6}} \color[HTML]{000000} 1.83 & {\cellcolor[HTML]{FFF6F6}} \color[HTML]{000000} 2.23 & {\cellcolor[HTML]{FFF6F6}} \color[HTML]{000000} 2.00 & {\cellcolor[HTML]{FFF6F6}} \color[HTML]{000000} 1.83 & {\cellcolor[HTML]{FFF8F8}} \color[HTML]{000000} 1.63 & {\cellcolor[HTML]{FFFAFA}} \color[HTML]{000000} 1.34 & {\cellcolor[HTML]{FFFAFA}} \color[HTML]{000000} 1.25 & {\cellcolor[HTML]{FFF8F8}} \color[HTML]{000000} 1.66 & {\cellcolor[HTML]{FFF4F4}} \color[HTML]{000000} 2.32 \\
0.6 & {\cellcolor[HTML]{FFF8F8}} \color[HTML]{000000} 1.48 & {\cellcolor[HTML]{FFF8F8}} \color[HTML]{000000} 1.68 & {\cellcolor[HTML]{FFF6F6}} \color[HTML]{000000} 2.06 & {\cellcolor[HTML]{FFF6F6}} \color[HTML]{000000} 1.84 & {\cellcolor[HTML]{FFF8F8}} \color[HTML]{000000} 1.72 & {\cellcolor[HTML]{FF9494}} \color[HTML]{000000} 24.22 & {\cellcolor[HTML]{FFF8F8}} \color[HTML]{000000} 1.40 & {\cellcolor[HTML]{FFF8F8}} \color[HTML]{000000} 1.48 & {\cellcolor[HTML]{FFF6F6}} \color[HTML]{000000} 2.14 & {\cellcolor[HTML]{FFF2F2}} \color[HTML]{000000} 3.09 \\
0.8 & {\cellcolor[HTML]{FFF8F8}} \color[HTML]{000000} 1.38 & {\cellcolor[HTML]{FFF8F8}} \color[HTML]{000000} 1.55 & {\cellcolor[HTML]{FFF6F6}} \color[HTML]{000000} 1.89 & {\cellcolor[HTML]{FFF8F8}} \color[HTML]{000000} 1.69 & {\cellcolor[HTML]{FFF8F8}} \color[HTML]{000000} 1.58 & {\cellcolor[HTML]{FFF8F8}} \color[HTML]{000000} 1.47 & {\cellcolor[HTML]{FFF8F8}} \color[HTML]{000000} 1.43 & {\cellcolor[HTML]{FFF8F8}} \color[HTML]{000000} 1.64 & {\cellcolor[HTML]{FFF4F4}} \color[HTML]{000000} 2.50 & {\cellcolor[HTML]{FFEEEE}} \color[HTML]{000000} 3.72 \\
1 & {\cellcolor[HTML]{FFF8F8}} \color[HTML]{000000} 1.50 & {\cellcolor[HTML]{FFF8F8}} \color[HTML]{000000} 1.66 & {\cellcolor[HTML]{FFF6F6}} \color[HTML]{000000} 2.03 & {\cellcolor[HTML]{FFF8F8}} \color[HTML]{000000} 1.77 & {\cellcolor[HTML]{FFF8F8}} \color[HTML]{000000} 1.60 & {\cellcolor[HTML]{FFF8F8}} \color[HTML]{000000} 1.52 & {\cellcolor[HTML]{FFF8F8}} \color[HTML]{000000} 1.55 & {\cellcolor[HTML]{FFF6F6}} \color[HTML]{000000} 1.80 & {\cellcolor[HTML]{FFF2F2}} \color[HTML]{000000} 2.77 & {\cellcolor[HTML]{FFECEC}} \color[HTML]{000000} 4.21 \\
1.5 & {\cellcolor[HTML]{FFF6F6}} \color[HTML]{000000} 1.91 & {\cellcolor[HTML]{FFF6F6}} \color[HTML]{000000} 1.99 & {\cellcolor[HTML]{FFF4F4}} \color[HTML]{000000} 2.42 & {\cellcolor[HTML]{FFF6F6}} \color[HTML]{000000} 2.05 & {\cellcolor[HTML]{FFF6F6}} \color[HTML]{000000} 1.85 & {\cellcolor[HTML]{FFF6F6}} \color[HTML]{000000} 1.87 & {\cellcolor[HTML]{FFF6F6}} \color[HTML]{000000} 2.14 & {\cellcolor[HTML]{FFF2F2}} \color[HTML]{000000} 2.88 & {\cellcolor[HTML]{FFEAEA}} \color[HTML]{000000} 4.60 & {\cellcolor[HTML]{FFE0E0}} \color[HTML]{000000} 6.99 \\
2 & {\cellcolor[HTML]{FFF6F6}} \color[HTML]{000000} 1.99 & {\cellcolor[HTML]{FFF6F6}} \color[HTML]{000000} 2.10 & {\cellcolor[HTML]{FFF4F4}} \color[HTML]{000000} 2.58 & {\cellcolor[HTML]{FFF6F6}} \color[HTML]{000000} 2.11 & {\cellcolor[HTML]{FFF6F6}} \color[HTML]{000000} 1.91 & {\cellcolor[HTML]{FFF6F6}} \color[HTML]{000000} 2.07 & {\cellcolor[HTML]{FFF4F4}} \color[HTML]{000000} 2.60 & {\cellcolor[HTML]{FFEEEE}} \color[HTML]{000000} 3.79 & {\cellcolor[HTML]{FFE2E2}} \color[HTML]{000000} 6.36 & {\cellcolor[HTML]{FFD4D4}} \color[HTML]{000000} 9.83 \\
2.5 & {\cellcolor[HTML]{FFF6F6}} \color[HTML]{000000} 1.99 & {\cellcolor[HTML]{FFF6F6}} \color[HTML]{000000} 2.04 & {\cellcolor[HTML]{FFF4F4}} \color[HTML]{000000} 2.56 & {\cellcolor[HTML]{FFF6F6}} \color[HTML]{000000} 2.02 & {\cellcolor[HTML]{FFF6F6}} \color[HTML]{000000} 1.81 & {\cellcolor[HTML]{FFF6F6}} \color[HTML]{000000} 2.17 & {\cellcolor[HTML]{FFF2F2}} \color[HTML]{000000} 2.96 & {\cellcolor[HTML]{FFEAEA}} \color[HTML]{000000} 4.63 & {\cellcolor[HTML]{FFDCDC}} \color[HTML]{000000} 7.92 & {\cellcolor[HTML]{FFC8C8}} \color[HTML]{000000} 12.47 \\
3 & {\cellcolor[HTML]{FFF6F6}} \color[HTML]{000000} 1.89 & {\cellcolor[HTML]{FFF6F6}} \color[HTML]{000000} 1.89 & {\cellcolor[HTML]{FFF4F4}} \color[HTML]{000000} 2.44 & {\cellcolor[HTML]{FFF6F6}} \color[HTML]{000000} 1.87 & {\cellcolor[HTML]{FFF8F8}} \color[HTML]{000000} 1.74 & {\cellcolor[HTML]{FFF6F6}} \color[HTML]{000000} 2.19 & {\cellcolor[HTML]{FFF0F0}} \color[HTML]{000000} 3.30 & {\cellcolor[HTML]{FFE8E8}} \color[HTML]{000000} 5.36 & {\cellcolor[HTML]{FFD6D6}} \color[HTML]{000000} 9.39 & {\cellcolor[HTML]{FFBCBC}} \color[HTML]{000000} 14.93 \\
4 & {\cellcolor[HTML]{FFF8F8}} \color[HTML]{000000} 1.65 & {\cellcolor[HTML]{FFF8F8}} \color[HTML]{000000} 1.45 & {\cellcolor[HTML]{FFF6F6}} \color[HTML]{000000} 2.13 & {\cellcolor[HTML]{FFF8F8}} \color[HTML]{000000} 1.57 & {\cellcolor[HTML]{FFF8F8}} \color[HTML]{000000} 1.51 & {\cellcolor[HTML]{FFF4F4}} \color[HTML]{000000} 2.29 & {\cellcolor[HTML]{FFEEEE}} \color[HTML]{000000} 3.80 & {\cellcolor[HTML]{FFE2E2}} \color[HTML]{000000} 6.67 & {\cellcolor[HTML]{FFCACA}} \color[HTML]{000000} 11.91 & {\cellcolor[HTML]{FFAAAA}} \color[HTML]{000000} 19.25 \\
5 & {\cellcolor[HTML]{FFF8F8}} \color[HTML]{000000} 1.70 & {\cellcolor[HTML]{FFFAFA}} \color[HTML]{000000} 1.00 & {\cellcolor[HTML]{FFF8F8}} \color[HTML]{000000} 1.75 & {\cellcolor[HTML]{FFFAFA}} \color[HTML]{000000} 1.23 & {\cellcolor[HTML]{FFFAFA}} \color[HTML]{000000} 1.26 & {\cellcolor[HTML]{FFF4F4}} \color[HTML]{000000} 2.29 & {\cellcolor[HTML]{FFECEC}} \color[HTML]{000000} 4.22 & {\cellcolor[HTML]{FFDCDC}} \color[HTML]{000000} 7.72 & {\cellcolor[HTML]{FFC0C0}} \color[HTML]{000000} 14.05 & {\cellcolor[HTML]{FF9898}} \color[HTML]{000000} 23.03 \\
6 & {\cellcolor[HTML]{FFFCFC}} \color[HTML]{000000} 0.84 & {\cellcolor[HTML]{FFFEFE}} \color[HTML]{000000} 0.44 & {\cellcolor[HTML]{FFFAFA}} \color[HTML]{000000} 1.30 & {\cellcolor[HTML]{FFFAFA}} \color[HTML]{000000} 0.95 & {\cellcolor[HTML]{FFFAFA}} \color[HTML]{000000} 1.02 & {\cellcolor[HTML]{FFF6F6}} \color[HTML]{000000} 2.23 & {\cellcolor[HTML]{FFEAEA}} \color[HTML]{000000} 4.50 & {\cellcolor[HTML]{FFD8D8}} \color[HTML]{000000} 8.66 & {\cellcolor[HTML]{FFB8B8}} \color[HTML]{000000} 15.86 & {\cellcolor[HTML]{FF8A8A}} \color[HTML]{000000} 26.26 \\
7 & {\cellcolor[HTML]{FFFEFE}} \color[HTML]{000000} 0.30 & {\cellcolor[HTML]{FEFEFF}} \color[HTML]{000000} -0.21 & {\cellcolor[HTML]{FFFCFC}} \color[HTML]{000000} 0.85 & {\cellcolor[HTML]{FFFCFC}} \color[HTML]{000000} 0.57 & {\cellcolor[HTML]{FFFCFC}} \color[HTML]{000000} 0.73 & {\cellcolor[HTML]{FFF6F6}} \color[HTML]{000000} 2.15 & {\cellcolor[HTML]{FFEAEA}} \color[HTML]{000000} 4.71 & {\cellcolor[HTML]{FFD6D6}} \color[HTML]{000000} 9.42 & {\cellcolor[HTML]{FFB2B2}} \color[HTML]{000000} 17.47 & {\cellcolor[HTML]{FF7E7E}} \color[HTML]{F1F1F1} 29.05 \\
8 & {\cellcolor[HTML]{FEFEFF}} \color[HTML]{000000} -0.26 & {\cellcolor[HTML]{FCFCFF}} \color[HTML]{000000} -0.89 & {\cellcolor[HTML]{FFFEFE}} \color[HTML]{000000} 0.31 & {\cellcolor[HTML]{FFFEFE}} \color[HTML]{000000} 0.10 & {\cellcolor[HTML]{FFFEFE}} \color[HTML]{000000} 0.35 & {\cellcolor[HTML]{FFF6F6}} \color[HTML]{000000} 2.02 & {\cellcolor[HTML]{FFEAEA}} \color[HTML]{000000} 4.89 & {\cellcolor[HTML]{FFD2D2}} \color[HTML]{000000} 10.07 & {\cellcolor[HTML]{FFACAC}} \color[HTML]{000000} 18.83 & {\cellcolor[HTML]{FF7272}} \color[HTML]{F1F1F1} 31.55 \\
9 & {\cellcolor[HTML]{FAFAFF}} \color[HTML]{000000} -0.97 & {\cellcolor[HTML]{F8F8FF}} \color[HTML]{000000} -1.60 & {\cellcolor[HTML]{FEFEFF}} \color[HTML]{000000} -0.28 & {\cellcolor[HTML]{FEFEFF}} \color[HTML]{000000} -0.23 & {\cellcolor[HTML]{FFFEFE}} \color[HTML]{000000} 0.07 & {\cellcolor[HTML]{FFF8F8}} \color[HTML]{000000} 1.77 & {\cellcolor[HTML]{FFEAEA}} \color[HTML]{000000} 4.91 & {\cellcolor[HTML]{FFD0D0}} \color[HTML]{000000} 10.59 & {\cellcolor[HTML]{FFA6A6}} \color[HTML]{000000} 20.10 & {\cellcolor[HTML]{FF6868}} \color[HTML]{F1F1F1} 33.82 \\
10 & {\cellcolor[HTML]{F8F8FF}} \color[HTML]{000000} -1.67 & {\cellcolor[HTML]{F4F4FF}} \color[HTML]{000000} -2.41 & {\cellcolor[HTML]{FAFAFF}} \color[HTML]{000000} -0.97 & {\cellcolor[HTML]{FCFCFF}} \color[HTML]{000000} -0.74 & {\cellcolor[HTML]{FEFEFF}} \color[HTML]{000000} -0.32 & {\cellcolor[HTML]{FFF8F8}} \color[HTML]{000000} 1.65 & {\cellcolor[HTML]{FFE8E8}} \color[HTML]{000000} 4.99 & {\cellcolor[HTML]{FFCECE}} \color[HTML]{000000} 11.02 & {\cellcolor[HTML]{FFA2A2}} \color[HTML]{000000} 21.14 & {\cellcolor[HTML]{FF6060}} \color[HTML]{F1F1F1} 35.84 \\
12 & {\cellcolor[HTML]{F0F0FF}} \color[HTML]{000000} -3.18 & {\cellcolor[HTML]{ECECFF}} \color[HTML]{000000} -4.17 & {\cellcolor[HTML]{F4F4FF}} \color[HTML]{000000} -2.28 & {\cellcolor[HTML]{F6F6FF}} \color[HTML]{000000} -1.83 & {\cellcolor[HTML]{FAFAFF}} \color[HTML]{000000} -1.24 & {\cellcolor[HTML]{FFFAFA}} \color[HTML]{000000} 1.07 & {\cellcolor[HTML]{FFEAEA}} \color[HTML]{000000} 4.90 & {\cellcolor[HTML]{FFCACA}} \color[HTML]{000000} 11.74 & {\cellcolor[HTML]{FF9898}} \color[HTML]{000000} 23.02 & {\cellcolor[HTML]{FF5050}} \color[HTML]{F1F1F1} 39.42 \\
14 & {\cellcolor[HTML]{EAEAFF}} \color[HTML]{000000} -4.84 & {\cellcolor[HTML]{E4E4FF}} \color[HTML]{000000} -5.96 & {\cellcolor[HTML]{EEEEFF}} \color[HTML]{000000} -3.66 & {\cellcolor[HTML]{F2F2FF}} \color[HTML]{000000} -2.91 & {\cellcolor[HTML]{F6F6FF}} \color[HTML]{000000} -2.10 & {\cellcolor[HTML]{FFFEFE}} \color[HTML]{000000} 0.41 & {\cellcolor[HTML]{FFEAEA}} \color[HTML]{000000} 4.68 & {\cellcolor[HTML]{FFC8C8}} \color[HTML]{000000} 12.26 & {\cellcolor[HTML]{FF9292}} \color[HTML]{000000} 24.54 & {\cellcolor[HTML]{FF4242}} \color[HTML]{F1F1F1} 42.42 \\
16 & {\cellcolor[HTML]{E0E0FF}} \color[HTML]{000000} -6.78 & {\cellcolor[HTML]{DCDCFF}} \color[HTML]{000000} -7.68 & {\cellcolor[HTML]{E8E8FF}} \color[HTML]{000000} -5.18 & {\cellcolor[HTML]{EEEEFF}} \color[HTML]{000000} -4.01 & {\cellcolor[HTML]{F2F2FF}} \color[HTML]{000000} -3.06 & {\cellcolor[HTML]{FEFEFF}} \color[HTML]{000000} -0.19 & {\cellcolor[HTML]{FFECEC}} \color[HTML]{000000} 4.23 & {\cellcolor[HTML]{FFC8C8}} \color[HTML]{000000} 12.52 & {\cellcolor[HTML]{FF8C8C}} \color[HTML]{000000} 25.82 & {\cellcolor[HTML]{FF3636}} \color[HTML]{F1F1F1} 45.01 \\
18 & {\cellcolor[HTML]{DADAFF}} \color[HTML]{000000} -8.53 & {\cellcolor[HTML]{D4D4FF}} \color[HTML]{000000} -9.55 & {\cellcolor[HTML]{E2E2FF}} \color[HTML]{000000} -6.74 & {\cellcolor[HTML]{E8E8FF}} \color[HTML]{000000} -5.17 & {\cellcolor[HTML]{EEEEFF}} \color[HTML]{000000} -4.00 & {\cellcolor[HTML]{FCFCFF}} \color[HTML]{000000} -0.85 & {\cellcolor[HTML]{FFEEEE}} \color[HTML]{000000} 4.02 & {\cellcolor[HTML]{FFC6C6}} \color[HTML]{000000} 12.72 & {\cellcolor[HTML]{FF8888}} \color[HTML]{F1F1F1} 26.97 & {\cellcolor[HTML]{FF2C2C}} \color[HTML]{F1F1F1} 47.36 \\
20 & {\cellcolor[HTML]{D2D2FF}} \color[HTML]{000000} -10.20 & {\cellcolor[HTML]{CCCCFF}} \color[HTML]{000000} -11.37 & {\cellcolor[HTML]{DADAFF}} \color[HTML]{000000} -8.32 & {\cellcolor[HTML]{E2E2FF}} \color[HTML]{000000} -6.35 & {\cellcolor[HTML]{EAEAFF}} \color[HTML]{000000} -4.83 & {\cellcolor[HTML]{F8F8FF}} \color[HTML]{000000} -1.67 & {\cellcolor[HTML]{FFF0F0}} \color[HTML]{000000} 3.57 & {\cellcolor[HTML]{FFC6C6}} \color[HTML]{000000} 12.85 & {\cellcolor[HTML]{FF8484}} \color[HTML]{F1F1F1} 27.89 & {\cellcolor[HTML]{FF2222}} \color[HTML]{F1F1F1} 49.52 \\
25 & {\cellcolor[HTML]{BEBEFF}} \color[HTML]{000000} -14.63 & {\cellcolor[HTML]{BABAFF}} \color[HTML]{000000} -15.64 & {\cellcolor[HTML]{CACAFF}} \color[HTML]{000000} -11.74 & {\cellcolor[HTML]{D6D6FF}} \color[HTML]{000000} -9.07 & {\cellcolor[HTML]{E0E0FF}} \color[HTML]{000000} -7.12 & {\cellcolor[HTML]{F0F0FF}} \color[HTML]{000000} -3.38 & {\cellcolor[HTML]{FFF4F4}} \color[HTML]{000000} 2.51 & {\cellcolor[HTML]{FFC6C6}} \color[HTML]{000000} 12.93 & {\cellcolor[HTML]{FF7C7C}} \color[HTML]{F1F1F1} 29.64 & {\cellcolor[HTML]{FF1010}} \color[HTML]{F1F1F1} 53.86 \\
30 & {\cellcolor[HTML]{ACACFF}} \color[HTML]{000000} -18.50 & {\cellcolor[HTML]{A8A8FF}} \color[HTML]{000000} -19.63 & {\cellcolor[HTML]{BCBCFF}} \color[HTML]{000000} -15.13 & {\cellcolor[HTML]{CCCCFF}} \color[HTML]{000000} -11.69 & {\cellcolor[HTML]{D6D6FF}} \color[HTML]{000000} -9.34 & {\cellcolor[HTML]{E8E8FF}} \color[HTML]{000000} -5.13 & {\cellcolor[HTML]{FFF8F8}} \color[HTML]{000000} 1.39 & {\cellcolor[HTML]{FFC6C6}} \color[HTML]{000000} 12.79 & {\cellcolor[HTML]{FF7676}} \color[HTML]{F1F1F1} 30.96 & {\cellcolor[HTML]{FF0000}} \color[HTML]{F1F1F1} 57.59 \\
\end{tabular}

\end{center}
\end{table}

\begin{table}[ht]
\caption{Updated BC Table 4 with a few extra columns and rows.}
\smallskip
\tiny
\begin{center}
\begin{tabular}{lrrrrrrrrrrrrr}
$\sin(\theta)$ & 0 & 0.05 & 0.1 & 0.2 & 0.3 & 0.4 & 0.5 & 0.6 & 0.7 & 0.8 & 0.9 & 0.99 & 1 \\
$x$ &  &  &  &  &  &  &  &  &  &  &  &  &  \\
0.001 & 9990 & 9990 & 9990 & 9990 & 9990 & 9990 & 9990 & 9990 & 9990 & 9990 & 9990 & 9990 & 9990 \\
0.01 & 9901 & 9901 & 9901 & 9901 & 9901 & 9901 & 9901 & 9901 & 9901 & 9901 & 9901 & 9902 & 9902 \\
0.1 & 9098 & 9098 & 9098 & 9098 & 9100 & 9102 & 9105 & 9109 & 9114 & 9120 & 9127 & 9134 & 9135 \\
0.2 & 8359 & 8359 & 8359 & 8362 & 8366 & 8373 & 8382 & 8394 & 8409 & 8426 & 8447 & 8468 & 8470 \\
0.4 & 7232 & 7232 & 7233 & 7240 & 7252 & 7270 & 7294 & 7324 & 7358 & 7399 & 7444 & 7490 & 7495 \\
0.6 & 6420 & 6421 & 6423 & 6435 & 6455 & 6483 & 6519 & 6562 & 6612 & 6669 & 6731 & 6793 & 6800 \\
0.8 & 5810 & 5811 & 5815 & 5831 & 5858 & 5894 & 5939 & 5992 & 6052 & 6119 & 6193 & 6264 & 6272 \\
1 & 5336 & 5338 & 5343 & 5363 & 5395 & 5437 & 5488 & 5547 & 5614 & 5687 & 5767 & 5843 & 5852 \\
1.5 & 4509 & 4512 & 4519 & 4546 & 4586 & 4637 & 4696 & 4763 & 4836 & 4915 & 5000 & 5081 & 5090 \\
2 & 3969 & 3972 & 3982 & 4013 & 4057 & 4111 & 4172 & 4241 & 4315 & 4394 & 4478 & 4558 & 4567 \\
2.5 & 3584 & 3588 & 3598 & 3632 & 3677 & 3732 & 3793 & 3861 & 3934 & 4011 & 4093 & 4170 & 4178 \\
3 & 3291 & 3296 & 3307 & 3342 & 3388 & 3442 & 3503 & 3569 & 3639 & 3714 & 3793 & 3867 & 3875 \\
4 & 2872 & 2877 & 2889 & 2925 & 2970 & 3022 & 3079 & 3142 & 3208 & 3277 & 3350 & 3418 & 3426 \\
5 & 2580 & 2586 & 2598 & 2633 & 2676 & 2726 & 2781 & 2839 & 2901 & 2966 & 3034 & 3097 & 3104 \\
6 & 2361 & 2368 & 2380 & 2414 & 2456 & 2503 & 2555 & 2610 & 2668 & 2730 & 2793 & 2852 & 2859 \\
7 & 2191 & 2197 & 2210 & 2243 & 2283 & 2328 & 2377 & 2429 & 2484 & 2542 & 2602 & 2658 & 2664 \\
8 & 2052 & 2059 & 2071 & 2103 & 2141 & 2185 & 2231 & 2281 & 2334 & 2389 & 2446 & 2498 & 2504 \\
9 & 1937 & 1944 & 1956 & 1987 & 2024 & 2065 & 2110 & 2157 & 2208 & 2260 & 2314 & 2365 & 2370 \\
10 & 1839 & 1846 & 1858 & 1888 & 1923 & 1963 & 2006 & 2052 & 2100 & 2150 & 2202 & 2250 & 2256 \\
12 & 1681 & 1688 & 1699 & 1727 & 1761 & 1798 & 1838 & 1880 & 1925 & 1971 & 2019 & 2064 & 2069 \\
14 & 1558 & 1565 & 1575 & 1602 & 1634 & 1668 & 1706 & 1746 & 1787 & 1831 & 1876 & 1918 & 1922 \\
16 & 1459 & 1465 & 1475 & 1501 & 1530 & 1563 & 1599 & 1636 & 1676 & 1717 & 1759 & 1798 & 1803 \\
18 & 1376 & 1382 & 1392 & 1416 & 1445 & 1476 & 1510 & 1545 & 1583 & 1622 & 1662 & 1699 & 1703 \\
20 & 1306 & 1312 & 1321 & 1345 & 1372 & 1402 & 1434 & 1468 & 1504 & 1541 & 1579 & 1614 & 1618 \\
25 & 1169 & 1175 & 1183 & 1205 & 1229 & 1256 & 1286 & 1316 & 1348 & 1382 & 1416 & 1448 & 1452 \\
30 & 1068 & 1073 & 1081 & 1101 & 1124 & 1149 & 1175 & 1204 & 1233 & 1264 & 1295 & 1325 & 1328 \\
100 & 0586 & 0589 & 0594 & 0605 & 0618 & 0632 & 0647 & 0663 & 0680 & 0697 & 0714 & 0731 & 0733 \\
1000 & 0185 & 0187 & 0188 & 0192 & 0196 & 0200 & 0205 & 0210 & 0215 & 0221 & 0227 & 0232 & 0232 \\
\end{tabular}

\end{center}
\end{table}

\begin{table}[ht]
\caption{Precision of original BC Table 4 compared to updated values (NB: MENTION PRECISION DEF USED).}
\smallskip
\tiny
\begin{center}
\begin{tabular}{lrrrrrrrrrr}
$\sin(\theta)$ & 0.05 & 0.1 & 0.2 & 0.3 & 0.4 & 0.5 & 0.6 & 0.7 & 0.8 & 0.9 \\
$x$ &  &  &  &  &  &  &  &  &  &  \\
0.1 & {\cellcolor[HTML]{3232FF}} \color[HTML]{F1F1F1} -10.26 & {\cellcolor[HTML]{3A3AFF}} \color[HTML]{F1F1F1} -9.94 & {\cellcolor[HTML]{4444FF}} \color[HTML]{F1F1F1} -9.35 & {\cellcolor[HTML]{4040FF}} \color[HTML]{F1F1F1} -9.63 & {\cellcolor[HTML]{3A3AFF}} \color[HTML]{F1F1F1} -9.89 & {\cellcolor[HTML]{3232FF}} \color[HTML]{F1F1F1} -10.26 & {\cellcolor[HTML]{2828FF}} \color[HTML]{F1F1F1} -10.76 & {\cellcolor[HTML]{2020FF}} \color[HTML]{F1F1F1} -11.16 & {\cellcolor[HTML]{2020FF}} \color[HTML]{F1F1F1} -11.24 & {\cellcolor[HTML]{2222FF}} \color[HTML]{F1F1F1} -11.11 \\
0.2 & {\cellcolor[HTML]{A0A0FF}} \color[HTML]{F1F1F1} -4.74 & {\cellcolor[HTML]{A6A6FF}} \color[HTML]{000000} -4.47 & {\cellcolor[HTML]{B0B0FF}} \color[HTML]{000000} -4.00 & {\cellcolor[HTML]{AAAAFF}} \color[HTML]{000000} -4.28 & {\cellcolor[HTML]{A6A6FF}} \color[HTML]{000000} -4.47 & {\cellcolor[HTML]{A0A0FF}} \color[HTML]{F1F1F1} -4.77 & {\cellcolor[HTML]{9696FF}} \color[HTML]{F1F1F1} -5.24 & {\cellcolor[HTML]{9292FF}} \color[HTML]{F1F1F1} -5.53 & {\cellcolor[HTML]{9292FF}} \color[HTML]{F1F1F1} -5.43 & {\cellcolor[HTML]{9898FF}} \color[HTML]{F1F1F1} -5.14 \\
0.4 & {\cellcolor[HTML]{D6D6FF}} \color[HTML]{000000} -2.10 & {\cellcolor[HTML]{DADAFF}} \color[HTML]{000000} -1.86 & {\cellcolor[HTML]{E2E2FF}} \color[HTML]{000000} -1.49 & {\cellcolor[HTML]{DCDCFF}} \color[HTML]{000000} -1.73 & {\cellcolor[HTML]{D8D8FF}} \color[HTML]{000000} -1.92 & {\cellcolor[HTML]{D4D4FF}} \color[HTML]{000000} -2.11 & {\cellcolor[HTML]{D0D0FF}} \color[HTML]{000000} -2.41 & {\cellcolor[HTML]{CCCCFF}} \color[HTML]{000000} -2.51 & {\cellcolor[HTML]{E6E6FF}} \color[HTML]{000000} -1.29 & {\cellcolor[HTML]{E4E4FF}} \color[HTML]{000000} -1.38 \\
0.6 & {\cellcolor[HTML]{E4E4FF}} \color[HTML]{000000} -1.33 & {\cellcolor[HTML]{E8E8FF}} \color[HTML]{000000} -1.15 & {\cellcolor[HTML]{FFDCDC}} \color[HTML]{000000} 1.72 & {\cellcolor[HTML]{EAEAFF}} \color[HTML]{000000} -1.04 & {\cellcolor[HTML]{E8E8FF}} \color[HTML]{000000} -1.19 & {\cellcolor[HTML]{E4E4FF}} \color[HTML]{000000} -1.31 & {\cellcolor[HTML]{E2E2FF}} \color[HTML]{000000} -1.48 & {\cellcolor[HTML]{E4E4FF}} \color[HTML]{000000} -1.39 & {\cellcolor[HTML]{F2F2FF}} \color[HTML]{000000} -0.68 & {\cellcolor[HTML]{FFF8F8}} \color[HTML]{000000} 0.35 \\
0.8 & {\cellcolor[HTML]{EAEAFF}} \color[HTML]{000000} -1.01 & {\cellcolor[HTML]{EEEEFF}} \color[HTML]{000000} -0.87 & {\cellcolor[HTML]{F4F4FF}} \color[HTML]{000000} -0.53 & {\cellcolor[HTML]{F0F0FF}} \color[HTML]{000000} -0.77 & {\cellcolor[HTML]{EEEEFF}} \color[HTML]{000000} -0.90 & {\cellcolor[HTML]{ECECFF}} \color[HTML]{000000} -0.98 & {\cellcolor[HTML]{EAEAFF}} \color[HTML]{000000} -1.04 & {\cellcolor[HTML]{F0F0FF}} \color[HTML]{000000} -0.78 & {\cellcolor[HTML]{FFFCFC}} \color[HTML]{000000} 0.10 & {\cellcolor[HTML]{FFE4E4}} \color[HTML]{000000} 1.40 \\
1 & {\cellcolor[HTML]{EEEEFF}} \color[HTML]{000000} -0.88 & {\cellcolor[HTML]{F0F0FF}} \color[HTML]{000000} -0.75 & {\cellcolor[HTML]{F6F6FF}} \color[HTML]{000000} -0.45 & {\cellcolor[HTML]{F2F2FF}} \color[HTML]{000000} -0.66 & {\cellcolor[HTML]{F0F0FF}} \color[HTML]{000000} -0.80 & {\cellcolor[HTML]{EEEEFF}} \color[HTML]{000000} -0.84 & {\cellcolor[HTML]{F0F0FF}} \color[HTML]{000000} -0.79 & {\cellcolor[HTML]{F6F6FF}} \color[HTML]{000000} -0.45 & {\cellcolor[HTML]{FFF4F4}} \color[HTML]{000000} 0.60 & {\cellcolor[HTML]{FFD6D6}} \color[HTML]{000000} 2.08 \\
1.5 & {\cellcolor[HTML]{ECECFF}} \color[HTML]{000000} -0.94 & {\cellcolor[HTML]{EEEEFF}} \color[HTML]{000000} -0.86 & {\cellcolor[HTML]{F4F4FF}} \color[HTML]{000000} -0.55 & {\cellcolor[HTML]{F0F0FF}} \color[HTML]{000000} -0.79 & {\cellcolor[HTML]{EEEEFF}} \color[HTML]{000000} -0.88 & {\cellcolor[HTML]{EEEEFF}} \color[HTML]{000000} -0.81 & {\cellcolor[HTML]{F4F4FF}} \color[HTML]{000000} -0.58 & {\cellcolor[HTML]{FEFEFF}} \color[HTML]{000000} -0.00 & {\cellcolor[HTML]{FFE6E6}} \color[HTML]{000000} 1.29 & {\cellcolor[HTML]{FFC2C2}} \color[HTML]{000000} 3.09 \\
2 & {\cellcolor[HTML]{E8E8FF}} \color[HTML]{000000} -1.19 & {\cellcolor[HTML]{E8E8FF}} \color[HTML]{000000} -1.15 & {\cellcolor[HTML]{EEEEFF}} \color[HTML]{000000} -0.82 & {\cellcolor[HTML]{EAEAFF}} \color[HTML]{000000} -1.06 & {\cellcolor[HTML]{E8E8FF}} \color[HTML]{000000} -1.13 & {\cellcolor[HTML]{ECECFF}} \color[HTML]{000000} -0.99 & {\cellcolor[HTML]{F2F2FF}} \color[HTML]{000000} -0.60 & {\cellcolor[HTML]{FFFAFA}} \color[HTML]{000000} 0.24 & {\cellcolor[HTML]{FFD8D8}} \color[HTML]{000000} 1.98 & {\cellcolor[HTML]{FFA8A8}} \color[HTML]{000000} 4.41 \\
2.5 & {\cellcolor[HTML]{E2E2FF}} \color[HTML]{000000} -1.49 & {\cellcolor[HTML]{E2E2FF}} \color[HTML]{000000} -1.45 & {\cellcolor[HTML]{E8E8FF}} \color[HTML]{000000} -1.12 & {\cellcolor[HTML]{E4E4FF}} \color[HTML]{000000} -1.34 & {\cellcolor[HTML]{E2E2FF}} \color[HTML]{000000} -1.41 & {\cellcolor[HTML]{E8E8FF}} \color[HTML]{000000} -1.17 & {\cellcolor[HTML]{F2F2FF}} \color[HTML]{000000} -0.67 & {\cellcolor[HTML]{FFF6F6}} \color[HTML]{000000} 0.41 & {\cellcolor[HTML]{FFCCCC}} \color[HTML]{000000} 2.54 & {\cellcolor[HTML]{FF9292}} \color[HTML]{000000} 5.53 \\
3 & {\cellcolor[HTML]{DCDCFF}} \color[HTML]{000000} -1.79 & {\cellcolor[HTML]{DCDCFF}} \color[HTML]{000000} -1.79 & {\cellcolor[HTML]{E2E2FF}} \color[HTML]{000000} -1.41 & {\cellcolor[HTML]{DEDEFF}} \color[HTML]{000000} -1.66 & {\cellcolor[HTML]{DEDEFF}} \color[HTML]{000000} -1.69 & {\cellcolor[HTML]{E4E4FF}} \color[HTML]{000000} -1.39 & {\cellcolor[HTML]{F0F0FF}} \color[HTML]{000000} -0.75 & {\cellcolor[HTML]{FFF4F4}} \color[HTML]{000000} 0.54 & {\cellcolor[HTML]{FFC4C4}} \color[HTML]{000000} 2.98 & {\cellcolor[HTML]{FF7E7E}} \color[HTML]{F1F1F1} 6.46 \\
4 & {\cellcolor[HTML]{D0D0FF}} \color[HTML]{000000} -2.36 & {\cellcolor[HTML]{CECEFF}} \color[HTML]{000000} -2.43 & {\cellcolor[HTML]{D8D8FF}} \color[HTML]{000000} -1.97 & {\cellcolor[HTML]{D2D2FF}} \color[HTML]{000000} -2.21 & {\cellcolor[HTML]{D4D4FF}} \color[HTML]{000000} -2.18 & {\cellcolor[HTML]{DCDCFF}} \color[HTML]{000000} -1.77 & {\cellcolor[HTML]{ECECFF}} \color[HTML]{000000} -0.94 & {\cellcolor[HTML]{FFF2F2}} \color[HTML]{000000} 0.66 & {\cellcolor[HTML]{FFB6B6}} \color[HTML]{000000} 3.65 & {\cellcolor[HTML]{FF6262}} \color[HTML]{F1F1F1} 7.93 \\
5 & {\cellcolor[HTML]{C4C4FF}} \color[HTML]{000000} -2.92 & {\cellcolor[HTML]{C2C2FF}} \color[HTML]{000000} -3.04 & {\cellcolor[HTML]{CCCCFF}} \color[HTML]{000000} -2.54 & {\cellcolor[HTML]{CACAFF}} \color[HTML]{000000} -2.71 & {\cellcolor[HTML]{CACAFF}} \color[HTML]{000000} -2.64 & {\cellcolor[HTML]{D4D4FF}} \color[HTML]{000000} -2.14 & {\cellcolor[HTML]{E8E8FF}} \color[HTML]{000000} -1.16 & {\cellcolor[HTML]{FFF2F2}} \color[HTML]{000000} 0.69 & {\cellcolor[HTML]{FFAEAE}} \color[HTML]{000000} 4.11 & {\cellcolor[HTML]{FF4C4C}} \color[HTML]{F1F1F1} 9.00 \\
6 & {\cellcolor[HTML]{BABAFF}} \color[HTML]{000000} -3.50 & {\cellcolor[HTML]{B8B8FF}} \color[HTML]{000000} -3.58 & {\cellcolor[HTML]{C2C2FF}} \color[HTML]{000000} -3.08 & {\cellcolor[HTML]{C0C0FF}} \color[HTML]{000000} -3.18 & {\cellcolor[HTML]{C2C2FF}} \color[HTML]{000000} -3.09 & {\cellcolor[HTML]{CECEFF}} \color[HTML]{000000} -2.50 & {\cellcolor[HTML]{E2E2FF}} \color[HTML]{000000} -1.42 & {\cellcolor[HTML]{E0E0FF}} \color[HTML]{000000} -1.51 & {\cellcolor[HTML]{FFA8A8}} \color[HTML]{000000} 4.41 & {\cellcolor[HTML]{FF3C3C}} \color[HTML]{F1F1F1} 9.84 \\
7 & {\cellcolor[HTML]{AEAEFF}} \color[HTML]{000000} -4.06 & {\cellcolor[HTML]{ACACFF}} \color[HTML]{000000} -4.19 & {\cellcolor[HTML]{B8B8FF}} \color[HTML]{000000} -3.59 & {\cellcolor[HTML]{B6B6FF}} \color[HTML]{000000} -3.66 & {\cellcolor[HTML]{B8B8FF}} \color[HTML]{000000} -3.55 & {\cellcolor[HTML]{C6C6FF}} \color[HTML]{000000} -2.85 & {\cellcolor[HTML]{9C9CFF}} \color[HTML]{F1F1F1} -4.94 & {\cellcolor[HTML]{FFF4F4}} \color[HTML]{000000} 0.59 & {\cellcolor[HTML]{FFA4A4}} \color[HTML]{000000} 4.60 & {\cellcolor[HTML]{FF2E2E}} \color[HTML]{F1F1F1} 10.49 \\
8 & {\cellcolor[HTML]{9292FF}} \color[HTML]{F1F1F1} -5.43 & {\cellcolor[HTML]{A0A0FF}} \color[HTML]{F1F1F1} -4.73 & {\cellcolor[HTML]{AEAEFF}} \color[HTML]{000000} -4.09 & {\cellcolor[HTML]{AEAEFF}} \color[HTML]{000000} -4.08 & {\cellcolor[HTML]{B0B0FF}} \color[HTML]{000000} -3.96 & {\cellcolor[HTML]{BEBEFF}} \color[HTML]{000000} -3.24 & {\cellcolor[HTML]{D8D8FF}} \color[HTML]{000000} -1.94 & {\cellcolor[HTML]{FFF6F6}} \color[HTML]{000000} 0.49 & {\cellcolor[HTML]{FFA0A0}} \color[HTML]{000000} 4.75 & {\cellcolor[HTML]{FF2424}} \color[HTML]{F1F1F1} 11.02 \\
9 & {\cellcolor[HTML]{9A9AFF}} \color[HTML]{F1F1F1} -5.08 & {\cellcolor[HTML]{9696FF}} \color[HTML]{F1F1F1} -5.25 & {\cellcolor[HTML]{A4A4FF}} \color[HTML]{000000} -4.61 & {\cellcolor[HTML]{A4A4FF}} \color[HTML]{000000} -4.53 & {\cellcolor[HTML]{A8A8FF}} \color[HTML]{000000} -4.36 & {\cellcolor[HTML]{B8B8FF}} \color[HTML]{000000} -3.59 & {\cellcolor[HTML]{D4D4FF}} \color[HTML]{000000} -2.20 & {\cellcolor[HTML]{FFF8F8}} \color[HTML]{000000} 0.34 & {\cellcolor[HTML]{FF9E9E}} \color[HTML]{000000} 4.87 & {\cellcolor[HTML]{FF1C1C}} \color[HTML]{F1F1F1} 11.40 \\
10 & {\cellcolor[HTML]{8E8EFF}} \color[HTML]{F1F1F1} -5.63 & {\cellcolor[HTML]{8C8CFF}} \color[HTML]{F1F1F1} -5.80 & {\cellcolor[HTML]{9A9AFF}} \color[HTML]{F1F1F1} -5.07 & {\cellcolor[HTML]{9C9CFF}} \color[HTML]{F1F1F1} -4.96 & {\cellcolor[HTML]{A0A0FF}} \color[HTML]{F1F1F1} -4.75 & {\cellcolor[HTML]{B0B0FF}} \color[HTML]{000000} -3.95 & {\cellcolor[HTML]{CECEFF}} \color[HTML]{000000} -2.48 & {\cellcolor[HTML]{FFFCFC}} \color[HTML]{000000} 0.19 & {\cellcolor[HTML]{FF9E9E}} \color[HTML]{000000} 4.88 & {\cellcolor[HTML]{FF1616}} \color[HTML]{F1F1F1} 11.71 \\
12 & {\cellcolor[HTML]{7C7CFF}} \color[HTML]{F1F1F1} -6.57 & {\cellcolor[HTML]{7878FF}} \color[HTML]{F1F1F1} -6.77 & {\cellcolor[HTML]{8888FF}} \color[HTML]{F1F1F1} -5.98 & {\cellcolor[HTML]{8C8CFF}} \color[HTML]{F1F1F1} -5.78 & {\cellcolor[HTML]{9292FF}} \color[HTML]{F1F1F1} -5.50 & {\cellcolor[HTML]{A4A4FF}} \color[HTML]{000000} -4.62 & {\cellcolor[HTML]{C2C2FF}} \color[HTML]{000000} -3.05 & {\cellcolor[HTML]{FCFCFF}} \color[HTML]{000000} -0.15 & {\cellcolor[HTML]{FF9E9E}} \color[HTML]{000000} 4.90 & {\cellcolor[HTML]{FF0C0C}} \color[HTML]{F1F1F1} 12.21 \\
14 & {\cellcolor[HTML]{6868FF}} \color[HTML]{F1F1F1} -7.58 & {\cellcolor[HTML]{6464FF}} \color[HTML]{F1F1F1} -7.76 & {\cellcolor[HTML]{7676FF}} \color[HTML]{F1F1F1} -6.87 & {\cellcolor[HTML]{7A7AFF}} \color[HTML]{F1F1F1} -6.65 & {\cellcolor[HTML]{8282FF}} \color[HTML]{F1F1F1} -6.26 & {\cellcolor[HTML]{9696FF}} \color[HTML]{F1F1F1} -5.27 & {\cellcolor[HTML]{B8B8FF}} \color[HTML]{000000} -3.59 & {\cellcolor[HTML]{F4F4FF}} \color[HTML]{000000} -0.53 & {\cellcolor[HTML]{FFA0A0}} \color[HTML]{000000} 4.76 & {\cellcolor[HTML]{FF0606}} \color[HTML]{F1F1F1} 12.48 \\
16 & {\cellcolor[HTML]{5656FF}} \color[HTML]{F1F1F1} -8.46 & {\cellcolor[HTML]{5252FF}} \color[HTML]{F1F1F1} -8.68 & {\cellcolor[HTML]{6666FF}} \color[HTML]{F1F1F1} -7.70 & {\cellcolor[HTML]{6C6CFF}} \color[HTML]{F1F1F1} -7.35 & {\cellcolor[HTML]{7474FF}} \color[HTML]{F1F1F1} -7.00 & {\cellcolor[HTML]{8888FF}} \color[HTML]{F1F1F1} -5.94 & {\cellcolor[HTML]{ACACFF}} \color[HTML]{000000} -4.18 & {\cellcolor[HTML]{ECECFF}} \color[HTML]{000000} -0.94 & {\cellcolor[HTML]{FFA4A4}} \color[HTML]{000000} 4.62 & {\cellcolor[HTML]{FF0202}} \color[HTML]{F1F1F1} 12.67 \\
18 & {\cellcolor[HTML]{4646FF}} \color[HTML]{F1F1F1} -9.34 & {\cellcolor[HTML]{4040FF}} \color[HTML]{F1F1F1} -9.55 & {\cellcolor[HTML]{5656FF}} \color[HTML]{F1F1F1} -8.49 & {\cellcolor[HTML]{5E5EFF}} \color[HTML]{F1F1F1} -8.08 & {\cellcolor[HTML]{6666FF}} \color[HTML]{F1F1F1} -7.66 & {\cellcolor[HTML]{7C7CFF}} \color[HTML]{F1F1F1} -6.61 & {\cellcolor[HTML]{A0A0FF}} \color[HTML]{F1F1F1} -4.75 & {\cellcolor[HTML]{E4E4FF}} \color[HTML]{000000} -1.37 & {\cellcolor[HTML]{FFA8A8}} \color[HTML]{000000} 4.34 & {\cellcolor[HTML]{FF0000}} \color[HTML]{F1F1F1} 12.84 \\
20 & {\cellcolor[HTML]{3636FF}} \color[HTML]{F1F1F1} -10.13 & {\cellcolor[HTML]{3232FF}} \color[HTML]{F1F1F1} -10.32 & {\cellcolor[HTML]{4444FF}} \color[HTML]{F1F1F1} -9.35 & {\cellcolor[HTML]{4E4EFF}} \color[HTML]{F1F1F1} -8.89 & {\cellcolor[HTML]{5A5AFF}} \color[HTML]{F1F1F1} -8.34 & {\cellcolor[HTML]{6E6EFF}} \color[HTML]{F1F1F1} -7.26 & {\cellcolor[HTML]{9696FF}} \color[HTML]{F1F1F1} -5.32 & {\cellcolor[HTML]{DADAFF}} \color[HTML]{000000} -1.84 & {\cellcolor[HTML]{FFAEAE}} \color[HTML]{000000} 4.11 & {\cellcolor[HTML]{FF0000}} \color[HTML]{F1F1F1} 12.86 \\
25 & {\cellcolor[HTML]{0E0EFF}} \color[HTML]{F1F1F1} -12.06 & {\cellcolor[HTML]{0A0AFF}} \color[HTML]{F1F1F1} -12.28 & {\cellcolor[HTML]{2222FF}} \color[HTML]{F1F1F1} -11.09 & {\cellcolor[HTML]{2E2EFF}} \color[HTML]{F1F1F1} -10.52 & {\cellcolor[HTML]{3A3AFF}} \color[HTML]{F1F1F1} -9.91 & {\cellcolor[HTML]{5050FF}} \color[HTML]{F1F1F1} -8.75 & {\cellcolor[HTML]{7A7AFF}} \color[HTML]{F1F1F1} -6.71 & {\cellcolor[HTML]{C4C4FF}} \color[HTML]{000000} -2.92 & {\cellcolor[HTML]{FFBCBC}} \color[HTML]{000000} 3.35 & {\cellcolor[HTML]{FF0202}} \color[HTML]{F1F1F1} 12.76 \\
30 & {\cellcolor[HTML]{0000FF}} \color[HTML]{F1F1F1} -13.80 & {\cellcolor[HTML]{0000FF}} \color[HTML]{F1F1F1} -14.07 & {\cellcolor[HTML]{0000FF}} \color[HTML]{F1F1F1} -12.79 & {\cellcolor[HTML]{0E0EFF}} \color[HTML]{F1F1F1} -12.07 & {\cellcolor[HTML]{1C1CFF}} \color[HTML]{F1F1F1} -11.37 & {\cellcolor[HTML]{3636FF}} \color[HTML]{F1F1F1} -10.07 & {\cellcolor[HTML]{6060FF}} \color[HTML]{F1F1F1} -7.94 & {\cellcolor[HTML]{B0B0FF}} \color[HTML]{000000} -3.98 & {\cellcolor[HTML]{FFCCCC}} \color[HTML]{000000} 2.56 & {\cellcolor[HTML]{FF0606}} \color[HTML]{F1F1F1} 12.48 \\
\end{tabular}

\end{center}
\end{table}


%%\begin{figure}[ht] %
%%\begin{center}
%%\includegraphics[width=0.6\textwidth]{generated/prec_classicrecipe_2d_primary.pdf}
%%\end{center}
%%\caption{precision 2d classic recipe (vs. new lux recipe). This is primary.}
%%\labfig{prec2dplot:primary}
%%\end{figure}
%%
%%\begin{figure}[ht] %
%%\begin{center}
%%\includegraphics[width=0.6\textwidth]{generated/prec_classicrecipe_2d_scndgauss.pdf}
%%\end{center}
%%\caption{precision 2d classic recipe (vs. new lux recipe). This is scndgauss.}
%%\labfig{prec2dplot:scndgauss}
%%\end{figure}
%%
%%\begin{figure}[ht] %
%%\begin{center}
%%\includegraphics[width=0.6\textwidth]{generated/prec_classicrecipe_2d_scndlorentz.pdf}
%%\end{center}
%%\caption{precision 2d classic recipe (vs. new lux recipe). This is scndlorentz.}
%%\labfig{prec2dplot:scndlorentz}
%%\end{figure}
%%
%%\begin{figure}[ht] %
%%\begin{center}
%%\includegraphics[width=0.6\textwidth]{generated/prec_classicrecipe_2d_scndfresnel.pdf}
%%\end{center}
%%\caption{precision 2d classic recipe (vs. new lux recipe). This is scndfresnel.}
%%\labfig{prec2dplot:scndfresnel}
%%\end{figure}
%%

\section{Evaluating the integrals}
 The integrals (only general ideas for all 4 models, more stuff in appendix?).

 plot integrands

\begin{figure}[htp]
    \centering
    \begin{subfigure}[t]{0.49\textwidth}
        \centering
        \includegraphics[width=\textwidth]{generated/prec_classicrecipe_2d_primary.pdf}
        \caption{$M=P$}\labfig{prec2dplotNEW:subfig:P}
    \end{subfigure}
    \hfill
    \begin{subfigure}[t]{0.49\textwidth}
        \centering
        \includegraphics[width=\textwidth]{generated/prec_classicrecipe_2d_scndgauss.pdf}
        \caption{$M=G$}\labfig{prec2dplotNEW:subfig:G}
    \end{subfigure}
    \vspace{0.5cm}
    \begin{subfigure}[t]{0.49\textwidth}
        \centering
        \includegraphics[width=\textwidth]{generated/prec_classicrecipe_2d_scndlorentz.pdf}
        \caption{$M=L$}\labfig{prec2dplotNEW:subfig:L}
    \end{subfigure}
    \hfill
    \begin{subfigure}[t]{0.49\textwidth}
        \centering
        \includegraphics[width=\textwidth]{generated/prec_classicrecipe_2d_scndfresnel.pdf}
        \caption{$M=F$}\labfig{prec2dplotNEW:subfig:F}
    \end{subfigure}
    \caption{Bla. precision 2d classic recipe (vs. new lux recipe)}
    \labfig{prec2dplotNEW}
\end{figure}

%\begin{figure}[ht]
%    \centering
%    \begin{subfigure}{0.45\textwidth}
%        \centering
%        \includegraphics[width=\textwidth]{generated/integrand_gPgGgLgF_of_eta_g.pdf}
%        \caption{}\labfig{integrandfcts:subfig:gPGLF}
%    \end{subfigure}
%    \hfill
%    \begin{subfigure}{0.45\textwidth}
%        \centering
%        \includegraphics[width=\textwidth]{generated/integrand_gP_of_eta_g.pdf}
%        \caption{}\labfig{integrandfcts:subfig:gP}
%    \end{subfigure}
%    \caption{Bla..}
%    \labfig{prec2dplot}
%\end{figure}
%


\begin{figure}[ht]
    \centering
    \begin{subfigure}{0.45\textwidth}
        \centering
        \includegraphics[width=\textwidth]{generated/integrand_gPgGgLgF_of_eta_g.pdf}
        \caption{}\labfig{integrandfcts:subfig:gPGLF}
    \end{subfigure}
    \hfill
    \begin{subfigure}{0.45\textwidth}
        \centering
        \includegraphics[width=\textwidth]{generated/integrand_gP_of_eta_g.pdf}
        \caption{}\labfig{integrandfcts:subfig:gP}
    \end{subfigure}
    \caption{\Refsubfigfrommaincaption{integrandfcts:subfig:gPGLF} shows the
      integrand of \refeqn{yintegral} for the four different models at
      $x=1$ and $\theta45^\circ$. \Refsubfigfrommaincaption{integrandfcts:subfig:gP}
      shows the integrand for the primary extinction model ($M=P$) at various values of
      $x$ and $\theta$.}
    \labfig{integrandfcts}
\end{figure}

ARGH see \reffig{integrandfcts:subfig:gP} and \reffig{integrandfcts:subfig:gPGLF} and \reffig{integrandfcts}.

% \begin{figure}[ht] %
%\begin{center}
%\includegraphics[width=0.6\textwidth]{generated/integrand_gP_of_eta_g.pdf}
%\end{center}
%\caption{Bla.} % NB \protect\cite{...} is required in floating figures
%\labfig{integrandgP}
%\end{figure}
%
%\begin{figure}[ht] %
%\begin{center}
%\includegraphics[width=0.6\textwidth]{generated/integrand_gPgGgLgF_of_eta_g.pdf}
%\end{center}
%\caption{Bla. x=1, theta=45degree.} % NB \protect\cite{...} is required in floating figures
%\labfig{integrandgPGLF}
%\end{figure}

 mention how we evaluate the integrand (mpmath, taylor, etc.)


mention general tools: taylor + quad + tail. Fresnel needs robinson instead
since tail estimate not easy, and gauss does not need a tail estimate at all.

Show results? I.e. y(x) rainbow curves.

\section{Developing new recipes}
: The new recipes, how they are developed and what they are (only show 1,
         put 7 in appendix?)
    MENTION HERE OR IN PREVIOUS? Y=y0+sin3/2*(ypi-yo) = y0+sin3/2*ydelta. But
    AFTER showing the original recipe to make the point that one does not need
    to parameterise in theta - Mention taylor + prime+legendre+unprime + highx
    extrapolation.

bla talk about the new recipes. Two parts to the work: high res integrations +
finding reliable alg or parameterisation to produce results consistent with that.

Bla bla.. theta dep. By using \refeqn{phi} and standard rules of integrals, we can write
\begin{align}
  y_M(\theta,x) = y^0_M(x) + {\sin^{3/2}\theta}\,y^\delta_M(x\sqrt{\sin\theta})
  \labeqn{yintegralbyy0ydelta}
\end{align}
where we have defined
\begin{align}
y^0_M(x)\equiv y_M(0,x)\\
y^\pi_M(x)\equiv y_M(\pi,x)\\
y^{\Delta}_M(x)\equiv y_M^\pi(x)-y_M^0(x)
  \labeqn{y0pideltadef}
\end{align}
The advantage of this form is that reduces the numerical issue of having a
recipe for a two dimensional surface to instead having two one-dimensional
curves, which is significantly less challenging.

\begin{figure}[ht] %
\begin{center}
\includegraphics[width=0.6\textwidth]{generated/highx.pdf}
\end{center}
\caption{Bla highx 0.5 0.933.}
\labfig{highx}
\end{figure}

Extrapolation to extremely high $x$ values: Bla bla power law. Show scaling plot
and approximate fit. Mention that there will be no well determined precision of
the provided recipes at these extreme values, but that the solution with a
simple power law is both simple and trivially ensures well behaved numerical
results not matter the size of x.


\begin{algorithm}
    \caption{Algorithm for calculating $y_P(x,\sin\theta)$.
             When $x\le1000$ the result has an absolute error less than $10^{-3}\cdot\min(y_\text{true},1-y_\text{true})$.
             All variables must be double precision floating point types or better, as per the \texttt{binary64} format\protect\cite{IEEE754}.}
    \begin{algorithmic}[1]
        \Procedure{YP}{$x, \sin\theta$}
            \If{$x < 0.1$}
                \State ${y_0} \gets 1-{x}{\cdot}(0.94285714-{x}{\cdot}(0.82043-{x}{\cdot}(0.593-{x}{\cdot}(0.364-0.19{\cdot}{x}))))$
            \Else
                \If{$x>1000$}
                    \State \textbf{return} $\text{YP}(1000,\sin\theta)\cdot\sqrt{1000/x}$
                \EndIf
                \State $x_s \gets \sqrt{x} $
                \State $x' \gets (x_s-1)/(x_s+1) $
                \State ${y_0} \gets 0.518212-{x'}{\cdot}(0.93036-{x'}{\cdot}(0.182006+{x'}{\cdot}(1.10097-{x'}{\cdot}(0.62625$\\
                  \hspace{6em}$+{x'}{\cdot}(1.73562-{x'}{\cdot}(1.08506+{x'}{\cdot}(2.19459-{x'}{\cdot}(1.40451+{x'}{\cdot}(1.62083$\\
                  \hspace{6em}$-{x'}{\cdot}(1.1031+{x'}{\cdot}(0.49125-0.357611{\cdot}{x'})))))))))))$
            \EndIf
                \State $s \gets \sqrt{\sin\theta} $
                \State $u \gets x{\cdot}s$
            \If{$u < 0.1$}
                \State ${y^\Delta} \gets {u}{\cdot}{u}{\cdot}(0.41021645-{u}{\cdot}(1.1866-{u}{\cdot}(2.37-4.18{\cdot}{u}))))$
            \Else
                \State $u_s \gets \sqrt{u} $
                \State $u' \gets (u_s-1)/(u_s+1) $
                \State ${y^\Delta} \gets 0.05508+{u'}{\cdot}(0.1166-{u'}{\cdot}(0.2099+{u'}{\cdot}(0.5482-{u'}{\cdot}(0.5248+{u'}{\cdot}(1.402$\\
                  \hspace{6em}$-{u'}{\cdot}(1.168+{u'}{\cdot}(2.096-{u'}{\cdot}(2.116+{u'}{\cdot}(1.155-{u'}{\cdot}(1.952$\\
                  \hspace{6em}$-0.6046{\cdot}{u'}))))))))))$
            \EndIf
            \State \textbf{return} $y_0 + \sin\theta{\cdot}s{\cdot}y^\Delta$
        \EndProcedure
    \end{algorithmic}
    \label{alg:yfct_primary}
\end{algorithm}

\begin{algorithm}
    \caption{Algorithm for calculating $y_G(x,\sin\theta)$.
             When $x\le1000$ the result has an absolute error less than $10^{-3}\cdot\min(y_\text{true},1-y_\text{true})$.
             All variables must be double precision floating point types or better, as per the \texttt{binary64} format\protect\cite{IEEE754}.}
    \begin{algorithmic}[1]
        \Procedure{YG}{$x, \sin\theta$}
            \If{$x < 0.1$}
                \State ${y_0} \gets 1-{x}{\cdot}(1.0606602-{x}{\cdot}(0.92376-{x}{\cdot}(0.667-{x}{\cdot}(0.409-0.22{\cdot}{x}))))$
            \Else
                \If{$x>1000$}
                    \State \textbf{return} $\text{YG}(1000,\sin\theta)\cdot(0.001{\cdot}x)^{-0.933}$
                \EndIf
                \State $x_s \gets \sqrt{x} $
                \State $x' \gets (x_s-1)/(x_s+1) $
                \State ${y_0} \gets 0.4588909-{x'}{\cdot}(1.038687-{x'}{\cdot}(0.2401003+{x'}{\cdot}(1.288282-{x'}{\cdot}(0.7641972$\\
                  \hspace{6em}$+{x'}{\cdot}(1.880246-{x'}{\cdot}(1.886916+{x'}{\cdot}(2.171852-{x'}{\cdot}(3.273034+{x'}{\cdot}(0.9771599$\\
                  \hspace{6em}$-{x'}{\cdot}(2.988445-{x'}{\cdot}(0.4993548+{x'}{\cdot}(1.037121-0.4353142{\cdot}{x'}))))))))))))$
            \EndIf
                \State $s \gets \sqrt{\sin\theta} $
                \State $u \gets x{\cdot}s$
            \If{$u < 0.1$}
                \State ${y^\Delta} \gets {u}{\cdot}{u}{\cdot}(0.46188022-{u}{\cdot}(1.3333-{u}{\cdot}(2.66-4.68{\cdot}{u}))))$
            \Else
                \State $u_s \gets \sqrt{u} $
                \State $u' \gets (u_s-1)/(u_s+1) $
                \State ${y^\Delta} \gets 0.062289443+{u'}{\cdot}(0.13177896-{u'}{\cdot}(0.240705+{u'}{\cdot}(0.61857545$\\
                  \hspace{6em}$-{u'}{\cdot}(0.61744404+{u'}{\cdot}(1.4812474-{u'}{\cdot}(1.5419561+{u'}{\cdot}(1.9976424$\\
                  \hspace{6em}$-{u'}{\cdot}(2.8090858+{u'}{\cdot}(0.74297172-{u'}{\cdot}(2.3120683-0.8661981{\cdot}{u'}))))))))))$
            \EndIf
            \State \textbf{return} $y_0 + \sin\theta{\cdot}s{\cdot}y^\Delta$
        \EndProcedure
    \end{algorithmic}
    \label{alg:yfct_scndgauss}
\end{algorithm}

\begin{algorithm}
    \caption{Algorithm for calculating $y_L(x,\sin\theta)$.
             When $x\le1000$ the result has an absolute error less than $10^{-3}\cdot\min(y_\text{true},1-y_\text{true})$.
             All variables must be double precision floating point types or better, as per the \texttt{binary64} format\protect\cite{IEEE754}.}
    \begin{algorithmic}[1]
        \Procedure{YL}{$x, \sin\theta$}
            \If{$x < 0.1$}
                \State ${y_0} \gets 1-{x}{\cdot}(1-{x}{\cdot}(1.0667-{x}{\cdot}(0.988-{x}{\cdot}(0.79-0.55{\cdot}{x}))))$
            \Else
                \If{$x>1000$}
                    \State \textbf{return} $\text{YL}(1000,\sin\theta)\cdot\sqrt{1000/x}$
                \EndIf
                \State $x_s \gets \sqrt{x} $
                \State $x' \gets (x_s-1)/(x_s+1) $
                \State ${y_0} \gets 0.53379-{x'}{\cdot}(0.84182-{x'}{\cdot}(0.16806+{x'}{\cdot}(0.65124-{x'}{\cdot}(0.67623$\\
                  \hspace{6em}$+{x'}{\cdot}(0.47199-{x'}{\cdot}(1.0872+{x'}{\cdot}(0.030142-{x'}{\cdot}(0.91361-{x'}{\cdot}(0.28313$\\
                  \hspace{6em}$+{x'}{\cdot}(0.30078-0.1507{\cdot}{x'}))))))))))$
            \EndIf
                \State $s \gets \sqrt{\sin\theta} $
                \State $u \gets x{\cdot}s$
            \If{$u < 0.1$}
                \State ${y^\Delta} \gets {u}{\cdot}{u}{\cdot}(0.53333333-{u}{\cdot}(1.9753-{u}{\cdot}(5.14-11.9{\cdot}{u}))))$
            \Else
                \State $u_s \gets \sqrt{u} $
                \State $u' \gets (u_s-1)/(u_s+1) $
                \State ${y^\Delta} \gets 0.0514714+{u'}{\cdot}(0.0863117-{u'}{\cdot}(0.191581+{u'}{\cdot}(0.266342-{u'}{\cdot}(0.504516$\\
                  \hspace{6em}$+{u'}{\cdot}(0.32195-{u'}{\cdot}(0.894662-{u'}{\cdot}(0.0162501+{u'}{\cdot}(0.708855$\\
                  \hspace{6em}$-0.33707{\cdot}{u'}))))))))$
            \EndIf
            \State \textbf{return} $y_0 + \sin\theta{\cdot}s{\cdot}y^\Delta$
        \EndProcedure
    \end{algorithmic}
    \label{alg:yfct_scndlorentz}
\end{algorithm}

\begin{algorithm}
    \caption{Algorithm for calculating $y_F(x,\sin\theta)$.
             When $x\le1000$ the result has an absolute error less than $10^{-3}\cdot\min(y_\text{true},1-y_\text{true})$.
             All variables must be double precision floating point types or better, as per the \texttt{binary64} format\protect\cite{IEEE754}.}
    \begin{algorithmic}[1]
        \Procedure{YF}{$x, \sin\theta$}
            \If{$x < 0.1$}
                \State ${y_0} \gets 1-{x}{\cdot}(1-{x}{\cdot}(0.88-{x}{\cdot}(0.639-{x}{\cdot}(0.394-0.21{\cdot}{x}))))$
            \Else
                \If{$x>1000$}
                    \State \textbf{return} $\text{YF}(1000,\sin\theta)\cdot\sqrt{1000/x}$
                \EndIf
                \State $x_s \gets \sqrt{x} $
                \State $x' \gets (x_s-1)/(x_s+1) $
                \State ${y_0} \gets 0.493354-{x'}{\cdot}(0.963692-{x'}{\cdot}(0.235067+{x'}{\cdot}(1.18222-{x'}{\cdot}(0.672931$\\
                  \hspace{6em}$+{x'}{\cdot}(1.78522-{x'}{\cdot}(1.09976+{x'}{\cdot}(2.10882-{x'}{\cdot}(1.34721+{x'}{\cdot}(1.46841$\\
                  \hspace{6em}$-{x'}{\cdot}(1.0054+{x'}{\cdot}(0.426279-0.313436{\cdot}{x'})))))))))))$
            \EndIf
                \State $s \gets \sqrt{\sin\theta} $
                \State $u \gets x{\cdot}s$
            \If{$u < 0.1$}
                \State ${y^\Delta} \gets {u}{\cdot}{u}{\cdot}(0.44-{u}{\cdot}(1.2783-{u}{\cdot}(2.56-4.52{\cdot}{u}))))$
            \Else
                \State $u_s \gets \sqrt{u} $
                \State $u' \gets (u_s-1)/(u_s+1) $
                \State ${y^\Delta} \gets 0.05839+{u'}{\cdot}(0.12063-{u'}{\cdot}(0.233343+{u'}{\cdot}(0.578753-{u'}{\cdot}(0.584531$\\
                  \hspace{6em}$+{u'}{\cdot}(1.42753-{u'}{\cdot}(1.28278+{u'}{\cdot}(1.95436-{u'}{\cdot}(2.18561+{u'}{\cdot}(0.877761$\\
                  \hspace{6em}$-{u'}{\cdot}(1.80505-0.599956{\cdot}{u'}))))))))))$
            \EndIf
            \State \textbf{return} $y_0 + \sin\theta{\cdot}s{\cdot}y^\Delta$
        \EndProcedure
    \end{algorithmic}
    \label{alg:yfct_scndfresnel}
\end{algorithm}


\begin{algorithm}
    \caption{Algorithm for calculating $y_P(x,\sin\theta)$.
             When $x\le1000$ the result has an absolute error less than $10^{-6}\cdot\min(y_\text{true},1-y_\text{true})$.
             All variables must be double precision floating point types or better, as per the \texttt{binary64} format\protect\cite{IEEE754}.}
    \begin{algorithmic}[1]
        \Procedure{YPLux}{$x, \sin\theta$}
            \If{$x < 0.01$}
                \State ${y_0} \gets 1-{x}{\cdot}(0.94285714285714-{x}{\cdot}(0.8204329004329-{x}{\cdot}(0.59332136724729$\\
                  \hspace{6em}$-{x}{\cdot}(0.36445371604802-0.19428532960037{\cdot}{x}))))$
            \Else
                \If{$x>1000$}
                    \State \textbf{return} $\text{YPLux}(1000,\sin\theta)\cdot\sqrt{1000/x}$
                \EndIf
                \State $x_s \gets \sqrt{x} $
                \State $x' \gets (x_s-1)/(x_s+1) $
                \State ${y_0} \gets 0.5181739387159-{x'}{\cdot}(0.9312351765688-{x'}{\cdot}(0.1838330254976$\\
                  \hspace{6em}$+{x'}{\cdot}(1.136227750812-{x'}{\cdot}(0.6430023013107+{x'}{\cdot}(2.203730711967$\\
                  \hspace{6em}$-{x'}{\cdot}(1.106859163687+{x'}{\cdot}(5.250378502198-{x'}{\cdot}(1.160733102899$\\
                  \hspace{6em}$+{x'}{\cdot}(13.53773020006-{x'}{\cdot}(2.890627780353+{x'}{\cdot}(35.57482018146$\\
                  \hspace{6em}$-{x'}{\cdot}(45.42477742421+{x'}{\cdot}(116.3992429074-{x'}{\cdot}(344.1765370083$\\
                  \hspace{6em}$+{x'}{\cdot}(493.5436892409-{x'}{\cdot}(1529.908846124+{x'}{\cdot}(1939.034798807$\\
                  \hspace{6em}$-{x'}{\cdot}(4545.962961959+{x'}{\cdot}(5797.769026808-{x'}{\cdot}(9488.71109524$\\
                  \hspace{6em}$+{x'}{\cdot}(12589.02849097-{x'}{\cdot}(14131.24414393+{x'}{\cdot}(19655.56515557$\\
                  \hspace{6em}$-{x'}{\cdot}(14950.68734465+{x'}{\cdot}(21827.36286125-{x'}{\cdot}(10979.64949738$\\
                  \hspace{6em}$+{x'}{\cdot}(16822.09661562-{x'}{\cdot}(5321.34227364+{x'}{\cdot}(8554.750838143$\\
                  \hspace{6em}$-{x'}{\cdot}(1530.481305745+{x'}{\cdot}(2582.400834841-{x'}{\cdot}(197.8358674336$\\
                  \hspace{6em}$+350.5878203564{\cdot}{x'}))))))))))))))))))))))))))))))))$
            \EndIf
                \State $s \gets \sqrt{\sin\theta} $
                \State $u \gets x{\cdot}s$
            \If{$u < 0.01$}
                \State ${y^\Delta} \gets {u}{\cdot}{u}{\cdot}(0.41021645021645-{u}{\cdot}(1.1866427344946-{u}{\cdot}(2.3689491543122$\\
                  \hspace{6em}$-4.1771345864079{\cdot}{u}))))$
            \Else
                \State $u_s \gets \sqrt{u} $
                \State $u' \gets (u_s-1)/(u_s+1) $
                \State ${y^\Delta} \gets 0.05508579937334+{u'}{\cdot}(0.1171544212659-{u'}{\cdot}(0.2094910547035$\\
                  \hspace{6em}$+{u'}{\cdot}(0.5753579035796-{u'}{\cdot}(0.5024400915276+{u'}{\cdot}(1.826900092486$\\
                  \hspace{6em}$-{u'}{\cdot}(0.9094508110575+{u'}{\cdot}(5.534641711093-{u'}{\cdot}(1.08260571543$\\
                  \hspace{6em}$+{u'}{\cdot}(20.56364020803-{u'}{\cdot}(4.869343230026+{u'}{\cdot}(102.9563469135$\\
                  \hspace{6em}$-{u'}{\cdot}(68.00002474973+{u'}{\cdot}(556.4765377499-{u'}{\cdot}(513.5266226562$\\
                  \hspace{6em}$+{u'}{\cdot}(2535.434739068-{u'}{\cdot}(2439.29543339+{u'}{\cdot}(8810.590045429$\\
                  \hspace{6em}$-{u'}{\cdot}(8079.00393415+{u'}{\cdot}(22716.17422582-{u'}{\cdot}(19467.21452423$\\
                  \hspace{6em}$+{u'}{\cdot}(42968.7118192-{u'}{\cdot}(34589.4608283+{u'}{\cdot}(58751.62948418$\\
                  \hspace{6em}$-{u'}{\cdot}(45057.24103498+{u'}{\cdot}(56474.65177073-{u'}{\cdot}(41955.93650756$\\
                  \hspace{6em}$+{u'}{\cdot}(36199.30270958-{u'}{\cdot}(26472.59066196+{u'}{\cdot}(13895.22702324$\\
                  \hspace{6em}$-{u'}{\cdot}(10141.18548965+{u'}{\cdot}(2417.098215649$\\
                  \hspace{6em}$-1779.660818125{\cdot}{u'})))))))))))))))))))))))))))))))$
            \EndIf
            \State \textbf{return} $y_0 + \sin\theta{\cdot}s{\cdot}y^\Delta$
        \EndProcedure
    \end{algorithmic}
    \label{alg:yfct_primary_lux}
\end{algorithm}

\begin{algorithm}
    \caption{Algorithm for calculating $y_G(x,\sin\theta)$.
             When $x\le1000$ the result has an absolute error less than $10^{-6}\cdot\min(y_\text{true},1-y_\text{true})$.
             All variables must be double precision floating point types or better, as per the \texttt{binary64} format\protect\cite{IEEE754}.}
    \begin{algorithmic}[1]
        \Procedure{YGLux}{$x, \sin\theta$}
            \If{$x < 0.01$}
                \State ${y_0} \gets 1-{x}{\cdot}(1.0606601717798-{x}{\cdot}(0.9237604307034-{x}{\cdot}(0.66666666666667$\\
                  \hspace{6em}$-{x}{\cdot}(0.40888100159996-0.21773242158073{\cdot}{x}))))$
            \Else
                \If{$x>1000$}
                    \State \textbf{return} $\text{YGLux}(1000,\sin\theta)\cdot(0.001{\cdot}x)^{-0.933}$
                \EndIf
                \State $x_s \gets \sqrt{x} $
                \State $x' \gets (x_s-1)/(x_s+1) $
                \State ${y_0} \gets 0.458984305748-{x'}{\cdot}(1.03988766848-{x'}{\cdot}(0.233481493625$\\
                  \hspace{6em}$+{x'}{\cdot}(1.33196188935-{x'}{\cdot}(0.660013233947+{x'}{\cdot}(2.4541064046$\\
                  \hspace{6em}$-{x'}{\cdot}(1.19382855336+{x'}{\cdot}(5.95477588365-{x'}{\cdot}(0.515887633752$\\
                  \hspace{6em}$+{x'}{\cdot}(15.5539313381+{x'}{\cdot}(7.06968625952-{x'}{\cdot}(37.2332878318$\\
                  \hspace{6em}$+{x'}{\cdot}(35.5989115244-{x'}{\cdot}(74.5751517966+{x'}{\cdot}(102.844087729$\\
                  \hspace{6em}$-{x'}{\cdot}(119.034265439+{x'}{\cdot}(206.588594755-{x'}{\cdot}(146.681948708$\\
                  \hspace{6em}$+{x'}{\cdot}(299.813474624-{x'}{\cdot}(135.504487803+{x'}{\cdot}(314.167337841$\\
                  \hspace{6em}$-{x'}{\cdot}(90.3876932037+{x'}{\cdot}(232.115104836-{x'}{\cdot}(41.0324379496$\\
                  \hspace{6em}$+{x'}{\cdot}(114.760752113-{x'}{\cdot}(11.3379678407+{x'}{\cdot}(34.0792787331$\\
                  \hspace{6em}$-{x'}{\cdot}(1.43947016407+4.59564246328{\cdot}{x'})))))))))))))))))))))))))))$
            \EndIf
                \State $s \gets \sqrt{\sin\theta} $
                \State $u \gets x{\cdot}s$
            \If{$u < 0.01$}
                \State ${y^\Delta} \gets {u}{\cdot}{u}{\cdot}(0.4618802153517-{u}{\cdot}(1.3333333333333-{u}{\cdot}(2.6577265103997$\\
                  \hspace{6em}$-4.6812470639856{\cdot}{u}))))$
            \Else
                \State $u_s \gets \sqrt{u} $
                \State $u' \gets (u_s-1)/(u_s+1) $
                \State ${y^\Delta} \gets 0.06224945677291+{u'}{\cdot}(0.1328148715703-{u'}{\cdot}(0.2367191183053$\\
                  \hspace{6em}$+{u'}{\cdot}(0.6588518159271-{u'}{\cdot}(0.5406805120311+{u'}{\cdot}(2.032630120584$\\
                  \hspace{6em}$-{u'}{\cdot}(0.9393953652666+{u'}{\cdot}(5.784078539319-{u'}{\cdot}(0.02368756317468$\\
                  \hspace{6em}$+{u'}{\cdot}(15.97012500428+{u'}{\cdot}(8.977966879657-{u'}{\cdot}(40.36034869005$\\
                  \hspace{6em}$+{u'}{\cdot}(43.96162777448-{u'}{\cdot}(88.73413073017+{u'}{\cdot}(130.0724012716$\\
                  \hspace{6em}$-{u'}{\cdot}(164.7896006596+{u'}{\cdot}(267.8997764342-{u'}{\cdot}(252.6501017196$\\
                  \hspace{6em}$+{u'}{\cdot}(392.6403040775-{u'}{\cdot}(309.779282021+{u'}{\cdot}(402.7413298521$\\
                  \hspace{6em}$-{u'}{\cdot}(289.0035973536+{u'}{\cdot}(275.176835677-{u'}{\cdot}(190.1231677624$\\
                  \hspace{6em}$+{u'}{\cdot}(112.5635155648-{u'}{\cdot}(77.61959840503+{u'}{\cdot}(20.85731684605$\\
                  \hspace{6em}$-14.65676322637{\cdot}{u'}))))))))))))))))))))))))))$
            \EndIf
            \State \textbf{return} $y_0 + \sin\theta{\cdot}s{\cdot}y^\Delta$
        \EndProcedure
    \end{algorithmic}
    \label{alg:yfct_scndgauss_lux}
\end{algorithm}

\begin{algorithm}
    \caption{Algorithm for calculating $y_L(x,\sin\theta)$.
             When $x\le1000$ the result has an absolute error less than $10^{-6}\cdot\min(y_\text{true},1-y_\text{true})$.
             All variables must be double precision floating point types or better, as per the \texttt{binary64} format\protect\cite{IEEE754}.}
    \begin{algorithmic}[1]
        \Procedure{YLLux}{$x, \sin\theta$}
            \If{$x < 0.01$}
                \State ${y_0} \gets 1-{x}{\cdot}(1-{x}{\cdot}(1.0666666666667-{x}{\cdot}(0.98765432098765$\\
                  \hspace{6em}$-{x}{\cdot}(0.79012345679012-0.55308641975309{\cdot}{x}))))$
            \Else
                \If{$x>1000$}
                    \State \textbf{return} $\text{YLLux}(1000,\sin\theta)\cdot\sqrt{1000/x}$
                \EndIf
                \State $x_s \gets \sqrt{x} $
                \State $x' \gets (x_s-1)/(x_s+1) $
                \State ${y_0} \gets 0.5336440714-{x'}{\cdot}(0.8429109763-{x'}{\cdot}(0.1770158718+{x'}{\cdot}(0.6797206737$\\
                  \hspace{6em}$-{x'}{\cdot}(0.8196250789+{x'}{\cdot}(0.7065849507-{x'}{\cdot}(2.167539611+{x'}{\cdot}(0.947538487$\\
                  \hspace{6em}$-{x'}{\cdot}(5.583355582+{x'}{\cdot}(2.114837452-{x'}{\cdot}(13.85717444+{x'}{\cdot}(6.005361128$\\
                  \hspace{6em}$-{x'}{\cdot}(30.50702161+{x'}{\cdot}(14.66333827-{x'}{\cdot}(55.45431274+{x'}{\cdot}(26.79341786$\\
                  \hspace{6em}$-{x'}{\cdot}(79.16353634+{x'}{\cdot}(35.10022132-{x'}{\cdot}(85.27396522+{x'}{\cdot}(31.88897056$\\
                  \hspace{6em}$-{x'}{\cdot}(66.36505854+{x'}{\cdot}(19.04296095-{x'}{\cdot}(35.03907293+{x'}{\cdot}(6.713303014$\\
                  \hspace{6em}$-{x'}{\cdot}(11.1928655+{x'}{\cdot}(1.057457802$\\
                  \hspace{6em}$-1.628737767{\cdot}{x'})))))))))))))))))))))))))$
            \EndIf
                \State $s \gets \sqrt{\sin\theta} $
                \State $u \gets x{\cdot}s$
            \If{$u < 0.01$}
                \State ${y^\Delta} \gets {u}{\cdot}{u}{\cdot}(0.53333333333333-{u}{\cdot}(1.9753086419753-{u}{\cdot}(5.1358024691358$\\
                  \hspace{6em}$-11.891358024691{\cdot}{u}))))$
            \Else
                \State $u_s \gets \sqrt{u} $
                \State $u' \gets (u_s-1)/(u_s+1) $
                \State ${y^\Delta} \gets 0.05157979572+{u'}{\cdot}(0.08702095023-{u'}{\cdot}(0.1993447288$\\
                  \hspace{6em}$+{u'}{\cdot}(0.289977587-{u'}{\cdot}(0.6373281731+{u'}{\cdot}(0.5519041851-{u'}{\cdot}(1.94650911$\\
                  \hspace{6em}$+{u'}{\cdot}(0.9907616782-{u'}{\cdot}(5.428801081+{u'}{\cdot}(2.318934191-{u'}{\cdot}(13.68502225$\\
                  \hspace{6em}$+{u'}{\cdot}(6.398706822-{u'}{\cdot}(29.60860216+{u'}{\cdot}(15.11962229-{u'}{\cdot}(51.70467978$\\
                  \hspace{6em}$+{u'}{\cdot}(26.27261166-{u'}{\cdot}(69.10955851+{u'}{\cdot}(31.57234238-{u'}{\cdot}(67.19987531$\\
                  \hspace{6em}$+{u'}{\cdot}(24.75321921-{u'}{\cdot}(44.50455466+{u'}{\cdot}(11.38071533-{u'}{\cdot}(17.89269095$\\
                  \hspace{6em}$+{u'}{\cdot}(2.326886585-3.289352571{\cdot}{u'})))))))))))))))))))))))$
            \EndIf
            \State \textbf{return} $y_0 + \sin\theta{\cdot}s{\cdot}y^\Delta$
        \EndProcedure
    \end{algorithmic}
    \label{alg:yfct_scndlorentz_lux}
\end{algorithm}

\begin{algorithm}
    \caption{Algorithm for calculating $y_F(x,\sin\theta)$.
             When $x\le1000$ the result has an absolute error less than $10^{-6}\cdot\min(y_\text{true},1-y_\text{true})$.
             All variables must be double precision floating point types or better, as per the \texttt{binary64} format\protect\cite{IEEE754}.}
    \begin{algorithmic}[1]
        \Procedure{YFLux}{$x, \sin\theta$}
            \If{$x < 0.01$}
                \State ${y_0} \gets 1-{x}{\cdot}(1-{x}{\cdot}(0.88-{x}{\cdot}(0.63915343915344-{x}{\cdot}(0.39352481733434$\\
                  \hspace{6em}$-0.2100936347603{\cdot}{x}))))$
            \Else
                \If{$x>1000$}
                    \State \textbf{return} $\text{YFLux}(1000,\sin\theta)\cdot\sqrt{1000/x}$
                \EndIf
                \State $x_s \gets \sqrt{x} $
                \State $x' \gets (x_s-1)/(x_s+1) $
                \State ${y_0} \gets 0.493269511538-{x'}{\cdot}(0.9650579561441-{x'}{\cdot}(0.2391574631397$\\
                  \hspace{6em}$+{x'}{\cdot}(1.231743529955-{x'}{\cdot}(0.7150998765463+{x'}{\cdot}(2.39512039449$\\
                  \hspace{6em}$-{x'}{\cdot}(1.242880751013+{x'}{\cdot}(6.121219917985-{x'}{\cdot}(1.290509624725$\\
                  \hspace{6em}$+{x'}{\cdot}(20.80033791369-{x'}{\cdot}(0.8507771998392+{x'}{\cdot}(92.48547880716$\\
                  \hspace{6em}$-{x'}{\cdot}(19.77420147711+{x'}{\cdot}(451.7158702461-{x'}{\cdot}(164.3222758496$\\
                  \hspace{6em}$+{x'}{\cdot}(1930.941124517-{x'}{\cdot}(694.4535319857+{x'}{\cdot}(6457.672947679$\\
                  \hspace{6em}$-{x'}{\cdot}(1848.071491788+{x'}{\cdot}(16317.98164344-{x'}{\cdot}(3295.667628162$\\
                  \hspace{6em}$+{x'}{\cdot}(30806.79618825-{x'}{\cdot}(3959.321950465+{x'}{\cdot}(43047.70157153$\\
                  \hspace{6em}$-{x'}{\cdot}(3083.476664459+{x'}{\cdot}(43814.62484922-{x'}{\cdot}(1378.499617051$\\
                  \hspace{6em}$+{x'}{\cdot}(31541.33078748-{x'}{\cdot}(200.305245992+{x'}{\cdot}(15210.8385604$\\
                  \hspace{6em}$+{x'}{\cdot}(89.06000453532-{x'}{\cdot}(4407.642969232+{x'}{\cdot}(32.20146603275$\\
                  \hspace{6em}$-580.1276064014{\cdot}{x'}))))))))))))))))))))))))))))))))$
            \EndIf
                \State $s \gets \sqrt{\sin\theta} $
                \State $u \gets x{\cdot}s$
            \If{$u < 0.01$}
                \State ${y^\Delta} \gets {u}{\cdot}{u}{\cdot}(0.44-{u}{\cdot}(1.2783068783069-{u}{\cdot}(2.5579113126732$\\
                  \hspace{6em}$-4.5170131473465{\cdot}{u}))))$
            \Else
                \State $u_s \gets \sqrt{u} $
                \State $u' \gets (u_s-1)/(u_s+1) $
                \State ${y^\Delta} \gets 0.0583967885359+{u'}{\cdot}(0.121536980788-{u'}{\cdot}(0.232718121245$\\
                  \hspace{6em}$+{u'}{\cdot}(0.616387292797-{u'}{\cdot}(0.55890143011+{u'}{\cdot}(1.94001237751$\\
                  \hspace{6em}$-{u'}{\cdot}(1.01198622536+{u'}{\cdot}(5.2138639488-{u'}{\cdot}(0.952289303785$\\
                  \hspace{6em}$+{u'}{\cdot}(10.7857185707-{u'}{\cdot}(0.253236080473+{u'}{\cdot}(1.44547180975$\\
                  \hspace{6em}$-{u'}{\cdot}(19.5394751459-{u'}{\cdot}(119.087396102+{u'}{\cdot}(172.169663581$\\
                  \hspace{6em}$-{u'}{\cdot}(622.465240494+{u'}{\cdot}(738.504693661-{u'}{\cdot}(1891.82897847$\\
                  \hspace{6em}$+{u'}{\cdot}(1970.26994857-{u'}{\cdot}(3911.23420125+{u'}{\cdot}(3488.2042579$\\
                  \hspace{6em}$-{u'}{\cdot}(5674.37556774+{u'}{\cdot}(4118.52337359-{u'}{\cdot}(5722.603149$\\
                  \hspace{6em}$+{u'}{\cdot}(3126.83375841-{u'}{\cdot}(3839.36372179+{u'}{\cdot}(1383.96533161$\\
                  \hspace{6em}$-{u'}{\cdot}(1545.80746193+{u'}{\cdot}(271.909527265$\\
                  \hspace{6em}$-283.161354174{\cdot}{u'}))))))))))))))))))))))))))))$
            \EndIf
            \State \textbf{return} $y_0 + \sin\theta{\cdot}s{\cdot}y^\Delta$
        \EndProcedure
    \end{algorithmic}
    \label{alg:yfct_scndfresnel_lux}
\end{algorithm}



\section{Conclusions}
show cmprecipes plots

\begin{figure}[htp]
    \centering
    \begin{subfigure}[t]{0.49\textwidth}
        \centering
        \includegraphics[width=\textwidth]{generated/cmprecipes_primary.pdf}
        \caption{$M=P$}\labfig{cmprecipes:subfig:P}
    \end{subfigure}
    \hfill
    \begin{subfigure}[t]{0.49\textwidth}
        \centering
        \includegraphics[width=\textwidth]{generated/cmprecipes_scndgauss.pdf}
        \caption{$M=G$}\labfig{cmprecipes:subfig:G}
    \end{subfigure}
    \vspace{0.5cm}
    \begin{subfigure}[t]{0.49\textwidth}
        \centering
        \includegraphics[width=\textwidth]{generated/cmprecipes_scndlorentz.pdf}
        \caption{$M=L$}\labfig{cmprecipes:subfig:L}
    \end{subfigure}
    \hfill
    \begin{subfigure}[t]{0.49\textwidth}
        \centering
        \includegraphics[width=\textwidth]{generated/cmprecipes_scndfresnel.pdf}
        \caption{$M=F$}\labfig{cmprecipes:subfig:F}
    \end{subfigure}
    \caption{Bla bla cmprecipes std. Note: breakdown near 45degrees in $M=L$ plot (we have split
      that line into $\pm45^\circ$ to show it) }
    \labfig{cmprecipes}
\end{figure}

%%\begin{figure}[htp]
%%    \centering
%%    \begin{subfigure}[t]{0.49\textwidth}
%%        \centering
%%        \includegraphics[width=\textwidth]{generated/cmprecipes_primary_lux.pdf}
%%        \caption{$M=P$}\labfig{cmprecipeslux:subfig:P}
%%    \end{subfigure}
%%    \hfill
%%    \begin{subfigure}[t]{0.49\textwidth}
%%        \centering
%%        \includegraphics[width=\textwidth]{generated/cmprecipes_scndgauss_lux.pdf}
%%        \caption{$M=G$}\labfig{cmprecipeslux:subfig:G}
%%    \end{subfigure}
%%    \vspace{0.5cm}
%%    \begin{subfigure}[t]{0.49\textwidth}
%%        \centering
%%        \includegraphics[width=\textwidth]{generated/cmprecipes_scndlorentz_lux.pdf}
%%        \caption{$M=L$}\labfig{cmprecipeslux:subfig:L}
%%    \end{subfigure}
%%    \hfill
%%    \begin{subfigure}[t]{0.49\textwidth}
%%        \centering
%%        \includegraphics[width=\textwidth]{generated/cmprecipes_scndfresnel_lux.pdf}
%%        \caption{$M=F$}\labfig{cmprecipeslux:subfig:F}
%%    \end{subfigure}
%%    \caption{Bla bla cmprecipes lux. Note: breakdown near 45degrees in $M=L$ plot (we have split
%%      that line into $\pm45^\circ$ to show it) }
%%    \labfig{cmprecipeslux}
%%\end{figure}


%%\begin{figure}[ht] %
%%\begin{center}
%%\includegraphics[width=0.6\textwidth]{generated/cmprecipes_primary.pdf}
%%\end{center}
%%\caption{cmprecipes primary.}
%%\labfig{cmprecipes:primary}
%%\end{figure}
%%
%%\begin{figure}[ht] %
%%\begin{center}
%%\includegraphics[width=0.6\textwidth]{generated/cmprecipes_primary_lux.pdf}
%%\end{center}
%%\caption{cmprecipes primary lux.}
%%\labfig{cmprecipes:primary:lux}
%%\end{figure}
%%
%%\begin{figure}[ht] %
%%\begin{center}
%%\includegraphics[width=0.6\textwidth]{generated/cmprecipes_scndgauss.pdf}
%%\end{center}
%%\caption{cmprecipes scndgauss.}
%%\labfig{cmprecipes:scndgauss}%fixme
%%\end{figure}
%%
%%\begin{figure}[ht] %
%%\begin{center}
%%\includegraphics[width=0.6\textwidth]{generated/cmprecipes_scndgauss_lux.pdf}
%%\end{center}
%%\caption{cmprecipes scndgauss lux.}
%%\labfig{cmprecipes:scndgauss:lux}%fixme
%%\end{figure}
%%
%%\begin{figure}[ht] %
%%\begin{center}
%%\includegraphics[width=0.6\textwidth]{generated/cmprecipes_scndlorentz.pdf}
%%\end{center}
%%\caption{cmprecipes scndlorentz. NOTE BREAKDOWN NEAR 45degrees (we have split
%%  that line into $\pm45^\circ$ to show it).}
%%\labfig{cmprecipes:scndlorentz}%fixme
%%\end{figure}
%%
%%\begin{figure}[ht] %
%%\begin{center}
%%\includegraphics[width=0.6\textwidth]{generated/cmprecipes_scndlorentz_lux.pdf}
%%\end{center}
%%\caption{cmprecipes scndlorentz lux. NOTE BREAKDOWN NEAR 45degrees (we have split
%%  that line into $\pm45^\circ$ to show it).}
%%\labfig{cmprecipes:scndlorentz:lux}%fixme
%%\end{figure}
%%
%%\begin{figure}[ht] %
%%\begin{center}
%%\includegraphics[width=0.6\textwidth]{generated/cmprecipes_scndfresnel.pdf}
%%\end{center}
%%\caption{cmprecipes scndfresnel.}
%%\labfig{cmprecipes:scndfresnel}%fixme
%%\end{figure}
%%
%%\begin{figure}[ht] %
%%\begin{center}
%%\includegraphics[width=0.6\textwidth]{generated/cmprecipes_scndfresnel_lux.pdf}
%%\end{center}
%%\caption{cmprecipes scndfresnel lux.}
%%\labfig{cmprecipes:scndfresnel:lux}%fixme
%%\end{figure}

%%\begin{align}
%%  \labeqn{fofetaL}
%%\end{align}
%%\begin{align}
%%  % sinc(x) = sin(x)/x
%%  % eq. 44, sigma(0) = Q*alpha
%%  %
%%  % eta = pi * eps1 * beta = (4/3)*pi*eps1*alpha, so pi*eps1*alpha = 3eta/4
%%  %
%%  % f(eta) = sigma(eps1(eta))/sigma(0)
%%  %        = sinc^2( pi*eps1*alpha )
%%  %        = sinc^2( 3eta/4 )
%%  \labeqn{fofetaF}
%%  \labeqn{fofeta}
%%
%%\end{align}
%%
%
%capture the width of the Bragg  the shape of the Bragg diffraction cross
%section as a function of deviation from the
%
%
% of the intrinsic angular response...
% ``rocking curves'' intrinsic
%to each
%
%The $f_M(\eta)$ functions capture the shape of the intrinsic angular response...
% ``rocking curves'' intrinsic
%to each
%
%
%
%
%We should probably plot phi0 and phipi? (and perhaps phi at some intermediate
%values of theta as well)? And also plot the fM functions.
%
%interpolation between the solutions as:.. note phipi > phi0,
%strictly true that phi is monotonically increasing with theta, and since
%f,C,N>0, so is y accordingly. Can we also see from the equation that y must drop
%monotonically with x? And that y=1 at x=0 and y=0 at x=inf?
%
%primary eps1: ``divergence of incident beam''.
%
%
%NB: $x$ is also defined differently for the different models, refer to paper,
%but for convenience we will refer to it simply as $x$ with no subscript.
%
%In the
%work by Becker-Coppens \cite{BC1974}, four specific models and recipes are
%provided for the determination of $y$: one for primary extinction, and three for
%secondary extinction, differing only in the choice of distribution used to
%describe the mosaic spread of coherent domains within the crystalline materials.
%





%$
%$
%$
%$
%$
%$\begin{itemize}
%$\item[Bla] bhejsa
%$\item[Bla2] bhejsa
%$\end{itemize}
%$
%$In
%$Becker-Coppens
%$
%$, which always fulfils ,
%$
%$The extinction factor $y$ will always be
%$observed
%$
%$Iobs = y*Ikin
%$
%$Introduce y, x, theta, and the 4 models.
%$
%
%$
%$
%$For instance, assume a particle fulfils the condition for
%$scattering on a reflection plane with d-spacing $d$ and Bragg angle $\theta$
%$({\em}Bragg's Law):
%$\begin{align}
%$  \lambda=2d\sin\theta
%$\end{align}
%$If the particle undergoes a scattering event, Braggs law will still be fulfilled
%$for the outgoing particle as well, and it is possible that a second scattering.
%$
%$
%$  for scattering on a plane with d-spacing $d$, will
%$and undergoing a Bragg diffraction scattering, will still fulfil the Bragg
%$condition for scattering
%$
%$ , will always have a non-zero
%$
%$
%$Iobs = y*Ikin
%$
%$important point, BC in their paper explicitly mentions and deals with 2theta=pi,
%$so they are NOT working on the assumption of small theta only. However, pg 21
%$bottom of first column, they discuss that they do assume small extinction only
%$at 2theta=pi, so perhaps we should not be surprised in case of issues.
%$
%$
%$but we should:
%$%1/sin2th = 1/(2*sinth*costh) = (d/wl) / sqrt(1-(wl/2d)^2) = 1/(2*s*sqrt(1-s^2))
%
%$sinth = wl/2d
%$
%$multiple scattering
%$
%$Extinction  is the phenomen... Primary extinction is...
%$
%$NB: $\theta$ is the Bragg angle, defined by:
%$\begin{align}
%$  \sin\theta = \frac{\lambda}{2d_{hkl}}
%$  \labeqn{braggequation}
%$\end{align}
%$
%$And $x$ is defined by (fixme: here $l$ is actually tbar which is 3/2 times
%$sphere radius):
%$\begin{align}
%$  x = \frac{ 2l^2\lambda^2F^2}{3V^2}
%$  \labeqn{xdef}
%$\end{align}
%$
%$Introduction, mention how bragg diffraction and how $y_p(\theta,d_{hkl},F^2)$
%$will depend on the geometrical layout of the crystallites. Mention Bragg angle,
%$and x-value, and show the original recipe, and how it is used with $l$, $x$,
%$$F^2$, $\theta$, $d_{hkl}$.
%$
%Also show the break-down!




%\begin{figure}[ht] %
%\label{fig:figure1}
%\begin{center}
%\includegraphics[width=0.6\textwidth]{fig1.png} % NB use pdflatex for non-postscript
%\end{center}
%\caption{Caption \protect\cite{knuth84}} % NB \protect\cite{...} is required in floating figures
%\end{figure}






%\section{New recipes}

% Basic table

%\begin{table}[ht]
%\caption{Caption to table \protect\cite{lamport86}} % NB \protect\cite{...} is required in floating tables
%\smallskip
%\begin{center}
%\begin{tabular}{llcr}
%\midrule
% HEADING    & FOR        & EACH       & COLUMN     \\
%\midrule
% entry      & entry      & entry      & entry      \\
% entry      & entry      & entry      & entry      \\
% entry      & entry      & entry      & entry      \\
%\end{tabular}
%\end{center}
%\end{table}
%


\section{STUFF FOR APPENDIX OR MAIN PART?}
\subsection{Tail integration}

For positive $\eta$ values we have:

\begin{align}
  f(\eta) \le f^+(\eta) \equiv \frac{\eta^2+\eta+1}{\eta^4}
  \labeqn{fplus}
\end{align}
\begin{align}
  f(\eta) \ge f^-(\eta) \equiv \frac{\eta^2-\eta}{\eta^4}
  \labeqn{fminus}
\end{align}

Where $f^-(\eta)$ is only strictly positive when $\eta>1$, so we will restrict
ourselves to $\eta>2$ (??)

Since $\varphi(\theta,s)$ and $f(\eta)$ decreases monotonically for increasing
$s$ and $\eta$ respectively for all positive arguments, we can conclude that:

\begin{align}
\int_{a}^{\infty} f(\eta)\varphi\left(\theta,xf(\eta)\right)d\eta \le
\int_{a}^{\infty} f^+(\eta)\varphi\left(\theta,xf^-(\eta)\right)d\eta \equiv T^+(\theta,x,a) \nonumber\\
\int_{a}^{\infty} f(\eta)\varphi\left(\theta,xf(\eta)\right)d\eta \ge
\int_{a}^{\infty} f^-(\eta)\varphi\left(\theta,xf^+(\eta)\right)d\eta \equiv T^-(\theta,x,a)
\end{align}

Using $k=x/a$ and ensuring $a>\max(x,2)$ we have $k<1$.

With the help of the SageMath (REF) Computer Algebra System, $T^\pm(\theta,x,a)$
can be evaluated analytically and expressed as a Taylor expansion in inverse
powers of $a$:

\begin{align}
  T^+(\theta,x,a) = \frac{1}{a}  + \frac{1-k}{2a^2 }
  +\left[\frac{1}{3} + \frac{8+4\sin^{5/2}\theta}{25}k^2  \right]\frac{1}{a^3}
  +\mathcal{O}\left(\frac{1}{a^4}\right)
\end{align}

\begin{align}
  T^-(\theta,x,a) = \frac{1}{a}  - \frac{1+k}{2a^2 }
  +\frac{8+4\sin^{5/2}\theta}{25}k^2\frac{1}{a^3}
  +\mathcal{O}\left(\frac{1}{a^4}\right)
\end{align}

Varying both $k$ and $\sin\theta$ over values the unit interval and evaluating
Taylor coefficients for terms up to $1/a^{20}$ numerically, it is seen that the
maximal absolute numerical value of coefficients is always within a factor of
$2.50$ when compared to the previous order. Thus, if we restrict ourselves to
$a>10$, we can conservatively limit the maximal contributions of the terms of
order $1/a^4$ or higher in REFEQNABOVE to be at most $10/a^4$. With that, we can
use $(T^+(\theta,x,a)+T^-(\theta,x,a))/2$ as an estimate of the tail integral
value, while $(T^+(\theta,x,a)-T^-(\theta,x,a))/2$ will be a conservative upper
bound of the error on the estimate. In other words, for $a>\max(10,x)$ we get:

\begin{align}
\int_{a}^{\infty} f(\eta)\varphi\left(\theta,xf(\eta)\right)d\eta \approx
\frac{1}{a}  - \frac{k}{2a^2 }
  +\left[\frac{1}{6} + \frac{8+4\sin^{5/2}\theta}{25}k^2  \right]\frac{1}{a^3}
  \pm
\left(\frac{1}{2a^2 }+\frac{1}{6a^3}+\frac{10}{a^4}\right)
\labeqn{tailintegralpenultimate}
\end{align}
Where the term in the final parenthesis represents an upper bound on the
error. Due to the leading factor of $1/a^2$ in the error term, there is not much
value in keeping the higher order terms of \refeqn{tailintegralpenultimate}. Thus, by
using $k=x/a$ and $a>10$ and a slight additional pessimisation of the error bound, we
can arrive at the following simple formula for the estimation of the tail of the
integral:
\begin{align}
\int_{a}^{\infty} f(\eta)\varphi\left(\theta,xf(\eta)\right)d\eta \approx
\frac{1}{a}  - \frac{x}{2a^3 } \pm\left(\frac{1}{2a^2 }+\frac{2}{a^3}\right),\quad{}a>10,\,a>x>0
\labeqn{tailintegral}
\end{align}

\subsection{Central integration}

With the integrand implemented via mpmath (REF), and a
Gauss-Legendre numerical integration, we integrate a particular integral
$[0,a]$. However, for reliable performance it is crucial that ...

as the integrand contains trigonometic functions in REFFOFETA,
it is important that each

\subsection{Direct evaluation at low $x$ via Taylor expansions}
bla
\subsection{Integration strategy}



At a particular $(\theta,x)$ point, the strategy to evaluate $y_p(\theta,x)$
within a target precision of $\epsilon=10^{-7}$ is described in the following.

First, for $x<1$, it is investigated if the direct Taylor expansion method
described in REFSECTION is suitable to achieve the required precision. If it is
not, the integral is solved by a combination of high-precision Gauss-Legendre
numerical integration over the interval $[0,a]$ while the contribution over the
interval $[a,\infty]$ is estimated by \refeqn{tailintegral}. The only problem
that remains is which value of $a$ to choose in this scenario.


strategy: $\delta{}a$ is $\max(100,x)$, rounded up to
For increasing values of i, set a=step*i and evaluate via a Gauss-Legendre
integration the integral over $[0,a]$. Check if combining this result with a
tail integration over $[a,\infty]$ would provide a final result with suitable
low error. If not, increase $a$ by $\Delta{}a$

\section{LEFTOVERS}


 of a particularly challenging numerical
integral....  Furthermore, the numerical evaluation in BC1974 is generally most
precise and useful in regions of intermediate extinction levels, where arguably
most data analysis ... a feature of
numerical precision and domain of validity of the recipes as presented in the
... phenomenologically, the observed Bragg diffraction cross section will thus be reduced by
a factor of $y$

... Although these models describe different
underlying physics, they are in practice mathematically similar and each require
the evaluation of a particular non-trivial integral

TODO: Mention BC highly cited, used for structure refinement, etc.

TODO: Mention y is monotonically decreasing function of x, and increases for
higher thetabragg. Mention limits y=1 at x=0, y=0 and x=inf. This is also
important when discussing the precision definition to use.

NEXT: One paragraph with a few details of the BC models, spherical particles,
primary+secondary, for different secondary mosaic distributions, etc. Then start
introducing equations for phi/phi0/phipi/f/y eq36/... and then introduce the
four models PGLF. Then mention the issues, show the NCrystal plot and the 2d
precision plots.


NB: $\theta$ is the Bragg angle, defined by:
\begin{align}
  \sin\theta = \frac{\lambda}{2d_{hkl}}
  \labeqn{braggequationREPEATEDFIXME}
\end{align}

And $x$ is defined by (fixme: is $l$ the sphere radius or tbar..., also this is
for primary):

\begin{align}
  x = \frac{ 2l^2\lambda^2F^2}{3V^2}
  \labeqn{xdefREPEATEDFIXME}
\end{align}

$N_P=\frac{3}{4\pi}$, $C_m=1$.
(FIXME: Don't focus first om primary, just do them all from the beginning,
meaning that the previous equation has to have index $M$ and two constants
depending on $M$ (the normalisation factors and the factor in front of x, the
only real difference is for the Lorentzian model I think).

hest


\subsection{New recipes}

Minimal update, simply updating the four parameters from the old model:

\begin{align}
  A(\theta) &= 0.559+0.537\cos(2\theta)\nonumber\\
  B(\theta) &= 0.604-0.222(0.5-\cos(2\theta))^2
\labeqn{ypfitfctparsupdated}
\end{align}

Improved and slightly more complicated recipe for more precise adherence to eq. 36:

\begin{align}
  y_{p}(\theta,x)=\left\{1+2x+\frac{A(\theta)x^2-0.1x}{1+B(\theta)x+C(\theta)x(1+x)^{-1} }\right\}^{-1/2}
  \labeqn{ypfitfctlux}
\end{align}

\begin{align}
  A(\theta) &= 0.8968-0.2928\sin(\theta)+2.528\sin^2(\theta)-11.21\sin^3(\theta)\nonumber\\
            &\hspace*{1em} +21.07\sin^4(\theta)-20.17\sin^5(\theta)+7.32\sin^6(\theta)\nonumber\\
  B(\theta) &= 0.4697-0.1688\sin(\theta)+5.37\sin^2(\theta)-20.68\sin^3(\theta)\nonumber\\
            &\hspace*{1em} +40.52\sin^4(\theta)-38.09\sin^5(\theta)+12.89\sin^6(\theta)\nonumber\\
  C(\theta) &= -0.7486-0.5671\sin(\theta)-1.248\sin^2(\theta)+17.56\sin^3(\theta)\nonumber\\
            &\hspace*{1em} -47.8\sin^4(\theta)+59.14\sin^5(\theta)-26.61\sin^6(\theta)
\labeqn{ypfitfctluxpars}
\end{align}


















%
%(amax+amin)/2 is:
%
%\begin{align}
%  \frac{1}{a}
%  -\frac{k}{2a^2}
%  +\frac{1}{6a^3}
%  +\frac{8+4\sin^{5/2}\theta}{25a^3}k^2
%  -\frac{4+8\sin^{3}\theta}{21a^4}k^3
%  +\mathcal{O}\left(\frac{1}{a^5}\right)
%\end{align}
%
%(amax-amin)/2 is:
%
%\begin{align}
%  \frac{1}{2a^2}
%  +\frac{1}{6a^3}
%  -2k^2\frac{2+\sin^{5/2}\theta}{15a^4}
%  +2k^3\frac{1+\sin^{3}\theta}{3a^5}
%  -2k^2\frac{2+\sin^{5/2}\theta}{35a^5}
%  +\mathcal{O}\left(\frac{1}{a^6}\right)
%\end{align}


% amax:
% power of t - power of x | power of t | coefficient
% 1    | 1  | 1
% 2    | 2  | 1/2
% 2    | 3  | -1/2*xx + 1/3
% 3    | 5  | 4/25*(sinth^(5/2) + 2)*xx^2
% 4    | 6  | -2/15*(sinth^(5/2) + 2)*xx^2 + 1/4*xx
% 4    | 7  | -4/21*(2*sinth^3 + 1)*xx^3
% 5    | 8  | 1/3*(2*sinth^3 + 1)*xx^3 - 1/10*(sinth^(5/2) + 2)*xx^2
% 5    | 9  | 16/315*(13*sinth^(7/2) + 2)*xx^4 - 4/27*(2*sinth^3 + 1)*xx^3 + 4/45*(sinth^(5/2) + 2)*xx^2
% 6    | 10 | -24/175*(13*sinth^(7/2) + 2)*xx^4 + 2/15*(2*sinth^3 + 1)*xx^3
% 6    | 11 | -4/165*(43*sinth^4 + 2)*xx^5 + 48/385*(13*sinth^(7/2) + 2)*xx^4 - 8/33*(2*sinth^3 + 1)*xx^3
% 7    | 12 | 4/45*(43*sinth^4 + 2)*xx^5 - 8/105*(13*sinth^(7/2) + 2)*xx^4 + 1/9*(2*sinth^3 + 1)*xx^3
% 7    | 13 | 64/12285*(311*sinth^(9/2) + 4)*xx^6 - 8/65*(43*sinth^4 + 2)*xx^5 + 48/455*(13*sinth^(7/2) + 2)*xx^4
% 8    | 14 | -32/1323*(311*sinth^(9/2) + 4)*xx^6 + 2/21*(43*sinth^4 + 2)*xx^5 - 24/245*(13*sinth^(7/2) + 2)*xx^4
% ...
%
% amin:
% power of t - power of x | power of t | coefficient
% 1     | 1  | 1
% 2     | 2  | -1/2
% 2     | 3  | -1/2*xx
% 3     | 5  | 4/25*(sinth^(5/2) + 2)*xx^2
% 4     | 6  | 2/15*(sinth^(5/2) + 2)*xx^2 + 1/4*xx
% 4     | 7  | -4/21*(2*sinth^3 + 1)*xx^3 + 4/35*(sinth^(5/2) + 2)*xx^2
% 5     | 8  | -1/3*(2*sinth^3 + 1)*xx^3 - 1/10*(sinth^(5/2) + 2)*xx^2
% 5     | 9  | 16/315*(13*sinth^(7/2) + 2)*xx^4 - 4/9*(2*sinth^3 + 1)*xx^3 - 4/45*(sinth^(5/2) + 2)*xx^2
% 6     | 10 | 24/175*(13*sinth^(7/2) + 2)*xx^4 - 2/15*(2*sinth^3 + 1)*xx^3 - 2/25*(sinth^(5/2) + 2)*xx^2
% 6     | 11 | -4/165*(43*sinth^4 + 2)*xx^5 + 96/385*(13*sinth^(7/2) + 2)*xx^4 + 4/33*(2*sinth^3 + 1)*xx^3
% 7     | 12 | -4/45*(43*sinth^4 + 2)*xx^5 + 8/35*(13*sinth^(7/2) + 2)*xx^4 + 1/3*(2*sinth^3 + 1)*xx^3
% 7     | 13 | 64/12285*(311*sinth^(9/2) + 4)*xx^6 - 8/39*(43*sinth^4 + 2)*xx^5 + 48/455*(13*sinth^(7/2) + 2)*xx^4 + 8/39*(2*sinth^3 + 1)*xx^3
% 8     | 14 | 32/1323*(311*sinth^(9/2) + 4)*xx^6 - 2/7*(43*sinth^4 + 2)*xx^5 - 24/245*(13*sinth^(7/2) + 2)*xx^4 + 2/21*(2*sinth^3 + 1)*xx^3
% ...



\subsection{Secondary extinction}

The BC model for secondary extinction ends up being mathematically similar to
the one for primary extinction, with the main difference being that the
distortion of cross sections captured in $f(\eta)$ now comes from the
description of a mosaic crystal grain rather than from an effective spread
inherent to the diffraction happening inside each coherent domain. Additionally,
the variable $x$ is replaced by a variable $X$ depending on both the mosaic
spreads and the average path length $\bar{T}$ (FIXME: Repeat here in more
detail? Also maybe the language used in the first paragraph in BC1974 II.3 is
better than the one used here??).

\begin{align}
  X = \frac{2}{3}...fixme.
  %FIXME Q or F/V?? Perhaps just define Q in the beginning.
  \labeqn{capitalXdef}
\end{align}

To produce equivalent of \refeqn{yintegral} (BC.36), one must go back to BC.14,
and replace $\sigma(\epsilon_1)$ with the convoluted $\bar\sigma(\epsilon_1)$
values for the three variants given in BC.40a, BC.41a, and BC.44
respectively. This results in

\begin{align}
  f_G(\eta) = \exp\left\{-\frac{9}{16\pi}\eta^2\right\}
  \labeqn{fofetaG}
\end{align}
\begin{align}
  f_L(\eta) = \frac{1}{1+\eta^2}
  \labeqn{fofetaL}
\end{align}
\begin{align}
  % sinc(x) = sin(x)/x
  % eq. 44, sigma(0) = Q*alpha
  %
  % eta = pi * eps1 * beta = (4/3)*pi*eps1*alpha, so pi*eps1*alpha = 3eta/4
  %
  % f(eta) = sigma(eps1(eta))/sigma(0)
  %        = sinc^2( pi*eps1*alpha )
  %        = sinc^2( 3eta/4 )
  f_F(\eta) = \frac{ \sin^2(3\eta/4) } { (3\eta/4)^2 }
  \labeqn{fofetaF}
\end{align}

\begin{align}
  %Note integration limit from 0..inf, so have multiplied factor in front by 2!!
  y_s^G(\theta,X_G) = \frac{3}{2\pi}\int_{0}^{\infty} f_G(\eta)\varphi\left(\theta,X_Gf_G(\eta)\right)d\eta
  \labeqn{ysintegralG}
\end{align}
\begin{align}
  %Note integration limit from 0..inf, so have multiplied factor in front by 2!!
  y_s^L(\theta,X_L) = \frac{2}{\pi}\int_{0}^{\infty} f_L(\eta)\varphi\left(\theta,\frac{4}{3}X_Lf_L(\eta)\right)d\eta
  \labeqn{ysintegralL}
\end{align}
\begin{align}
  %Note integration limit from 0..inf, so have multiplied factor in front by 2!!
  y_s^F(\theta,X_F) = \frac{3}{2\pi}\int_{0}^{\infty} f_F(\eta)\varphi\left(\theta,X_Ff_F(\eta)\right)d\eta
  \labeqn{ysintegralF}
\end{align}


% Factor in front of integral, based on eq. 14, should be:
%
% sigma(0)/Q * (deta/deps1)^(-1) = sigma(0)/ ( Q * pi * ( (4/3)*alphaGL) = ( sigma(0)/(alphaGL*Q) ) * 3/(4pi)
%
% Gauss: ( sigma(0)/(alphaG*Q) ) * 3/(4pi)
%      = ( (alphaG*Q)/(alphaG*Q) ) *  3/(4pi) = 3/(4pi)
% Lorentz: ( sigma(0)/(alphaL*Q) ) * 3/(4pi)
%           = ( ( (4/3) * Q * alphaL) / (alphaL*Q) ) * 3/(4pi)
%           = (4/3) * 3/(4pi) = 1/pi
% Fresnel: ( sigma(0)/(alphaG*Q) ) * 3/(4pi)
%      = ( (alphaG*Q)/(alphaG*Q) ) *  3/(4pi) = 3/(4pi)
%
% As a sanity check, one can let x->0 and we see that the factor should also be
% the normalisation constant of f (since phi(0)=1 and yp(x=0)=1):
% Gauss. This normalisation constant is easily calculated to be:
%
% Gauss: 3/(4pi) (CONSISTENT!!)
% Lorentz: 1/pi (CONSISTENT!!)
% FRESNEL: 3/(4pi) (CONSISTENT!!)

% Factor in front of X inside phi-function should be whatever comes out of
% transforming sigma(eps1)*r to factor*x*f(eta).
%
% x = (2/3)*Q*alpha*tbar
% r = (2/3) * tbar
% eta = pi * eps1 * ( (4/3)*alphaGL)
%
% sigma(eps1)*r = sigma(0)*f(eta)*(2/3)*tbar
%               = f(eta)*(2/3)*sigma(0)*tbar*(x/((2/3)*Q*alpha*tbar))
%               = ( sigma(0)/(Q*alpha) ) * x * f(eta)
% So the factor before x inside the call to phi should be sigma(0)/(Q*alpha):
%
% Gauss and Fresnel have sigma(0)=Q*alpha, so the factor is unity.
% Lorentzian have sigma(0)=(4/3)(Q*alpha so the factor is 4/3.
%

%ACTUALLY $X_F$ is $X_G$....


% Now let us also double-check the argument of the phi function (which is
% sigma(eps1) in eq. 14). in the integrand, and how/if it depends on the
% normalisation of f. In particular it could show up as we go from eq. 14 to 36
% (or the equivalent for the secondary models):
%
% Primary:
%    Argument of phi function should actually be sigma(eps1)*r, which
%    can be written as sigma(0)*f(eta)*r, and sigma(0)*r is given by
%    Q*alphabar*r = Q*alphabar*(tbar/(3/2)) which happens to be exactly the
%    definition of x.
%
% Scnd Lorentz:
%
%    If argument of phi function should also be sigmaL(eps1)*R, written as
%    sigmaL(0)*fL(eta)*R, then we can again look at sigmaL(0)*R:
%
%    sigmaL(0)*R = (4/3)*Q*alphaL*R = (4/3)*Q*alphaL*((2/3)*TBar)
%                = (4/3)* [ (2/3)*Q*alphaL*TBar ] = (4/3)*X
%
% Scnd Gauss:
%
%    If argument of phi function should also be sigmaG(eps1)*R, written as
%    sigmaG(0)*fG(eta)*R, then we can again look at sigmaG(0)*R:
%
%    sigmaG(0)*R = Q*alphaG*R = Q*alphaG*((2/3)*TBar)
%                = (2/3)*Q*alphaG*TBar = X
%
% Scnd Fresnel form: sigmaF(0)=Q*alphaG (note not alphaF!), so argument of phi
% is like the Gaussian case? However integral of (sin(x)/x)^2 from -inf to +inf is pi, not 4pi/3

%%BC provides
%%three models for the shape of the mosaic distributions: Gaussian (G), Lorentzian
%%(L), and a Fresnel distribution (F).
%%
%%the spread in cross-sections in principle a mosaic spread distribution
%%
%%with the exception a more standard mosaic spread function
%%plays the role  that the intrinsic spread  the BC Bla. Gauss/Lorentz/Fresnel, differs by
%%
%%
%%



%%%%%

% Eq. 14 -> 36?
%
% yp = (1/Q)*integral( sigma(eps1)*phi(sigma(eps1)) d(eps1)
%
% with f=sigma(eps1)/sigma(0) we get:
%
% yp = (sigma(0)/Q)*integral( f(eps1)*phi(sigma(eps1)) d(eps1)
%
% For primary in sphere, we have sigma(0)=(3/4)QBeta (eq. 29), so
%
% yp = 3/4Beta*integral( f(eps1)*phi(sigma(eps1)) d(eps1)
%
% Also eta = pi*eps1*beta, so deta = pi*deps1*beta
%
% yp = (3/(4pi))*integral( f(eta)*phi(sigma(eta)) d(eps1*pi*beta)
%
% For secondary extinction, Gaussian, we get:
%
% sigma(0) = Q*alphaG and eta=pi*eps1*(4/3)*alphaG
%
% and d(eta) = d(eps1 * pi*(4/3)*alphaG)

% SO:
% yG = (sigma(0)/Q)*integral( f(eps1)*phi(sigma(eps1)) d(eps1)
%    = 3/(4pi) * integral( f(eps1)*phi(sigma(eps1)) d(eta)
%
% And for Lorentzian we get:
%
% sigma(0) = (4/3)*Q*alphaL, eta=pi*eps1*(4/3)*alphaL
%
% SO:

% yL = (sigma(0)/Q)*integral( f(eps1)*phi(sigma(eps1)) d(eps1)
%    = (4/3)/(pi*(4/3))*integral( f(eps1)*phi(sigma(eps1)) d(eta)
%    = 1/pi * integral( f(eps1)*phi(sigma(eps1)) d(eta)
%
%

% Now let us also double-check the argument of the phi function (which is
% sigma(eps1) in eq. 14). in the integrand, and how/if it depends on the
% normalisation of f. In particular it could show up as we go from eq. 14 to 36
% (or the equivalent for the secondary models):
%
% Primary:
%    Argument of phi function should actually be sigma(eps1)*r, which
%    can be written as sigma(0)*f(eta)*r, and sigma(0)*r is given by
%    Q*alphabar*r = Q*alphabar*(tbar/(3/2)) which happens to be exactly the
%    definition of x.
%
% Scnd Lorentz:
%
%    If argument of phi function should also be sigmaL(eps1)*R, written as
%    sigmaL(0)*fL(eta)*R, then we can again look at sigmaL(0)*R:
%
%    sigmaL(0)*R = (4/3)*Q*alphaL*R = (4/3)*Q*alphaL*((2/3)*TBar)
%                = (4/3)* [ (2/3)*Q*alphaL*TBar ] = (4/3)*X
%
% Scnd Gauss:
%
%    If argument of phi function should also be sigmaG(eps1)*R, written as
%    sigmaG(0)*fG(eta)*R, then we can again look at sigmaG(0)*R:
%
%    sigmaG(0)*R = Q*alphaG*R = Q*alphaG*((2/3)*TBar)
%                = (2/3)*Q*alphaG*TBar = X
%
% Scnd Fresnel form: sigmaF(0)=Q*alphaG (note not alphaF!), so like the Gaussian
% case? However integral of (sin(x)/x)^2 from -inf to +inf is pi, not 4pi/3


\section{Taylor expansions}

USAGE IN RECIPES: Always use Taylor expansions (5th order) for x below a
threshold of 0.1 (std recipe) and 0.01 (lux recipe).

\begin{align}
  y_P(x,\theta)\approx&\,  1 - \tfrac{33}{35}x
+ \tfrac{2369}{5775}(\sin^{5/2}\theta + 2)x^2
 - \tfrac{54120476}{91216125}(1+2\sin^3\theta)x^3\\\nonumber
&+ \tfrac{350769113251}{1924903480500}(2+13\sin^{7/2}\theta)x^4
 - \tfrac{276478446363113}{2846107289025000}(2+43\sin^4\theta)x^5 \\
%& + \tfrac{19564394058822418571}{855860038150930312500}(4+311\sin^{9/2}\theta)x^6\\
%
  y_G(x,\theta)\approx&\,  1 - \tfrac{3\sqrt{2}}{4}x
  + \tfrac{4\sqrt{3}}{15}(2+\sin^{5/2}\theta)x^2
- \tfrac{2}{3}(1+2\sin^3\theta)x^3\\\nonumber
 &+ \tfrac{16\sqrt{5}}{175}(2+13\sin^{7/2}\theta)x^4
 - \tfrac{2\sqrt{6}}{45}(2+43\sin^4\theta)x^5\\
% +\tfrac{64\sqrt{7}}{6615}(4+311\sin^{9/2}\theta)x^6\\
 %
  y_L(x,\theta)\approx&\,  1-x
  + \tfrac{8}{15}(2+\sin^{5/2}\theta)x^2
- \tfrac{80}{81}(1+2\sin^3\theta)x^3\\\nonumber
&+ \tfrac{32}{81}(2+13\sin^{7/2}\theta)x^4
- \tfrac{112}{405}(2+43\sin^4\theta)x^5\\
% +\tfrac{2816}{32805}(4+311\sin^{9/2}\theta)x^6\\
%
  y_F(x,\theta)\approx&\, 1 - x
  + \tfrac{11}{25}(2+\sin^{5/2}\theta) x^2
  - \tfrac{604}{945}(1+2\sin^3\theta)x^3 \\\nonumber
  &+ \tfrac{15619}{79380}(2+13\sin^{7/2}\theta)x^4
   - \tfrac{655177}{6237000}(2+43\sin^4\theta)x^5
%  + \tfrac{27085381}{1094593500}(4+311\sin^{9/2}\theta)x^6
\end{align}

FIXME: How can we refer to these models? Both in this paper and in e.g. NCrystal
where we use them? We can use ``BCK'' (Becker-Coppens-Kittelmann), if such
self-reference is allowed? Or perhaps BC2025(lux|minimal|) vs. BC1974 ?

For ease of reference the three updated recipes provided here, will be referred
to as BCK-minimal, BCK, and BCKLux. The first is as the name implies a minimal,
which only ..... Next, the BCK model aims to provide numerical evaluations at
the $0.1\%$ level, while finally BCKLux provides an even more precise numerical
evaluation at the $10^-8$ level of precision. (define precision!)


NOTE: Mention that this should be evaluated using Horner's method!! (which is
also the default in numpy's polyval and how it has always been done in
NCrystal). (Although Motzkin's method or the Knuth-Eve algorithm is actually
interesting as it can save additional multiplications!!)

\begin{align}
  y^\Delta_P(x)\equiv  y^\pi_P(x)-y^0_P(x)\approx&\, \tfrac{2369}{5775}x^2 - \tfrac{108240952}{91216125}x^3
+ \tfrac{350769113251}{148069498500}x^4 - \tfrac{11888573193613859}{2846107289025000}x^5 \\
%
 y^\pi_G(x)-y^0_G(x)\approx&\, \tfrac{4\sqrt{3}}{15}x^2
- \tfrac{4}{3}x^3
 + \tfrac{208\sqrt{5}}{175}x^4
 - \tfrac{86\sqrt{6}}{45}x^5\\
%
  y^\pi_L(x)-y^0_L(x)\approx&\, \tfrac{8}{15}x^2
- \tfrac{160}{81}x^3
+ \tfrac{416}{81}x^4
- \tfrac{4816}{405}x^5\\
%
  y^\pi_F(x)-  y^0_F(x)\approx&\, \tfrac{11}{25}x^2
  - \tfrac{1208}{945}x^3
  + \tfrac{203047}{79380}x^4
   - \tfrac{28172611}{6237000}x^5
\end{align}





\appendix % if required
\section{Appendix title}

Text text text text text text text text text text text text text text
text text text text text text text.

\subsection{Appendix subsection title}

Text text text text text text text text text text text text text text
text text text text text text text.

\subsubsection{Appendix subsubsection title}

Text text text text text text text text text text text text text text
text text text text text text text.


\begin{acknowledgements}
The contributions of non-authors etc. should be given here.
\end{acknowledgements}

\begin{funding}
List funding organizations, recipients, grant numbers, etc.
\end{funding}

\ConflictsOfInterest{Please declare any conflicts of interest, or declare  that there are no conflicts of interest.
}

\DataAvailability{Please state how the data supporting the results reported in your article can be accessed, e.g. within the article, as published supporting material, in repositories, upon request...
}

\bibliography{iucr} % basename of .bib file

\end{document}
%%%%%%%%%%%%%%%%%%%%%%%%%%%%%%%%%%%%%%%%%%%%%%%%%%%%%%%%%%%%%%%%%%%%%%%%%%%%%%


% Sagemath timings: Unclear, but lowx_taylor was very long, perhaps >10 hours.

% createdata Timings: (total 15.5 hours with ~14 processes):

%%%%%%%%%%%%%%%%%%%%%%%%%%%%%%%%%%%%%%%%%
% Wrote bcdata.json ("1919/1919 [6:58:50<00:00, 13.10s/it]")
%
% real  418m51.385s
% user  5734m52.315s
% sys  0m58.339s
%
%%%%%%%%%%%%%%%%%%%%%%%%%%%%%%%%%%%%%%%%%
% Wrote bcdata_table1pts.json ("364/364 [1:27:30<00:00, 14.42s/it]")
%
% real  87m30.635s
% user  1168m6.251s
% sys  0m28.893s
%
%%%%%%%%%%%%%%%%%%%%%%%%%%%%%%%%%%%%%%%%%
% Wrote bcdata_thetascan.json ("2534/2534 [6:42:33<00:00,  9.53s/it]")
%
% real  402m34.514s
% user  5603m21.985s
% sys  0m22.534s
%


%BC x:
%
%r = radius of ideal spherical crystal
%
%x = sigma(0)*r = 3/4 Q*beta*r = 2/3 Q alphabar tbar
%    alphabar = (3/2)r sin2theta/lambda
%    beta = 2r sin2theta / lambda
%   so x = 3/4 Q* 2r sin2theta / lambda  *r = 3/2 Q * r^2 * Q * sin2theta
%   /lambda =
%    Q = ?

%  (3/2)*sphere_r^2 = (3/2)*(AVPATHLENGTH*3/4)^2

%
%    NCrystal code:
%%    bc_x =         wl^2*F^2/V^2 * ( 2 / 3 ) * blksz^2
%%    sabine_x =     wl^2*F^2/V^2 * blksz^2
%
%

%sabine_x =

%block_size =

% Average path length through a sphere of radius is (4/3)*radius = (2/3)*diameter.
% so radius = 3/4 * <average_pathlength>

%If we use blksz_Aa to mean the Average path length through the crystallite, we
%have to use r = 4/3 * ....

% BC x: factor of (2/3) is when using l to mean the average path length through
% the crystallite.

%Sabine x:


% shuqi BC xp:
% double Q_theta = NC::ncsquare(Nc * wl * F_hkl) * wl; //division by sin_2theta to be done later

% double xp = 2. / 3. * NC::ncsquare(Nc * wl * F_hkl) * l * l;



%BC x:
%
%35a:
%
%x = (2/3)*Q*alphabar*tbar
%alphabar = tbar * sin2theta/wl (eq. 35b with 3r/2 = tbar)
%tbar: average path length through the crystallite (tbar=3/2*radius for a sphere).
%     Q = |F/V|^2*wl^3/sin2theta
%
%so x = (2/3)*(|F/V|^2*wl^3/sin2theta)*(tbar * sin2theta/wl)*tbar
%     = (2/3)* |F/V|^2*wl^2 * tbar^2
%
%
% Sabine x (2006 paper):
%
%x = Qk T C D (6.4.5.7)  (C=1 for no absorption)
%T: average path length through crystallite (so (3/2)*radius for a sphere)
%
%
%Sabine x Shuqi:
%  double xp = NC::ncsquare(Nc * wl * F_hkl * l);
%
%Sabine x for spheres (1988 paper):
%
%x = Qk T D (middle of left column, p371)
%Qk = |F/V|^2*wl^2 / sintheta (top right column, p370)
%BUT NOTE: Qtheta = |F/V|^2*wl^3 / sin2theta ``on the glancing-angle scale''(top right column, p370) (FOR SECONDARY EXTINCTION????)
%
%371 notes for spheres T=D=(3/2)*radius
%
% So x = |F/V|^2*wl^2 / sintheta * (3radius/2)^2


%Nice formulation: "... the coherently scattering domain size, referred to here as the crystallite size for readability."

%Could we somehow use "Multiple Bragg reflection by a thick mosaic
%crystal. II. Simplified transport equation solved on a grid, Wuttke 2020" ?

%wilson pgs 41+42: 0.93, 0.94, 1.0 => O(1-6%) mistake if modelling a cubic
%domain as spherical. ??

%wilson eq 9-12, is how we can see that T = (3/4) * diameter = (3/2) * radius
%for a sphere. The "apparent particle size" is the volume average of the path
%length. Check also eq. 19 for how this could be used to take hkl anisotropy
%into account.

%Discuss also why we focus on Sabine + BC (because it is discussed in e.g. the
%BC paper that CR, Zach, etc. are incorrect).

%ARGH: Trying to "spherify" Sabine's El and Eb makes the result go FURTHER from
%BC, not closer. One solution is to look at low-x, and apply a correction to x
%in the sabine model, which makes the constant in the first order term of yp(x)
%~= 1-const*x match up. This turns out to be a factor of 19/15 ~= 1.26666,
%meaning that the block sizes of Sabine are essentially modified by
%sqrt(19/15)~=1.125 (very close to 9/8). The nice thing in practice here is how
%we would know that the two models have similar behaviour at low x.

%FIXME: Lorentz breakdown plot shows a breakdown of the updated-classic recipe,
%       which is missed in the standard cmprecipes plot, since it only occurs
%       for 45<theta<46. Solution: Rerun with the point at 45 replaced with
%       44.99 and 45.01 ? Or add those as extra points?


% FIXME: NOTICE THAT theta=90 and theta=pi are inconsistent. Once case is
% theta_bragg and one is theta_scatter=2*theta_bragg. But we are using pi as
% index on functions etc. to indicate backscattering. There is not direct easy
% fix, since theta in sintheta = lambda/2d is thetabragg, and theta=pi is
% theta_scatter. Solution: Use theta_bragg everywhere, but for convenience use
% pi as sub/super-script on functions to indicate 2*thetabragg=pi.


%OUTLOOK/ discussion: BC models used a lot, but the issues is by all means not
%settled (e.g Takagi's equations versus non-wave transfer equations). However,
%the aim of the present work is not to settle the field once and for all, but to
%ensure that the community has better recipes for applying one of the better
%developed models out there. (NB:
