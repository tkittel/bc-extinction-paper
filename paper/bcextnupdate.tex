%------------------------------------------------------------------------------
% Template file for the submission of articles to IUCr journals in LaTeX2e
% using the iucrjournals document class (file iucrjournals.cls)
% This work has been dedicated to the public domain
% License: CC0 1.0 Universal
% https://creativecommons.org/publicdomain/zero/1.0/
%------------------------------------------------------------------------------
% This template file and associated class and style files produce documents in
% a preprint style suitable for submission and review purposes.
% The iucrjournals.cls requires a small selection of packages from standard TeXLive
% distributions and contains a minimal set of macros to define content and apply
% formatting. BibTeX and iucr.bst should be used for references (using harvard.sty).
% If you wish to use additional packages, please reference them in this document and
% please only use packages included in standard TeXLive distributions in order to
% avoid compilation problems during the submission process.
%------------------------------------------------------------------------------

\documentclass{iucrjournals}

% Add extra packages here, e.g.
% \usepackage{myfavouritepackage}
\usepackage{amssymb}
\usepackage[fleqn]{amsmath}
\newcommand{\labtab}[1]{\label{tab:#1}}
\newcommand{\reftabnumonly}[1]{\ref{tab:#1}}
\newcommand{\reftab}[1]{Table~\reftabnumonly{#1}}
\newcommand{\Reftab}[1]{Table~\reftabnumonly{#1}}%same as \reftab, but always capitalised!
%%%%\newcommand{\lablisting}[1]{\label{lst:#1}}
%%%%\newcommand{\reflistingnumonly}[1]{\ref{lst:#1}}
%%%%\newcommand{\reflisting}[1]{Listing~\reflistingnumonly{#1}}
%%%%\newcommand{\Reflisting}[1]{Listing~\reflistingnumonly{#1}}%same as \reflisting, but always capitalised!
\newcommand{\labeqn}[1]{\label{eqn:#1}}
\newcommand{\refeqnnumonly}[1]{\ref{eqn:#1}}
\newcommand{\refeqn}[1]{Eq.~\refeqnnumonly{#1}}
\newcommand{\Refeqn}[1]{Eq.~\refeqnnumonly{#1}}%same as \refeqn, but always capitalised!
\newcommand{\reftwoeqns}[2]{Eqs.~\refeqnnumonly{#1} and \refeqnnumonly{#2}}
\def\LRA{\Leftrightarrow}
\def\RA{\Rightarrow}

\title{Article title}

% Authors and affiliations (uses the standard authblk package):
% Author affiliations are indicated by lowercase letters in square brackets in the \author macro.
% Affiliations (referenced by the lowercase letters in square brackets) are listed after all the authors have been defined.
% The email addresses of corresponding/contact authors can be included using:
% \IUCrCemaillink{corrauthor@org.org}
% Other co-author email addresses can be included using:
% \IUCrEmaillink{coauthor@org.org}
% ORCiDs can be included using:
% \IUCrOrcidlink{xxxx-xxxx-xxxx-xxxx}
% Author footnotes can be included using:
% \IUCrAufn{Text...}
% and to apply the same footnote to another author use:
% \IUCrAufn[1]{}
% where the number in square brackets refers to the numerical order of the
% previously defined footnote.
% For example:
% \author[a]{Anne Author\IUCrCemaillink{corrauthor@org.org}\IUCrOrcidlink{xxxx-xxxx-xxxx-xxxx}}


\author[a]{Thomas Kittelmann\IUCrCemaillink{thomas.kittelmann@ess.eu}\IUCrOrcidlink{0000-0002-7396-4922}}%
\author[b]{Secundus Segunda\IUCrEmaillink{coauthor@org.org}\IUCrOrcidlink{xxxx-xxxx-xxxx-xxxx}\IUCrAufn{Unique note.}}%
\author[a,b]{Trinity Terzi\IUCrCemaillink{anothercorrauthor@org.org}\IUCrOrcidlink{xxxx-xxxx-xxxx-xxxx}\IUCrAufn{Shared note.}}%
\author[a,b]{Clover Dufour\IUCrEmaillink{anothercoauthor@org.org}\IUCrAufn[2]{}}

\affil[a]{ESS DMSC, European Spallation Source ERIC, Denmark}
\affil[b]{Different Department, Different Organization, ..., Country }

\begin{document}
\maketitle

\begin{synopsis}
One or two sentences suitable for the Journal contents listing and use in promoting your article via social media, highlighting the findings and significance of your work.
\end{synopsis}

\begin{abstract}
Single paragraph stating as specifically and as quantitatively as possible the principal results obtained, and providing an indication of the broader significance of the work. The abstract should be capable of being understood on its own without access to the text or figures.
\end{abstract}

\keywords{ Three or four key words/phrases separated by semi-colons. }


\section{Section title}

Text text text text text text text text text text text text text text
text text text text text text text \cite{knuth84,lamport86}.

\begin{figure}[ht] %
\label{fig:figure1}
\begin{center}
\includegraphics[width=0.5\textwidth]{fig1.png} % NB use pdflatex for non-postscript
\end{center}
\caption{Caption \protect\cite{knuth84}} % NB \protect\cite{...} is required in floating figures
\end{figure}

\subsection{Subsection title}

Text text text text text text text text text text text text text text
text text text text text text text.

\subsubsection{Subsubsection title}

Text text text text text text text text text text text text text text
text text text text text text text.



\section{Section title}

Text text text text text text text text text text text text text text
text text text text text text text.

\subsection{Subsection title}

Text text text text text text text text text text text text text text
text text text text text text text.

\subsubsection{Subsubsection title}

Text text text text text text text text text text text text text text
text text text text text text text.

% Basic table

\begin{table}[ht]
\caption{Caption to table \protect\cite{lamport86}} % NB \protect\cite{...} is required in floating tables
\smallskip
\begin{center}
\begin{tabular}{llcr}
\midrule
 HEADING    & FOR        & EACH       & COLUMN     \\
\midrule
 entry      & entry      & entry      & entry      \\
 entry      & entry      & entry      & entry      \\
 entry      & entry      & entry      & entry      \\
\end{tabular}
\end{center}
\end{table}

\section{Dumping equations}

NB: $\theta$ is the Bragg angle, defined by:
\begin{align}
  \sin\theta = \frac{\lambda}{2d_{hkl}}
  \labeqn{braggequation}
\end{align}

And $x$ is defined by (fixme: is $l$ the sphere radius?):

\begin{align}
  x = \frac{ 2l^2\lambda^2F^2}{3V^2}
  \labeqn{xdef}
\end{align}

\begin{align}
  f(\eta) = \frac{\eta^2-\eta\sin(2\eta)+sin^2(\eta)}{\eta^4}
  \labeqn{fofeta}
\end{align}

\begin{align}
  \varphi^0(s) = \frac{3}{64s^3}\left\{8s^2+4s\exp(-4s)-(1-\exp(-4s))\right\}
  \labeqn{phizero}
\end{align}

\begin{align}
  \varphi^\pi(s) = \frac{3}{4s^3}\left\{s^2-s+\frac{1}{2}\log(1+2s)\right\}
  \labeqn{phipi}
\end{align}
(NOTE the mistake of factor of 1/2 in original equations)

\begin{align}
  \varphi(\theta,s) = \varphi^0(s) + \sin^{3/2}\theta\left\{\varphi^\pi\left(s\sqrt{\sin\theta}\right)-\varphi^0\left(s\sqrt{\sin\theta}\right)\right\}
  \labeqn{phi}
\end{align}

\begin{align}
  y_{p}(\theta,x) = \frac{3}{2\pi}\int_{0}^{\infty} f(\eta)\varphi\left(\theta,xf(\eta)\right)d\eta
  \labeqn{ypintegral}
\end{align}

(note we have used that $f(\eta)$ is an even function to change the integration
domain to $[0,\infty)$).


\subsection{Old recipes}

\begin{align}
  y_{p}(\theta,x)=\left\{1+2x+\frac{A(\theta)x^2}{1+B(\theta)x}\right\}
  \labeqn{ypfitfctorig}
\end{align}

\begin{align}
  A(\theta) &= 0.20 + 0.45\cos(2\theta) \\
  B(\theta) &= 0.22 - 0.12(0.5-\cos(2\theta))^2
  \labeqn{ypfitfctparsorig}
\end{align}

\subsection{New recipes}

\begin{align}
  A(\theta) &= 0.559+0.537\cos(2\theta)\nonumber\\
  B(\theta) &= 0.604-0.222(0.5-\cos(2\theta))^2
\labeqn{ypfitfctparsupdated}
\end{align}

\begin{align}
  y_{p}(\theta,x)=\left\{1+2x+\frac{A(\theta)x^2-0.1x}{1+B(\theta)x+C(\theta)x(1+x)^{-1} }\right\}^{-1/2}
  \labeqn{ypfitfctlux}
\end{align}

%\begin{align}
%  A(\theta) &=& \sum_{i=0}^6p^A_i(\sin\theta)^i \\
%  B(\theta) &=& \sum_{i=0}^6p^B_i(\sin\theta)^i \\
%  C(\theta) &=& \sum_{i=0}^6p^C_i(\sin\theta)^i \\
%  \labeqn{ypfitfctluxparsFIXME}
%\end{align}

%\begin{align}
%A(\theta) =& 0.8968-0.2928\sin(\theta)+2.528\sin^2(\theta)-11.21\sin^3(\theta)\nonumber\\
%           & +21.07\sin^4(\theta)-20.17\sin^5(\theta)+7.32\sin^6(\theta)\nonumber\\
%B(\theta) =& 0.4697-0.1688\sin(\theta)+5.37\sin^2(\theta)-20.68\sin^3(\theta)\nonumber\\
%           & +40.52\sin^4(\theta)-38.09\sin^5(\theta)+12.89\sin^6(\theta)\nonumber\\
%C(\theta) =& -0.7486-0.5671\sin(\theta)-1.248\sin^2(\theta)+17.56\sin^3(\theta)\nonumber\\
%           & -47.8\sin^4(\theta)+59.14\sin^5(\theta)-26.61\sin^6(\theta)
%  \labeqn{ypfitfctluxpars}
%\end{align}
%
%
%\begin{align}
%A(\theta) =&
%0.89677-0.2928\sin(\theta)+2.5285\sin^2(\theta)-11.207\sin^3(\theta)\nonumber\\
%           &+21.068\sin^4(\theta)-20.166\sin^5(\theta)+7.32\sin^6(\theta)\nonumber \\
%B(\theta) =&
%0.46968-0.16882\sin(\theta)+5.3704\sin^2(\theta)-20.675\sin^3(\theta)\nonumber\\
%           &+40.518\sin^4(\theta)-38.086\sin^5(\theta)+12.887\sin^6(\theta)\nonumber \\
%C(\theta) =&
%-0.7486-0.5671\sin(\theta)-1.2484\sin^2(\theta)+17.56\sin^3(\theta)\nonumber\\
%           &-47.804\sin^4(\theta)+59.144\sin^5(\theta)-26.606\sin^6(\theta)
%  \labeqn{ypfitfctluxpars}
%\end{align}
%
\begin{align}
  A(\theta) &= 0.8968-0.2928\sin(\theta)+2.528\sin^2(\theta)-11.21\sin^3(\theta)\nonumber\\
            &\hspace*{1em} +21.07\sin^4(\theta)-20.17\sin^5(\theta)+7.32\sin^6(\theta)\nonumber\\
  B(\theta) &= 0.4697-0.1688\sin(\theta)+5.37\sin^2(\theta)-20.68\sin^3(\theta)\nonumber\\
            &\hspace*{1em} +40.52\sin^4(\theta)-38.09\sin^5(\theta)+12.89\sin^6(\theta)\nonumber\\
  C(\theta) &= -0.7486-0.5671\sin(\theta)-1.248\sin^2(\theta)+17.56\sin^3(\theta)\nonumber\\
            &\hspace*{1em} -47.8\sin^4(\theta)+59.14\sin^5(\theta)-26.61\sin^6(\theta)
\labeqn{ypfitfctluxpars}
\end{align}

\subsection{Tail integration}

For positive $\eta$ values we have:

\begin{align}
  f(\eta) \le f^+(\eta) \equiv \frac{\eta^2+\eta+1}{\eta^4}
  \labeqn{fplus}
\end{align}
\begin{align}
  f(\eta) \ge f^-(\eta) \equiv \frac{\eta^2-\eta}{\eta^4}
  \labeqn{fminus}
\end{align}

Where $f^-(\eta)$ is only strictly positive when $\eta>1$, so we will restrict
ourselves to $\eta>2$ (??)

Since $\varphi(\theta,s)$ and $f(\eta)$ decreases monotonically for increasing
$s$ and $\eta$ respectively for all positive arguments, we can conclude that:

\begin{align}
\int_{a}^{\infty} f(\eta)\varphi\left(\theta,xf(\eta)\right)d\eta \le
\int_{a}^{\infty} f^+(\eta)\varphi\left(\theta,xf^-(\eta)\right)d\eta \equiv T^+(\theta,x,a) \nonumber\\
\int_{a}^{\infty} f(\eta)\varphi\left(\theta,xf(\eta)\right)d\eta \ge
\int_{a}^{\infty} f^-(\eta)\varphi\left(\theta,xf^+(\eta)\right)d\eta \equiv T^-(\theta,x,a)
\end{align}

Using $k=x/a$ and ensuring $a>\max(x,2)$ we have $k<1$.

With the help of the SageMath (REF) Computer Algebra System, $T^\pm(\theta,x,a)$
can be evaluated analytically and expressed as a Taylor expansion in inverse
powers of $a$:

\begin{align}
  T^+(\theta,x,a) = \frac{1}{a}  + \frac{1-k}{2a^2 }
  +\left[\frac{1}{3} + \frac{8+4\sin^{5/2}\theta}{25}k^2  \right]\frac{1}{a^3}
  +\mathcal{O}\left(\frac{1}{a^4}\right)
\end{align}

\begin{align}
  T^-(\theta,x,a) = \frac{1}{a}  - \frac{1+k}{2a^2 }
  +\frac{8+4\sin^{5/2}\theta}{25}k^2\frac{1}{a^3}
  +\mathcal{O}\left(\frac{1}{a^4}\right)
\end{align}

Varying both $k$ and $\sin\theta$ over values the unit interval and evaluating
Taylor coefficients for terms up to $1/a^{20}$ numerically, it is seen that the
maximal absolute numerical value of coefficients is always within a factor of
$2.50$ when compared to the previous order. Thus, if we restrict ourselves to
$a>10$, we can conservatively limit the maximal contributions of the terms of
order $1/a^4$ or higher in REFEQNABOVE to be at most $10/a^4$. With that, we can
use $(T^+(\theta,x,a)+T^-(\theta,x,a))/2$ as an estimate of the tail integral
value, while $(T^+(\theta,x,a)-T^-(\theta,x,a))/2$ will be a conservative upper
bound of the error on the estimate. In other words, for $a>\max(10,x)$ we get:

\begin{align}
\int_{a}^{\infty} f(\eta)\varphi\left(\theta,xf(\eta)\right)d\eta \approx
\frac{1}{a}  - \frac{k}{2a^2 }
  +\left[\frac{1}{6} + \frac{8+4\sin^{5/2}\theta}{25}k^2  \right]\frac{1}{a^3}
  \pm
\left(\frac{1}{2a^2 }+\frac{1}{6a^3}+\frac{10}{a^4}\right)
\labeqn{tailintegralpenultimate}
\end{align}
Where the term in the final parenthesis represents an upper bound on the
error. Due to the leading factor of $1/a^2$ in the error term, there is not much
value in keeping the higher order terms of \refeqn{tailintegralpenultimate}. Thus, by
using $k=x/a$ and $a>10$ and a slight additional pessimisation of the error bound, we
can arrive at the following simple formula for the estimation of the tail of the
integral:
\begin{align}
\int_{a}^{\infty} f(\eta)\varphi\left(\theta,xf(\eta)\right)d\eta \approx
\frac{1}{a}  - \frac{x}{2a^3 } \pm\left(\frac{1}{2a^2 }+\frac{2}{a^3}\right),\quad{}a>10,\,a>x>0
\labeqn{tailintegral}
\end{align}

\subsection{Central integration}

With the integrand implemented via mpmath (REF), and a
Gauss-Legendre numerical integration, we integrate a particular integral
$[0,a]$. However, for reliable performance it is crucial that the it is 

as the integrand contains trigonometic functions in REFFOFETA,
it is important that each

\subsection{Direct evaluation at low $x$ via Taylor expansions}


\subsection{Integration strategy}



At a particular $(\theta,x)$ point, the strategy to evaluate $y_p(\theta,x)$
within a target precision of $\epsilon=10^{-7}$ is described in the following.

First, for $x<1$, it is investigated if the direct Taylor expansion method
described in REFSECTION is suitable to achieve the required precision. If it is
not, the integral is solved by a combination of high-precision Gauss-Legendre
numerical integration over the interval $[0,a]$ while the contribution over the
interval $[a,\infty]$ is estimated by \refeqn{tailintegral}. The only problem
that remains is which value of $a$ to choose in this scenario.


strategy: $\delta{}a$ is $\max(100,x)$, rounded up to 
For increasing values of i, set a=step*i and evaluate via a Gauss-Legendre
integration the integral over $[0,a]$. Check if combining this result with a
tail integration over $[a,\infty]$ would provide a final result with suitable
low error. If not, increase $a$ by $\Delta{}a$



%
%(amax+amin)/2 is:
%
%\begin{align}
%  \frac{1}{a}
%  -\frac{k}{2a^2}
%  +\frac{1}{6a^3}
%  +\frac{8+4\sin^{5/2}\theta}{25a^3}k^2
%  -\frac{4+8\sin^{3}\theta}{21a^4}k^3
%  +\mathcal{O}\left(\frac{1}{a^5}\right)
%\end{align}
%
%(amax-amin)/2 is:
%
%\begin{align}
%  \frac{1}{2a^2}
%  +\frac{1}{6a^3}
%  -2k^2\frac{2+\sin^{5/2}\theta}{15a^4}
%  +2k^3\frac{1+\sin^{3}\theta}{3a^5}
%  -2k^2\frac{2+\sin^{5/2}\theta}{35a^5}
%  +\mathcal{O}\left(\frac{1}{a^6}\right)
%\end{align}


% amax:
% power of t - power of x | power of t | coefficient
% 1    | 1  | 1
% 2    | 2  | 1/2
% 2    | 3  | -1/2*xx + 1/3
% 3    | 5  | 4/25*(sinth^(5/2) + 2)*xx^2
% 4    | 6  | -2/15*(sinth^(5/2) + 2)*xx^2 + 1/4*xx
% 4    | 7  | -4/21*(2*sinth^3 + 1)*xx^3
% 5    | 8  | 1/3*(2*sinth^3 + 1)*xx^3 - 1/10*(sinth^(5/2) + 2)*xx^2
% 5    | 9  | 16/315*(13*sinth^(7/2) + 2)*xx^4 - 4/27*(2*sinth^3 + 1)*xx^3 + 4/45*(sinth^(5/2) + 2)*xx^2
% 6    | 10 | -24/175*(13*sinth^(7/2) + 2)*xx^4 + 2/15*(2*sinth^3 + 1)*xx^3
% 6    | 11 | -4/165*(43*sinth^4 + 2)*xx^5 + 48/385*(13*sinth^(7/2) + 2)*xx^4 - 8/33*(2*sinth^3 + 1)*xx^3
% 7    | 12 | 4/45*(43*sinth^4 + 2)*xx^5 - 8/105*(13*sinth^(7/2) + 2)*xx^4 + 1/9*(2*sinth^3 + 1)*xx^3
% 7    | 13 | 64/12285*(311*sinth^(9/2) + 4)*xx^6 - 8/65*(43*sinth^4 + 2)*xx^5 + 48/455*(13*sinth^(7/2) + 2)*xx^4
% 8    | 14 | -32/1323*(311*sinth^(9/2) + 4)*xx^6 + 2/21*(43*sinth^4 + 2)*xx^5 - 24/245*(13*sinth^(7/2) + 2)*xx^4
% ...
%
% amin:
% power of t - power of x | power of t | coefficient
% 1     | 1  | 1
% 2     | 2  | -1/2
% 2     | 3  | -1/2*xx
% 3     | 5  | 4/25*(sinth^(5/2) + 2)*xx^2
% 4     | 6  | 2/15*(sinth^(5/2) + 2)*xx^2 + 1/4*xx
% 4     | 7  | -4/21*(2*sinth^3 + 1)*xx^3 + 4/35*(sinth^(5/2) + 2)*xx^2
% 5     | 8  | -1/3*(2*sinth^3 + 1)*xx^3 - 1/10*(sinth^(5/2) + 2)*xx^2
% 5     | 9  | 16/315*(13*sinth^(7/2) + 2)*xx^4 - 4/9*(2*sinth^3 + 1)*xx^3 - 4/45*(sinth^(5/2) + 2)*xx^2
% 6     | 10 | 24/175*(13*sinth^(7/2) + 2)*xx^4 - 2/15*(2*sinth^3 + 1)*xx^3 - 2/25*(sinth^(5/2) + 2)*xx^2
% 6     | 11 | -4/165*(43*sinth^4 + 2)*xx^5 + 96/385*(13*sinth^(7/2) + 2)*xx^4 + 4/33*(2*sinth^3 + 1)*xx^3
% 7     | 12 | -4/45*(43*sinth^4 + 2)*xx^5 + 8/35*(13*sinth^(7/2) + 2)*xx^4 + 1/3*(2*sinth^3 + 1)*xx^3
% 7     | 13 | 64/12285*(311*sinth^(9/2) + 4)*xx^6 - 8/39*(43*sinth^4 + 2)*xx^5 + 48/455*(13*sinth^(7/2) + 2)*xx^4 + 8/39*(2*sinth^3 + 1)*xx^3
% 8     | 14 | 32/1323*(311*sinth^(9/2) + 4)*xx^6 - 2/7*(43*sinth^4 + 2)*xx^5 - 24/245*(13*sinth^(7/2) + 2)*xx^4 + 2/21*(2*sinth^3 + 1)*xx^3
% ...





\appendix % if required
\section{Appendix title}

Text text text text text text text text text text text text text text
text text text text text text text.

\subsection{Appendix subsection title}

Text text text text text text text text text text text text text text
text text text text text text text.

\subsubsection{Appendix subsubsection title}

Text text text text text text text text text text text text text text
text text text text text text text.


\begin{acknowledgements}
The contributions of non-authors etc. should be given here.
\end{acknowledgements}

\begin{funding}
List funding organizations, recipients, grant numbers, etc.
\end{funding}

\ConflictsOfInterest{Please declare any conflicts of interest, or declare  that there are no conflicts of interest.
}

\DataAvailability{Please state how the data supporting the results reported in your article can be accessed, e.g. within the article, as published supporting material, in repositories, upon request...
}


\bibliography{iucr} % basename of .bib file


\end{document}
%%%%%%%%%%%%%%%%%%%%%%%%%%%%%%%%%%%%%%%%%%%%%%%%%%%%%%%%%%%%%%%%%%%%%%%%%%%%%%
